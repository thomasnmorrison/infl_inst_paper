% File to write introduction

\section{Introduction} \label{sec:intro}

%We would like to be able extract information about fundemental physics from inflation.
%The approach we take is to conduct a numerical experiment.
%We choose a feature of interest and smoothly embed it into a baseline potential.
%The setup is akin to an in/out process where the attractor solution of the baseline potential provides the in state, the potential feature provides the interaction, and the baseline attactor again provides a comparison against which deviations in the out state are measured.
%The quantity of interest is the excess production of $\zeta$ and its non-Gaussian (NG) component in particular.

%The way to think of this potential is not as a theory of inflation in its own right, but rather as cutting one feature out of a potentially feature rich potential, about which we remain largely agnostic, and transplanting it onto a simple and otherwise featureless background where it can be studied in isolation.

% Lead in and setting within the broader picture
\textcolor{red}{[Stuff about context and establishing a narrative.]}
% Establish the narrative of a subdominant independent NG component.
% What's the narrative:
% Embedding inflation into a high energy theory generically predicts multiple fields
% The interactions of multiple fields opens a range of phenomina

% Independent subdominant NG component and an underlying Gaussian field.
% $\zeta$ having a Gaussian component, then at some point during inflation there is some interaction with the fields, this interaction sources a NG component to $\zeta$ which lies underneath the Gaussian one



% Set the scope
In the particular case on which we focus in this paper symmetry is broken in a single transverse field, with the symmetry being restored before the field as descends to the new minima and completes the phase transition. Over a wide range of parameters inflation continues through this process, meaning the symmetry breaking is not restricted to occur at the end of inflation, but may rather be at any point along the potential allowed by the observational constraints. This case lies within the broader context of the role of the phenomenology of transerves fields to generating NG. In addition to highlighting the concentrations of NG generated by an incomplete phase transition, this work should also be thought of as a proof of concept for how the broader context can be approached.   

A more thorough characterization of the parameter space and differential response of $\zeta$ is left to a future paper. As are numerious other avenues of exploration, including completion of the phase transition, asymmetries in the potential and initial conditions, and the extension to multiple transverse fields.

% Paper organization
In Sec. \ref{sec:setup} we provide details on our lattice simulation, equations of motion, and the setting of initial conditions, and the relation of quantum mechanics to the semiclassical calculation done on the lattice. The most technical of these details have been relegated to the appendix. We discuss the form of potential which we split into two parts: a baseline $V_0$ and a feature $\Delta V$. This split will prove useful when determining the NG contribution of $\zeta$. We conclude section with a discussion on the production of $\zeta$, paying attention to the inclusion of non-linear source terms and the limitations imposed by the assumptions of our lattice simulation.

In Sec. \ref{sec:results} we detail the results of our lattice simulations and introduce a formalism in which $\zeta$ is compared to a baseline $\zeta_0$ calculated from identical initial conditions, but on the baseline potential $V_0$. The difference $\Delta\zeta=\zeta-\zeta_0$ is the response of the system to incomplete phase transition we induce with $\Delta V$, and provides a useful proxy for the NG component of $\zeta$. We find $\Delta\zeta$ to be highly NG, forming local concentrations in real space. A measure of NG is made by calculating $\zeta$ peak statistics which are compared to the Gaussian expectation. We find the distribution of maxima has a heavy tail with the density of peaks far in excess of the expectation for a Gaussian field for peaks greater than $~5$ standard deviations. We make a study of the field dynamics relavent to the sourcing of $\Delta\zeta$, where we find \textcolor{red}{some stuff about extreme trajectories}.

In Sec. \ref{sec:conclusion} we provide discussion and concluding remarks.
% concluding remarks
% situation of this paper within the broader context of physics informed searches for NG
