% File to write introduction

\section{Introduction} \label{sec:intro}

\marginpar{
  This section is mostly a place holder for now. Still needs to situate the paper with connection to the literature.
}

We would like to be able extract information about fundemental physics from inflation.
The approach we take is to conduct a numerical experiment.
We choose a feature of interest and smoothly embed it into a baseline potential.
The setup is akin to an in/out process where the attractor solution of the baseline potential provides the in state, the potential feature provides the interaction, and the baseline attactor again provides a comparison against which deviations in the out state are measured.
The quantity of interest is the excess production of $\zeta$ and its non-Gaussian (NG) component in particular.

The way to think of this potential is not as a theory of inflation in its own right, but rather as cutting one feature out of a potentially feature rich potential, about which we remain largely agnostic, and transplanting it onto a simple and otherwise featureless background where it can be studied in isolation.

In this paper we restrict ourselves to single isolated features in the potential. However, there are interesting questions to be asked about the possibility of correlated signals from a sequence of features. Such a scenario could be studied in much the same way.

\marginpar{
  Could also mention the case of multiple features can be treated by choice of initial conditions.
}

% Breakdown of the sections of this paper
The remainder of this paper is organized as follows: in section \ref{sec:setup} we outline the equations which are sovled on the lattice and give a functional form to our potential feature, in section \ref{sec:results} we detail the results of our simulation both in terms of field dynamics and generation of NG, in section \ref{sec:conclusion} we provide concluding remarks.
