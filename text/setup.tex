% Setting up the problem

\section{Setup} \label{sec:setup}
% This would fit more in the potential subsection
We are interested in studying the effects of a localized feature in the inflationary potential.
To this end it will prove useful to split the potential as a sum of two terms $V(\phi^A) = V_0(\phi^A) + \Delta V(\phi^A)$.
This split can be though of as starting with a baseline system and applying a stimulus by deforming the potential, $V_0$ being the baseline and $\Delta V$ being the stimulus.
The system's response can be calculated by taking the difference between two lattice simulations, one run with the deformed potential $V_0+\Delta V$ and the other with the baseline potential $V_0$, with both runs starting from identical initial conditions.
Of particular interest in the cosmological context is the NG component of $\zeta$ sourced by this response.

In this section we setup our system, detailing the equations of motion solved on the lattice, the setting of initial conditions, the sourcing of $\zeta$, and the form of the potential considered.

\subsection{Equations of Motion}
% Open with introduction to lattice sims 
% Define $a$, $R$, $g$

The action for a set of scalar fields minimally coupled to gravity is given by
\begin{equation} \label{eq:action}
  S = \int \sqrt{-g}\left\{\frac{1}{2}\mpl^2R - \frac{1}{2}\sum_Ag^{\mu\nu}\partial_\mu\phi^A\partial_\nu\phi^A - V \right\}\dd^4x.
\end{equation}
Where $\phi^A$ for $A=1,2,...,N_\mathrm{fld}$ are scalar fields and $\mpl$ is the reduced Planck mass. 
Substuting the ansatz of a Friedman-Robertson-Walker (FRW) metric $\dd s^2 = a^2(\tau)\left( -\dd\tau^2 + \delta_{ij}\dd x^i\dd x^j \right)$ into \eqref{eq:action} and integrating by parts we arrive at the action
\begin{align} \label{eq:actionFRW}
  \begin{split}
    S_{\mathrm{FRW}} = &\int \dd\tau\dd^3xa^4\left\{
    - \frac{3}{\mpl^2}a'^2 \vphantom{\sum_A}\right. \\
    & + \left.\frac{1}{2}a^{-2}\sum_A\left[{{\phi^A}'}^2
      + \phi^A\nabla^2\phi^A\right] %- \nabla\phi^A\cdot\nabla\phi^A\right]
      - V \right\}
    \end{split}
\end{align}
where ${}' \equiv \frac{\dd}{\dd\tau}$ is the conformal time derivative.

Having constructed the action \eqref{eq:actionFRW} using the FRW metric ansatz we arrive at an action from which we can derive a self-consistent set of equations of motion. However, these equations will, by their construction, neglect the role played by metric fluctuations in the system corresponding to the full action \eqref{eq:action}.%$S$.
In particular this leads to neglecting the terms which include the metric perturbations from the equations of motion of the fields. So although metric perturbation source terms can be calculated on the lattice, they do not include back-reaction onto the fields.

%Varying the action \eqref{eq:actionFRW} leads to the equations of motion in the continuum.
From the action \eqref{eq:actionFRW} we derive the momentum densities $\Pi^a=-6\mpl^2a'$ and $\Pi^A=a^2{\phi^A}'$, and the Hamiltonian density
\begin{align} \label{eq:hamFRW}
  \begin{split}
    \mathcal{H}_\mathrm{FRW} = &-\frac{1}{12\mpl^2}{\Pi^a}^2 \\
    &+ \frac{1}{2}\sum_A\left[ a^{-2}{\Pi^A}^2 -a^2\phi^A\nabla^2\phi^A \right] +a^4V
    \end{split}
\end{align}
from which we derive the continuum equations of motion.

In the continuum the equations of motion form a set of nonlinear partial differential equations. Discretizing these equations onto a lattice transforms them into a set of ordinary differential equations which can be integrated numerically. To do this the fields and momenta are discetized, volume integrals replaced by sums over lattice sites, and spatial derivatives in the continuum replaced by pseudospectral derivatives on the lattice. After applying this proceedure the equations of motion on a cubic lattice with $N$ equally spaced points along each dimension, the disrcete equations of motion read
\begin{align}
  & \frac{\dd\phi^A_i}{\dd\tau} = a^{-2}\Pi^A_i,  \label{eq:lat eom a}\\
  & \frac{\dd\Pi^A_i}{\dd\tau} = -a^2\nabla^2_\lat(\phi^A_j)_i -a^4\frac{\partial V_i}{\partial \phi^A_i},  \label{eq:lat eom pi a}\\
  & \frac{\dd a}{\dd\tau} = -\frac{1}{6\mpl^2}\Pi^a,  \label{eq:lat eom phi}
\end{align}
\begin{align}
  \begin{split}
    \frac{\dd\Pi^a}{\dd\tau} = &-N^{-3}\sum_i\left\{
    \sum_A\left[-a^{-3}{\Pi^A_i}^2 \right.\right. \\
    &  \left.\left. + a\phi^A_i\nabla^2_\lat(\phi^A_j)_i \right]%\left.\left. a\nabla_\lat(\phi^A_j)_i\cdot\nabla_\lat(\phi^A_j)_i \right]
    + 4a^3V_i\vphantom{\sum_A}\right\}.
  \end{split} \label{eq:lat eom pi phi}  
\end{align}
Here the subscripts $i$ and $j$ run over lattice sites, and $\nabla^2_\lat()$ is the pseudospectrally defined Laplacian operator.
For bevity we will supress the subscripts $i,j$, and $\lat$ except where they are necessary to avoid confusion.
Full details of the discretization proceedure as well as the definition of the pseudospectral Laplacian and conversion to the dimensionless units used in the code can be found in the appendix.

With \eqref{eq:lat eom a}-\eqref{eq:lat eom pi phi} we have arrived at a Hamiltonian system which can be integrated forward from intial conditions. To do so we use a Hamiltonian splitting scheme, the details of which can also be found in the appendix.

\subsection{Relation of the Semiclassical Lattice to Quantum Mechanics}
\textcolor{red}{[Here's the outline of what I think should go in this section:]}
\begin{itemize}
  \color{red}
  \item Expectation value of an operator in terms of the Wigner function.
  \item Equation of motion of the Wigner function as the Louisville equation plus order $\hbar^2$ corrections.
  \item Dropping order $\hbar^2$ terms and solution to the Louisville equation in terms of solving the classical equations of motion point-by-point from the initial phase space distribution.
  \item A single realization of the lattice corresponds to one of these classical paths.
  \item Drawing initial conditions from a distribution matching the Wigner function of the intial state (implicit assumption here that the Wigner function of the initial state is everywhere non-negative, I think we satisfy this by assuming an initial coherent state) and performing ensemble averages we reproduce the quantum result up to order $\hbar^2$ corrections.
\end{itemize}
  
\subsection{Initial Conditions} \label{sec:ICs}
In order to evolve the system \eqref{eq:lat eom a}-\eqref{eq:lat eom pi phi} forward in time we must specify initial conditions. This is done in three parts: setting the homogeneous component of the scalar fields and their momenta, initializing a realization of the fluctuations, and setting $\Pi^a$ so that the Hubble parameter obeys a lattice averaged version of the energy constraint. How each of these parts is accomplished is described below.

% Homogeneous component
The homogeneous component of the fields and momenta are set on the attractor solution of the homogeneous system. To find the attractor, we start with an initial estimate well back from the point we intend to initialize the lattice. Evolving the system with the homogeneous equations of motion this initial estimate rapidly approaches the attractor. 

% Fluctuations
Fluctuations in the fields and momenta are set in Fourier space, each mode being drawn from a multivariate Gaussian distribution with variance set to match the symmetric part of the 2-point correlation matrix in Fourier space. This is done mode-by-mode using a Choleski decomposition to take the `square root' of the correlation matrix which then multiplies a vector of $2N_\mathrm{fld}$ complex standard normal deviates. A low pass filter is applied to smooth the fields at a scale below that of the lattice Nyquist. Performing the inverse Fourier transforms then initializes the fluctuations on the lattice.

The symmetric part of the correlation matrix is calculated from linear theory for modes along a radial profile which is over sampled relative to the fundamental mode $\Delta k$ on the lattice with modes on the lattice being interpolated from this profile. The calculation is done with modes initialized well within the horizon where they can be matched to the Minkowski space solution. The system is then evolved along the attractor to the point where the lattice will be initialized. In particular this method properly accounts for the cross correlation between fields and momenta as they cross the horizon. Details of the linear calculation can be found in the appendix.

%With the homogeneous attractor solution in hand, we perform a calculation of the symmetric part of the 2-point correlation matrix in linear perturbation theory. The calculation is done on a radial profile of Fourier modes, over sampled relative to fundamental mode $\Delta k$ on the lattice, with each mode being intialized well within the horizon where it can be matched to the Minkowski space solution. The system is then evolved along the attractor to the point where the lattice will be initialized. Details of the linear calculation can be found in the appendix.

%evolving the system along the attractor to the point where the lattice will be initialized. The calculation is done on a radial profile of Fourier modes, over sampled relative to fundamental mode $\Delta k$ on the lattice, with each mode being intialized well within the horizon where it can be matched to the Minkowski space solution.
%\textcolor{red}{[I need to double check this, the choice of variables may mean this is actually being matched onto the WKB solution]}.
%Details of the linear calculation can be found in the appendix.

%The evolution of each mode is calculated using a fast/slow split of the system which supresses the oscillations of the mode, but still allows for calculation of the symmetric part of the 2-point correlation matrix.

%Once the symmetric part of the 2-point correlation matrix has been calculated on the radial profile of Fourier modes, this profile is interpolated for each mode on the lattice. The `square root' of the interpolated correlation matrix is calculated by Choleski decomposition, and a vector of $2N_\mathrm{fld}$ complex standard normal deviates is generated. Multiplying the two gives a vector of $2N_\mathrm{fld}$ complex Gaussian variables drawn from a multivariate Gaussian distribution matching the computed symmetric part of the 2-point correlation matrix. In particular, modes crossing the horizon are initialize with the proper cross correlation between the fields and their momenta. \textcolor{red}{[Should mention the initialized modes are cut below Nyquist.]} Performing the inverse Fourier transforms then initializes the fluctuations on the lattice.

% Hubble
With scalar fields and momenta realized on the lattice, $\Pi^a$ is set to satisfy a lattice averaged version of the energy constraint
\begin{equation} \label{eq:energy constraint}
  %H^2 = \frac{1}{3M_{Pl}}\langle \rho \rangle_\lat.
  {\Pi^a}^2 = 12\mpl^2a^4\langle \rho \rangle_\lat.
\end{equation}

By convesion we set $a=1$ at the start of the lattice simulation.

\textcolor{red}{[Two subtleties with this proceedure: i) this is matching the spectra instead of the real space two-point, ii) the homogeneous calculation is not using a renormalized potential, so there is a slight mismatch when we initialize onto the lattice.]}

\subsection{$\zeta$ Source Equation} \label{sec:zeta source}
\textcolor{red}{[Still revising this section.]}

\textcolor{red}{[Start with some general remarks about $\zeta$ and relate what we are doing here with the perturbative definition.]}

\textcolor{red}{[Need to point to the appendix for the $\nabla()\cdot\nabla()$ definition.]}

We take $\zeta$ to be defined by the differential
\begin{equation} \label{eq:zeta differential}
  \delta\zeta = \frac{\delta\rho}{3(\rho + P)} + \delta\alpha
\end{equation} 
where $\rho$ and $P$ are the energy density and pressure respectively, and $\alpha \equiv \ln(a)$.
Applying the continuity equation we can then arrive at an expression for the sourcing of $\zeta$
\begin{equation} \label{eq:zeta source T}
  \dot{\zeta} = \frac{\partial_iT^i_0}{3(\rho+P)} + \delta H. % check that this is correct
\end{equation}

% fix this to match with the reworked \eqref{eg:zeta source T}
Computing \eqref{eq:zeta source T} for the fields on our lattice we arrive at the following source equation for $\zeta'$ on the lattice
%
\textcolor{red}{[Revised section starts here.]}
\begin{equation} \label{eq:zeta source fld}
  \zeta' = \frac{\sum_A\left(\Pi^A\nabla^2\phi^A + \nabla\Pi^A\cdot\nabla\phi^A\right)}{3a^2\sum_A\left(\frac{1}{a^4}\Pi^A\Pi^A + \frac{1}{3}\nabla\phi^A\cdot\nabla\phi^A \right)}.
\end{equation}
In our lattice simulations the source \eqref{eq:zeta source fld} is integrated alongside the fields to compute the quantity $\zeta(\tau) - \zeta(\tau_0)$.
We set the initial condition $\zeta(\tau_0)$ pertubatively from the initial conditions of the fields
\begin{equation}
  \zeta(\tau_0) = \frac{\rho(\tau_0)-\langle\rho(\tau_0)\rangle_\lat}{3\langle\rho(\tau_0)+P(\tau_0)\rangle_\lat}.
\end{equation}
%As we will see in Sec. \ref{sec:zeta ng} it is the excess produduction of $\zeta$ as compared to a baseline case  which will be of interest.

\textcolor{red}{[Mention the reason why we are bothering with the source term instead of setting $\zeta$ perturbatively throughout. We are interested in hysteristis effects which are present when integrating along physical trajectories, by abscent if setting $\zeta$ perturbatively.]}
\textcolor{red}{[Also, need to address droping the $\delta\alpha$ term with the FRW ansatz. The idea being the $\delta\rho/(3(\rho+P))$ term gives the dominant contribution on scales the size of the lattice. On scales larger than the lattice $\delta\alpha$ is aliased to the zero mode, so can be is a constant offset on the lattice that can be measured with comparisons between realizations. On scales smaller than the lattice we can quantify the error by calculating $H(x,t)$ assuming the FRW ansatz on each time slice. Integrating $H(x,t)$ without back reaction give a $\delta\alpha$ term that can be compared to $\zeta$ as integrated with our method (should do this test and put it in the appendix with convergence tests).]}

Although $\zeta$ is blind to the distinction between the various fields $\phi^A$, with only components of the total stress energy tensor appearing in the source term, splitting the source $\zeta'$ into channels can be useful for relating the sourcing of $\zeta$ to the field dynamics. The choice of how to split the source term is not unique, however we have found considering the split defined by
\begin{align}
  \zeta_{\Pi^A\nabla^2\phi^A}' = \frac{\Pi^A\nabla^2\phi^A}{3a^2\sum_A\left(\frac{1}{a^4}\Pi^A\Pi^A + \frac{1}{3}\nabla\phi^A\cdot\nabla\phi^A \right)}, \label{eq:zeta source lap}\\
  \zeta_{\nabla\Pi^A\cdot\nabla\phi^A}' = \frac{\nabla\Pi^A\cdot\nabla\phi^A}{3a^2\sum_A\left(\frac{1}{a^4}\Pi^A\Pi^A + \frac{1}{3}\nabla\phi^A\cdot\nabla\phi^A \right)} \label{eq:zeta source gdg}
\end{align}
is useful in the context of inflation proceeding through an incomplete phase transition.

The details of the contribution of the terms \eqref{eq:zeta source lap}, \eqref{eq:zeta source gdg} are discussed in section \ref{sec:zeta production}. However, it can be noted immediately that at the linear level the contribution comes from the $\zeta_{\Pi^A\nabla^2\phi^A}'$ channel for the inflaton. With the inclusion of the  other channels we are including nonlinear sourcing for $\zeta$ from what would be considered isocurvature modes in linear theory.

%[It is worthwhile to note the form of NG and what measure will be sensitive to it are not always known a priori. It is for this reason that the ability to make comparisons to a baseline potential on a realization by realization basis proves to be a major advantage of performing these calculations using lattice simulations. - move this to the $\zeta$ NG section]

%\marginpar{
%  I think we should give some emphasis to the idea that the form of NG that will be produced by a given physical mechanism is not known prior to a full calculation and so in order to properly measure NG the first step is to determine the informative statistics to measure.
%}

\subsection{Potential} \label{sec:potential}
Up to this point our description of the lattice simulation has been kept general and is applicable to a wide range of inflationary scenarios which are differentiated by the choice of potential. Here we introduce a form of potential which will focus our scope onto the scenario of inflation proceeding through an incomplete phase transition of the transverse field. 

In what follows we write the potential as a sum of two parts $V = V_0 + \Delta V$, where $V_0$ acts as a baseline contributing the majority of the energy density driving inflation and $\Delta V$ is a feature ontop of $V_0$, which allows us to choose a region of field space to break the symmetry in the transverse field.

We have chosen a parameterization of $\Delta V$ in the familiar form of a Landau-Ginzburg potential, with an overall amplitude modulated by the longitudinal field. And $V_0$ to be quadratic which provides a clean baseline as any nonlinearity of the field dynamics (up to coupling through the Hubble parameter) will be due solely to the $\Delta V$ part of the potential.
\begin{align}
  &V_0(\phi,\chi) = \frac{1}{2}m^2_\phi\phi^2 + \frac{1}{2}m^2_\chi\chi^2, \label{eq:V0} \\
  &\Delta V(\phi,\chi) = \frac{1}{4}\lambda(\phi)\left[ (\chi^2-v^2)^2 - v^4 \right]. \label{eq:DeltaV}
\end{align}
For convinience, the fields $\phi^A$ have been renamed $\phi$ and $\chi$ in this two field model. With $\phi$ being aligned with the direction of inflation at the level of homogeneous fields.

% move to field dynamics section
%Then at the linear level $\phi$ plays the role of the inflaton and $chi$ is a spectator field with an effective mass squared which becomes negative as it is modulated by the rolling of $\phi$. Although as we will see, nonlinear effects will play an important role in sourcing $\Delta\zeta$.

We will informally refer to the $\phi$ and $\chi$ field space directions respectively as being longitudinal and transverse, but it should be noted we are applying this notation somewhat loosely as this association does not account for the inhomogenous state of the fields and their momenta. \textcolor{red}{[Make this a footnote]}

% Part about not getting overly focused on the explicit potential form
Recall our goal in setting the potential is to study the phenomenology of $\zeta$ NG generated by a potential feature. The chosen form of the potential given in \eqref{eq:V0} and \eqref{eq:DeltaV} should not be thought of only as a defining a complete inflationary potential, but only as a parameterization within some small region around a feature of interest. So by choosing to aligning $\phi$ with the direction of inflation what we are doing is setting the orienting of $\Delta V$, our parameterization of the potential feature, relative to the direction of flow taken by inflation as it passes this feature, smoothed in a region on the scale of the lattice. \textcolor{red}{[Forward reference to figure exploring other potential forms for transverse symmetry breaking]}

The function $\lambda(\phi)$ controls where along the potential the transverse field $\chi$ has its symmetry broken and restored, with the longitudinal field $\phi$ acting as a clock and the parameter $v$ setting the positions of the new transverse minima. The potential surfaces of $V$ and $V_0$ are shown in Fig. \ref{fig:potential} with the potential parameters scaled arbitrarily for visual clarity.

For computational simplicity we have parameterized $\lambda(\phi)$ as a piece-wise polynomial
\begin{equation} \label{eq:lambda}
  \lambda(\phi) =
  \begin{cases}
    0 & \quad \text{if } |\phi-\phi_p|\ge\phi_w \\
    \lambda_\chi\left[\left(\frac{\phi-\phi_p}{\phi_w}\right)^2 - 1 \right]^2 & \quad \text{if  } |\phi-\phi_p|<\phi_w
  \end{cases}.
\end{equation}
The parameters $\phi_p$ and $\phi_w$ control the placement and duration of the symmetry breaking along the longitudinal field and $\lambda_\chi$ scales the strength of the associated instability around $\chi=0$.

While it is not strictly necessary to choose $\lambda(\phi)$ with compact support doing has the advantage of allowing us to initialize the simulation in a region where $\Delta V=0$. Doing so allows fluctuations to be initialized using the baseline potential $V_0$, the form of which \eqref{eq:V0} has been chosen so as to avoid any strong nonlinearities, allowing us to use the linear calculation described in Sec. \ref{sec:ICs} to calculate the initial fluctuation statistics. As an added benifit the compact support of $\lambda(\phi)$ facilitates the comparison to the baseline case as described in Sec. \ref{sec:results}.

It should be mentioned that while in \eqref{eq:DeltaV} we have specified a functional form of $\Delta V$, our exploration of incomplete phase transition using other functional forms of $\Delta V$ \textcolor{red}{[which were likewise broke/restored symmetry in the transverse direction and were $\pm\chi$ symmetric]} have yielded NG following the same general picture of forming concentrations of NG uncorrelated with the Gaussian part of the field. \textcolor{red}{[Also that the effect persists over a wide range of parameters (growth of fluctuations being exponentially sensitive to $\sqrt{|m^2_{eff}|}$ and duration of instability, but also the displacement of the new minima can be very large compared to the initial size of fluctuations).]}

\Fpotential

While our focus in this paper is on inflation proceeding through an incomplete phase transition the same form of $\Delta V$ can be used to study the case where $\chi$ fluctuations undergo sufficient growth to reach the new transverse minima of the symmetry broken potential and complete the phase transition. The case of inflation proceeding through a completed phase transition proves interesting, we leave its discussion to a dedicated future paper.

\Fpotparam

\textcolor{red}{[Here put the potential parameters for the run and reference to Fig. \ref{fig:potparam} of $\Delta V(\phi=\phi_p,\chi)$ and $V_{,\chi\chi}(\phi,\chi=0)$. Mention what is bounding these choices, ie. not trapping the fields, ending inflation, or completing the phase transition]}

%\marginpar{
%  I should give some motivation of looking at instabilities for generating NG.
%}

%We will consider a model with two fields $\phi$ and $\chi$ and by convention choose $\phi$ to be longitudinal, in the direction by which inflation proceeds, and $\chi$ to be transverse.
%We write the potential as a sum of a baseline and stimulus term $V = V_0 + \Delta V$ with the form of $\Delta V$ chosen to model the phenomenology of a potential feature breaking and restoring symmetry in the transverse direction.
%The baseline $V_0$ is taken to be quadratic in order to avoid nonlinearities in the baseline case.
%\marginpar{
%  Our labelling of $\phi$ and $\chi$ as being the longitudinal and transverse direction respectively should be interpretted loosely. We note the divegerence of trajectories during the instability requires a more accurate description of the longitudinal and transverse directions to be spatially dependent.
%  Should put this in a footnote, or paranthetically.
%}

%The functional forms we have chosen for $V_0$ and $\Delta V$ are given by
%\begin{align} \label{eq:potential}
%  &V_0(\phi,\chi) = \frac{1}{2}m^2_\phi\phi^2 + \frac{1}{2}m^2_\chi\chi^2 \\
%  &\Delta V(\phi,\chi) = \frac{1}{4}\lambda(\phi)\left[ (\chi^2-v^2)^2 - v^4 \right]. \label{eq:potential feature}
%  %&\Delta V(\phi,\chi) = \frac{1}{2}\mu^2(\phi)\chi^2 + \frac{1}{4}\lambda_\chi\chi^4,
%\end{align} 
%It can be seen this form of $\Delta V$ is a symmetry breaking potential in the transverse field $\chi$ with term $\lambda(\phi)$ using the longitudinal field $\phi$ as a clock to turn the symmetry breaking term on or off and $v$ setting the positions of the minima in the transverse direction.

%For computational simplicity $\lambda(\phi)$ is parameterized as a piece-wise polynomial and given compact support to allow for a more straight forward comparison to the baseline case
%\begin{equation} \label{eq:lambda}
%  \lambda(\phi) =
%  \begin{cases}
%    0 & \quad \text{if } |\phi-\phi_p|\ge\phi_w \\
%    \lambda_\chi\left[\left(\frac{\phi-\phi_p}{\phi_w}\right)^2 - 1 \right]^2 & \quad \text{if  } |\phi-\phi_p|<\phi_w
%  \end{cases}.
%\end{equation}
%The parameters $\phi_p$ and $\phi_w$ control the placement and duration of the symmetry breaking along the longitudinal field and $\lambda_\chi$ scales its strength.
%[Some reference to figure \ref{fig:potential} and general features of the potential.]

%There are regions of parameter space for which the potential [reference potential equations] has pair of local minima. These local minima give rise to the possibility of trapping the fields. While this is an interesting case we will focus in this paper on the altenative case where the fields do not become trapped and leave the trapped case to future work. 

