% Setting up the problem

\section{Setup} \label{sec:setup}
% This would fit more in the potential subsection
We are interested in studying the effects of a localized feature in the inflationary potential.
To this end it will prove useful to split the potential as a sum of two terms $V(\phi^A) = V_0(\phi^A) + \Delta V(\phi^A)$.
This split can be though of as starting with a baseline system and applying a stimulus by deforming the potential, $V_0$ being the baseline and $\Delta V$ being the stimulus.
The system's response can be calculated by taking the difference between two lattice simulations, one run with the deformed potential $V_0+\Delta V$ and the other with the baseline potential $V_0$, with both runs starting from identical initial conditions.
Of particular interest in the cosmological context is the NG component of $\zeta$ sourced by this response.

In this section we setup our system, detailing the equations of motion solved on the lattice, the setting of initial conditions, the sourcing of $\zeta$, and the form of the potential considered.

\subsection{Equations of Motion}
% Open with introduction to lattice sims 
% Define $a$, $g$

The action for a set of scalar fields minimally coupled to gravity is given by
\begin{equation} \label{eq:action}
  S = \int \sqrt{-g}\left\{\frac{1}{2}\mpl^2R - \frac{1}{2}\sum_Ag^{\mu\nu}\partial_\mu\phi^A\partial_\nu\phi^A - V \right\}\dd^4x.
\end{equation}
Where $\phi^A$ for $A=1,2,...,N_\mathrm{fld}$ are scalar fields, $R$ is the Ricci scalar, and $\mpl$ is the reduced Planck mass. 
Substituting the ansatz of a Friedman-Robertson-Walker (FRW) metric $\dd s^2 = a^2(\tau)\left( -\dd\tau^2 + \delta_{ij}\dd x^i\dd x^j \right)$ into \eqref{eq:action} and integrating by parts we arrive at the action
\begin{align} \label{eq:actionFRW}
  \begin{split}
    S_{\mathrm{FRW}} = &\int \dd\tau\dd^3xa^4\left\{
    - \frac{3}{\mpl^2}a'^2 \vphantom{\sum_A}\right. \\
    & + \left.\frac{1}{2}a^{-2}\sum_A\left[{{\phi^A}'}^2
      + \phi^A\nabla^2\phi^A\right] %- \nabla\phi^A\cdot\nabla\phi^A\right]
      - V \right\}
    \end{split}
\end{align}
where ${}' \equiv \frac{\dd}{\dd\tau}$ is the conformal time derivative.

Having constructed the action \eqref{eq:actionFRW} using the FRW metric ansatz we arrive at an action from which we can derive a self-consistent set of equations of motion. However, these equations will, by their construction, neglect the role played by metric fluctuations in the system corresponding to the full action \eqref{eq:action}. 
In particular this leads to neglecting the terms which include the metric perturbations from the equations of motion of the fields. So although metric perturbation source terms can be calculated on the lattice, they do not include back-reaction onto the fields.

%Varying the action \eqref{eq:actionFRW} leads to the equations of motion in the continuum.
From the action \eqref{eq:actionFRW} we derive the momentum densities $\Pi^a=-6\mpl^2a'$ and $\Pi^A=a^2{\phi^A}'$, and the Hamiltonian density
\begin{align} \label{eq:hamFRW}
  \begin{split}
    \mathcal{H}_\mathrm{FRW} = &-\frac{1}{12\mpl^2}{\Pi^a}^2 \\
    &+ \frac{1}{2}\sum_A\left[ a^{-2}{\Pi^A}^2 -a^2\phi^A\nabla^2\phi^A \right] +a^4V
    \end{split}
\end{align}
from which we derive the continuum equations of motion.

In the continuum the equations of motion form a set of nonlinear partial differential equations. Discretizing these equations onto a lattice transforms them into a set of ordinary differential equations which can be integrated numerically. To do this the fields and momenta are discetized, volume integrals replaced by sums over lattice sites, and spatial derivatives in the continuum replaced by pseudospectral derivatives on the lattice. After applying this procedure the equations of motion on a cubic lattice with $N$ equally spaced points along each dimension, the discrete equations of motion read
\begin{align}
  & \frac{\dd\phi^A_i}{\dd\tau} = a^{-2}\Pi^A_i,  \label{eq:lat eom phi}\\
  & \frac{\dd\Pi^A_i}{\dd\tau} = -a^2\nabla^2_\lat(\phi^A_j)_i -a^4\frac{\partial V_i}{\partial \phi^A_i},  \label{eq:lat eom pi phi}\\
  & \frac{\dd a}{\dd\tau} = -\frac{1}{6\mpl^2}\Pi^a,  \label{eq:lat eom a} \\
  & \frac{\dd\Pi^a}{\dd\tau} = -N^{-3}\sum_i\left\{ \sum_A\left[-a^{-3}{\Pi^A_i}^2 \right.\right. \nonumber \\
    & \hphantom{\frac{\dd\Pi^a}{\dd\tau} = } \left.\left. + a\phi^A_i\nabla^2_\lat(\phi^A_j)_i \right] + 4a^3V_i\vphantom{\sum_A}\right\}. \label{eq:lat eom pi a}  
\end{align}
Here the subscripts $i$ and $j$ run over lattice sites, and $\nabla^2_\lat()$ is the pseudospectrally defined Laplacian operator.
For brevity we will suppress the subscripts $i,j$, and $\lat$ except where they are necessary to avoid confusion.
Full details of the discretization procedure as well as the definition of the pseudospectral Laplacian and conversion to the dimensionless units used in the code can be found in the appendix.

With \eqref{eq:lat eom phi}-\eqref{eq:lat eom pi a} we have arrived at a Hamiltonian system which can be integrated forward from initial conditions. To do so we use a Hamiltonian splitting scheme, the details of which can also be found in the appendix.

\subsection{Relation of the Semiclassical Lattice to Quantum Mechanics}
\textcolor{red}{[The point of this section is not to go through the full theory of semiclassical/quantum, but just to say enough that it isn't left as magic to the reader.]}
\textcolor{red}{[Here's the outline of what I think should go in this section:]}
\begin{itemize}
  \color{red}
  \item Expectation value of an operator in terms of the Wigner function.
  \item Equation of motion of the Wigner function as the Louisville equation plus order $\hbar^2$ corrections.
  \item Dropping order $\hbar^2$ terms and solution to the Louisville equation in terms of solving the classical equations of motion point-by-point from the initial phase space distribution.
  \item A single realization of the lattice corresponds to one of these classical paths.
  \item Drawing initial conditions from a distribution matching the Wigner function of the initial state (implicit assumption here that the Wigner function of the initial state is everywhere non-negative, I think we satisfy this by assuming an initial coherent state) and performing ensemble averages we reproduce the quantum result up to order $\hbar^2$ corrections.
\end{itemize}
  
\subsection{Initial Conditions} \label{sec:ICs}
\textcolor{red}{[Add a part about how the lattice can be initialized at any point prior to the onset of nonlinearity.]}
In order to evolve the system \eqref{eq:lat eom phi}-\eqref{eq:lat eom pi a} forward in time we must specify initial conditions. This is done in three parts: setting the homogeneous component of the scalar fields and their momenta, initializing a realization of the fluctuations, and setting $\Pi^a$ so that the Hubble parameter obeys a lattice averaged version of the energy constraint. How each of these parts is accomplished is described below.

% Homogeneous component
The homogeneous component of the fields and momenta are set on the attractor solution of the homogeneous system. To find the attractor, we start with an initial estimate well back from the point we intend to initialize the lattice and evolve the system forward using the homogeneous equations of motion. The initial estimate we provide will then rapidly approaches the attractor. Details of this calculation can be found in the appendix. 

% Fluctuations
Fluctuations in the fields and momenta are set in Fourier space, each mode being drawn from a multivariate Gaussian distribution with variance set to match the real part of the 2-point correlation matrix in Fourier space \textcolor{red}{[and with uncorrelated and uniformly distributed complex phases]}. This is done mode-by-mode using a Choleski decomposition to take the `square root' of the correlation matrix which then multiplies a vector of $2N_\mathrm{fld}$ complex standard normal deviates. A low pass filter is applied to smooth the fields at a scale below that of the lattice Nyquist. Performing the inverse Fourier transforms then initializes the fluctuations on the lattice.

The real part of the correlation matrix is calculated from linear theory for modes along a radial profile which is over sampled relative to the fundamental mode $\Delta k$ on the lattice with modes on the lattice being interpolated from this profile. The calculation is done with modes initialized well within the horizon where they can be matched to the Minkowski space solution. The system is then evolved along the attractor to the point where the lattice will be initialized. In particular this method properly accounts for the cross correlation between fields and momenta as they cross the horizon. Details of the linear calculation can be found in the appendix.

% Discussion on phases
% Mention that the initialized Fourier modes are given random phases.
% Mention that the perturbative calculation does not make reference to phases and why that is okay.
% Explain why statistically we don't need to track the phases when intiallizing the lattice with Gaussian inital conditions.
% Explain how initializing at different times with the same generation of random variables is equivalent to resampling the ensemble from the Wigner function.

A few words should be said about how the complex phases are assigned to Fourier modes when initializing fluctuations on the lattice, in particular with respect to the relation between these phases and the time at which the lattice is initialized.
Using the method we have described, initializing a realization of fluctuations on the lattice and then evolving the lattice forward to a later time using \eqref{eq:lat eom phi}-\eqref{eq:lat eom pi a} is not equivalent to first evolving the perturbative system forward to compute the 2-point at that later time and then initializing a realization of the fluctuations using the evolved 2-point.
Said another way, the initialization of fluctuations and the evolution of those fluctuations on the lattice do not commute.
This is true even if the evolution on the lattice is linear and an identical set of random numbers is used for the realization in either case.
Owing to the fact that the evolution on the lattice evolves the phases of the fluctuations, the evolved phases will mismatch with those of later realized fluctuations as no phase information is carried in the perturbative calculation of the 2-point. 
\textcolor{red}{[I think techincally I am talking about a realization of a set of random number's not a actual set of random numbers.]}

While it may at first seem concerning that the phases of fluctuations differ depending upon the arbitrary choice of when the lattice is initialized there is in fact no issue.
From the point of view of sampling the Wigner function any choice of time to initialize the lattice is statistically equivalent. The mismatch of phases when comparing different initalization times corresponds to a change in the mapping between a particular realization of the random numbers used to initialize the fluctuations and a particular element from the ensemble of semi-classical calculations which describe the evolution of the Wigner function.

%For the same reason it is not necessary to track the phases of the modes from past infinity.

% Comment on phase matching an the comparison between runs.
\textcolor{red}{[Part about matching phases if we want to difference two simulations directly.]}

So long as the evolution of the system remains linear the computationally simpler perturbative calculation is sufficient.
\textcolor{red}{[There is a caveat here that the Wigner function must be Gaussian to be described by the 2-point, hence the relation to coherent states.]}
\textcolor{red}{[Second caveat for the possibility of nonlinear sourcing of $\zeta$ from linear field dynamics.]}
It is when the evolution of the system becomes nonlinear that we must resort to the computationally much more expensive lattice calculation.

% Hubble
Finally with the scalar fields and momenta realized on the lattice, $\Pi^a$ is set to satisfy a lattice averaged version of the energy constraint
\begin{equation} \label{eq:energy constraint}
  %H^2 = \frac{1}{3M_{Pl}}\langle \rho \rangle_\lat.
  {\Pi^a}^2 = 12\mpl^2a^4\langle \rho \rangle_\lat.
\end{equation}

By convention we set $a=1$ at the start of the lattice simulation.

\textcolor{red}{[Two subtleties with this procedure: i) this is matching the spectra instead of the real space two-point, ii) the homogeneous calculation is not using a renormalized potential, so there is a slight mismatch when we initialize onto the lattice.]}

% sec:zeta source
\subsection{$\zeta$ Source Equation} \label{sec:zeta source}
\textcolor{red}{[Start with some general remarks about $\zeta$ and relate what we are doing here with the perturbative definition.]}

%Of key interest in our study is the quantity $\zeta$
% relic that survives outside the horizon
% manifests by sourcing observables

% some limitations due to FRW ansatz
% integrated $\zeta$ vs perturbative $\zeta$, specifically with regards to asymptotic behaviour and dropping $\alpha$

% $\zeta$ differential
The defining differential of $\zeta$ is given by
\begin{equation}
  \dd\zeta = \dd\alpha + \frac{\dd\ln(\rho)}{3(1+w)} \label{eq:zeta diff w}
\end{equation}
where $\alpha \equiv \ln(a)$, and $w \equiv \rho/P$ is the equation of state for energy density $\rho$ and isotropic pressure $P$. 

% $\zeta$ derivative
From the differential \eqref{eq:zeta diff w} comes the time derivative
\begin{equation}
  N_\xi^{-1}\frac{\dd\zeta}{\dd\xi} = H + \frac{1}{3(1 + w)}N_\xi^{-1}\frac{\dd\ln(\rho)}{\dd\xi} \label{eq:zeta der xi}
\end{equation}
where $\xi$ is some timelike variable and $N_\xi$ is its shift.

% $\zeta$ derivative on the lattice
We can evaluate \eqref{eq:zeta der xi} on the lattice in the FRW ansatz using as the timelike variable the conformal time $\tau$. In the FRW ansatz the fields $a$ and $H$ are assumed to be spatially uniform, and the energy density and isotropic pressure become
\begin{align}
  & \rho = \frac{1}{2}a^{-2}\sum_A\left[a^{-4}\Pi^A\Pi^A + \nabla\phi^A\cdot\nabla\phi^A \right] + V, \label{eq:rho frw}\\
  & P = \frac{1}{2}a^{-2}\sum_A\left[a^{-4}\Pi^A\Pi^A - \frac{1}{3}\nabla\phi^A\cdot\nabla\phi^A \right] - V.
\end{align}
Substitution of $\rho$ and $P$ in terms of $\phi^A$ and $\Pi^A$ and using either the covariant conservation of the stress-energy tensor or the equations of motion \eqref{eq:lat eom phi}-\eqref{eq:lat eom pi a} to evaulate $\frac{\dd\ln(\rho)}{\dd\alpha}$ then gives a source equation for $\zeta$
\begin{equation}
  \frac{\dd\zeta}{\dd\tau} = \frac{\sum_A\left(\Pi^A\nabla^2\phi^A + \nabla\Pi^A\cdot\nabla\phi^A\right)}{3a^2\sum_A\left(a^{-4}\Pi^A\Pi^A + \frac{1}{3}\nabla\phi^A\cdot\nabla\phi^A \right)}. \label{eq:zeta source fld}
\end{equation}
On the lattice $\nabla()\cdot\nabla()$ is defined psuedospectrally as is detailed in the appendix.% In our lattice simulations we integrate \eqref{eq:zeta source fld} alongside the fields to compute $\zeta(x,\tau) - \zeta(x,\tau_0)$.

% The FRW ansatz
When applying the FRW ansatz we are making the assumption that, for our system and within the simulation volume, hypersurfaces of uniform $\tau$ are approximately hypersurfaces of uniform $\alpha$. We will need to verify that this approximation leads to a modification of $\zeta$ which is small compared to the size of the effect we measure by integrating \eqref{eq:zeta source fld}. Before turning to the details of how this is done, we will first introduce an approximation to $\zeta$ which will prove useful in making this comparison. 

% $\zeta$ perturbation
In addition to the derivative description of $\zeta$ given by \eqref{eq:zeta der xi}, the differential \eqref{eq:zeta diff w} can also be used to make a perturbative approximation 
\begin{equation}
  \delta\zeta_\mathrm{pert} = \delta\alpha + \frac{\delta\ln(\rho)}{3\langle 1+w \rangle} \label{eq:zeta pert w}
\end{equation}
where we use $\delta$ to indicate the perturbations in quantities from which a suitably defined mean has been subtracted and the average $\langle 1+w \rangle$ is taken to keep the approximation first order in perturbations. This perturbative approximation is exact when taken on hypersurfaces of uniform $H$. When taken on another hypersurface its validity can be shown to rely on the existance of a uniform $H$ hypersurface for which the denominator $(3(1+w))^{-1}$ remains approximately constant when measured along the actual trajectories connecting the two hypersurfaces.

The perturbative approximation \eqref{eq:zeta pert w} connects to the familiar definition of $\zeta$ from linear perturbation theory when $\delta\alpha$ and $\delta\ln(\rho)$ are determined by linear theory, and $\langle 1+w \rangle$ is determined by the background cosmology. When going beyond the linear level, \eqref{eq:zeta pert w} differs from the full calculation of integrating \eqref{eq:zeta der xi} when $w$ \textcolor{red}{[or equivalently $\epsilon_\mathrm{free}$?]} follows a different evolution along different trajectories.
\textcolor{blue}{[I should think of how to write this in terms of the $\epsilon$ formalism. Probably something about path dependence, where trajectories with the same $\epsilon_\mathrm{actual}$ can differ in $\epsilon_\mathrm{free}$. This effect is a qualitative difference between integrating $\zeta$ and the perturbative estimate.]} 

% why we use the integrated $\zeta$ instead of the perturbative one
\textcolor{red}{[Put in a paragraph explaining why we are using \eqref{eq:zeta source fld} to integrate $\zeta$ rather than using \eqref{eq:zeta pert w}.]}
%The choice to calculate $\zeta$ on the lattice by integration of a source is in part due to ...
%In our simulations we use the integrated $\zeta$ from the source \eqref{eq:zeta source fld} rather than the approximation \eqref{eq:zeta pert w} specifically because we are interested in those effects for which 

\textcolor{red}{[Put in paragraph about setting the initial conditions of $\zeta$ on the lattice.]}

% Comments on dropping the fluctuations in $\delta\alpha$
Returning now to checking the validity of using the FRW ansatz in deriving \eqref{eq:zeta source fld}, our method will be to estimate a characteristic scale for the magnitude of $\delta\alpha$ fluctuations and compare to $\delta\ln(\rho)$ fluctuations on the lattice's integration hypersurfaces. From \eqref{eq:zeta pert w}, the FRW ansatz will be valid when the scale of $\delta\alpha$ does not exceed that of $\delta\ln(\rho)$ by a factor of order $(3(1+w))^{-1}$ \textrm{[barring some nonlinear effect]}. In practice this condition is not difficult to meet throughout much of inflation when $w \approx -1$. In appendix \ref{appendix:frw_zeta} we detail how we have estimated $\delta\alpha$ and verify it is sufficiently small compared to $\delta\ln(\rho)$ for the system we study.

%From \eqref{eq:zeta pert w} and using the fact that $3\langle 1+w \rangle \ll 1$ during our simulations, or $3\langle 1+w \rangle < 2$ throughout inflation, the the FRW ansatz will be valid when $\delta\alpha \ll \delta\ln(\rho)$ \textrm{[barring some nonlinear effect]}. In appendix \ref{appendix:frw_zeta} we detail how we have estimated $\delta\alpha$ and verify it is small compared to $\delta\ln(\rho)$ for the system we study.

%We take $\zeta$ to be defined by the differential
%\begin{equation} \label{eq:zeta differential}
%  \delta\zeta = \frac{\delta\rho}{3(\rho + P)} + \delta\alpha
  %\delta\zeta = \frac{\delta\ln(\rho)}{3(1 + W)} + \delta\alpha
%\end{equation}
%where $\rho$ is the energy density, $W$ is the equation of state, and $\alpha \equiv \ln(a)$.
%where $\rho$ and $P$ are the energy density and pressure respectively, and $\alpha \equiv \ln(a)$.
%\textcolor{red}{[Should we be putting $(\rho+P)$ or $\rho(1+W)$ here? I think the physical quantity should be $-\int_{H(x,\langle\alpha\rangle-\delta\alpha)}^{H(x,\langle\alpha\rangle)}\epsilon^{-1}(x,H')\dd\ln(H')$ expressed as a differential, ie $-\epsilon^{-1}\delta\ln(H)$.]}

%The differential \eqref{eq:zeta differential} gives rise to a $\zeta$ source term which on the lattice, after applying the FRW ansatz and the continuity equation, becomes
%The differential \eqref{eq:zeta differential} gives rise to a $\zeta$ source term which on the lattice, after applying the continuity equation, becomes
%\begin{equation} \label{eq:zeta source T}
%  \dot{\zeta} = \frac{\partial_iT^i_0}{3(\rho+P)} + \delta H. % check that this is correct
  %\zeta' = \frac{\partial_iT^i_0}{3(\rho+P)}
%\end{equation}
%\textcolor{red}{[At this point \eqref{eq:zeta source T} is not using the FRW ansatz, so should be using covariant derivative.]}

%Computing \eqref{eq:zeta source T} for the fields on our lattice we arrive at the following source equation for $\zeta'$ on the lattice
%Computing \eqref{eq:zeta source T} in terms of the fields on the lattice, for which we have taken the ansatz of an FRW metric, we arrive at the following source equation for $\zeta'$ on the lattice
%\begin{equation} \label{eq:zeta source fld}
%  \zeta' = \frac{\sum_A\left(\Pi^A\nabla^2\phi^A + \nabla\Pi^A\cdot\nabla\phi^A\right)}{3a^2\sum_A\left(\frac{1}{a^4}\Pi^A\Pi^A + \frac{1}{3}\nabla\phi^A\cdot\nabla\phi^A \right)},
%\end{equation}
%$\nabla()\cdot\nabla()$ being defined psuedospectrally as is detailed in the appendix.
%In our lattice simulations the source \eqref{eq:zeta source fld} is integrated alongside the fields to compute the quantity $\zeta(\tau) - \zeta(\tau_0)$.
%The initial condition $\zeta(\tau_0)$ is set pertubatively from the initial conditions of the fields
%\begin{equation}
%  \zeta(\tau_0) = \frac{\rho(\tau_0)-\langle\rho(\tau_0)\rangle_\lat}{3\langle\rho(\tau_0)+P(\tau_0)\rangle_\lat}.
%\end{equation}

%\textcolor{red}{[Paragraph on why we are calculating $\zeta$ from an integrated source term instead of calculating perturbatively. We are interested in hysteresis effects which are present when integrating along physical trajectories, which are absent if setting $\zeta$ perturbatively.]}

%\textcolor{blue}{[Point of discussion: dropping the $\delta\alpha$ term in $\delta\zeta = \frac{\delta\rho}{3(\rho + P)} + \delta\alpha$ and then integrating $\zeta$ from a source is not equivalent to dropping the $\delta\alpha$ term and then setting $\zeta$ perturbatively. This is most obvious when as $\delta\rho \to 0$ then $\zeta$ set perturbatively vanishes, but integrating $\zeta$ from a source picks up a finite value due to the $(\rho+P)$ denominator time dependence.
%    The question is how does one compare between the perturbative and integrated $\zeta$? Should we consider the integrated $\zeta$ as recovering part of the $\delta\alpha$ term dropped from the perturbative $\zeta$, and is there a way to make this rigorous?]}

%\textcolor{red}{[The idea being the $\delta\rho/(3(\rho+P))$ term gives the dominant contribution on scales the size of the lattice. On scales larger than the lattice $\delta\alpha$ is aliased to the zero mode, so can be is a constant offset on the lattice that can be measured with comparisons between realizations (I no longer think that is a valid line of thinking due to how we are now thinking about the integrated vs perturbative $\zeta$). On scales smaller than the lattice we can quantify the error by calculating $H(x,t)$ assuming the FRW ansatz on each time slice. Integrating $H(x,t)$ without back reaction give a $\delta\alpha$ term that can be compared to $\zeta$ as integrated with our method (should do this test and put it in the appendix with convergence tests).]}

% $\zeta$ source channels on the lattice
Although $\zeta$ is blind to the distinction between the various fields $\phi^A$, with only components of the total stress energy tensor appearing in the source term, splitting the source \eqref{eq:zeta source fld} into channels can be useful for relating the sourcing of $\zeta$ to field dynamics. The choice of how to split the source term is not unique, however we have found considering the split defined by
\begin{align}
  \zeta_{\Pi^A\nabla^2\phi^A}' = \frac{\Pi^A\nabla^2\phi^A}{3a^2\sum_A\left(\frac{1}{a^4}\Pi^A\Pi^A + \frac{1}{3}\nabla\phi^A\cdot\nabla\phi^A \right)}, \label{eq:zeta source lap}\\
  \zeta_{\nabla\Pi^A\cdot\nabla\phi^A}' = \frac{\nabla\Pi^A\cdot\nabla\phi^A}{3a^2\sum_A\left(\frac{1}{a^4}\Pi^A\Pi^A + \frac{1}{3}\nabla\phi^A\cdot\nabla\phi^A \right)} \label{eq:zeta source gdg}
\end{align}
to be useful in the context of inflation proceeding through an incomplete phase transition.

The details of the contribution of the terms \eqref{eq:zeta source lap}, \eqref{eq:zeta source gdg} are discussed in section \ref{sec:zeta production}. However, it can be noted immediately that at the linear level only one of these channels contributes to sourcing $\zeta$, that being the $\zeta_{\Pi^A\nabla^2\phi^A}'$ channel for the inflaton. With the inclusion of the  other channels we are including the nonlinear sourcing for $\zeta$ from what would be considered isocurvature modes in linear theory.

\textcolor{blue}{[These next two paragraphs are the alternate formulation of $\zeta$ in terms of $\epsilon$. ]}
% this bit...
The differential \eqref{eq:zeta diff w} is alternatively described as
\begin{equation}
  \dd\zeta = \dd\alpha + \frac{\dd\ln(\rho)}{2\epsilon_\mathrm{free}} \label{eq:zeta diff eps}
\end{equation}
where $\epsilon \equiv -\frac{\dd\ln(H)}{\dd\alpha}$ and the subscripted $\epsilon_\mathrm{free}$ indicates that it is to be evaluated along a free trajectory. \textcolor{red}{[define free trajectories, could be in terms of the HJE equations or $\nabla_iT^i_0 = 0$ (with some stament about zero shift?). Does this depend on defining comoving coordinates on some smoothing scale?]}

From this alternative differential description and by using $\alpha$ as a timelike variable comes the derivative \textcolor{red}{mention this is using the local energy constraint with gradient terms dropped}
\begin{equation}
  \frac{\dd\zeta}{\dd\alpha} = 1 - \frac{\epsilon_\mathrm{actual}}{\epsilon_\mathrm{free}} \label{eq:zeta der alpha}
\end{equation}
where $\epsilon_\mathrm{actual}$ is distinguished from $\epsilon_\mathrm{free}$ in that it is evaluated along trajectories following the actual evolution of the system rather than along the free trajectories of the system. This form of the derivative leads to the physical interpretation that $\zeta$ is conserved when propagating along free trajectories and $\zeta$ is produced when trajectories propagating along the actual evolution of the system deviate from the free trajectories.
% ... is out of order.

\subsection{Potential} \label{sec:potential}
Up to this point our description of the lattice simulation has been kept general and is applicable to a wide range of inflationary scenarios which are differentiated by the choice of potential.
We will now focus our scope onto the scenario of inflation proceeding through an incomplete phase transition of the transverse field and introduce a form of potential for this scenario.

In what follows we write the potential as a sum of two parts $V = V_0 + \Delta V$, where $V_0$ acts as a baseline contributing the majority of the energy density driving inflation and $\Delta V$ is a feature on top of $V_0$, which allows us to choose a region of field space to break the symmetry in the transverse field.

We have chosen a parameterization of $\Delta V$ in the familiar form of a Landau-Ginzburg potential, with an overall amplitude modulated by the longitudinal field. We have chosen $V_0$ to be quadratic which provides a clean baseline as any nonlinearity of the field dynamics (up to coupling through the Hubble parameter) will be due solely to the $\Delta V$ part of the potential.
\begin{align}
  &V_0(\phi,\chi) = \frac{1}{2}m^2_\phi\phi^2 + \frac{1}{2}m^2_\chi\chi^2, \label{eq:V0} \\
  &\Delta V(\phi,\chi) = \frac{1}{4}\lambda(\phi)\left[ (\chi^2-v^2)^2 - v^4 \right]. \label{eq:DeltaV}
\end{align}
For convenience, the fields $\phi^A$ have been renamed $\phi$ and $\chi$ in this two field model, with $\phi$ being aligned with the direction of inflation at the level of homogeneous fields \footnote{We will informally refer to the $\phi$ and $\chi$ field space directions as being longitudinal and transverse respectively, but it should be noted we are applying this notation somewhat loosely as this association does not account for the inhomogenous state of the fields and their momenta.}.

The function $\lambda(\phi)$ controls where along the potential the transverse field $\chi$ has its symmetry broken and restored, with the longitudinal field $\phi$ acting as a clock and the parameter $v$ setting the positions of the new transverse minima. The potential surfaces of $V$ and $V_0$ are shown in Fig. \ref{fig:potential} with the potential parameters scaled arbitrarily for visual clarity.

For computational simplicity we have parameterized $\lambda(\phi)$ as a piece-wise polynomial
\begin{equation} \label{eq:lambda}
  \lambda(\phi) =
  \begin{cases}
    0 & \quad \text{if } |\phi-\phi_p|\ge\phi_w \\
    \lambda_\chi\left[\left(\frac{\phi-\phi_p}{\phi_w}\right)^2 - 1 \right]^2 & \quad \text{if  } |\phi-\phi_p|<\phi_w
  \end{cases}.
\end{equation}
The parameters $\phi_p$ and $\phi_w$ control the placement and duration of the symmetry breaking along the longitudinal field and $\lambda_\chi$ scales the strength of the associated instability around $\chi=0$.

While it is not strictly necessary to choose $\lambda(\phi)$ with compact support, but doing has the advantage of allowing us to initialize the simulation in a region where $\Delta V=0$. % Doing so allows fluctuations to be initialized using the baseline potential $V_0$, the form of which \eqref{eq:V0} has been chosen so as to avoid any strong nonlinearities, allowing us to use the linear calculation described in Sec. \ref{sec:ICs} to calculate the initial fluctuation statistics.
This facilitates the comparison between a simulation run in the potential $V=V_0+\Delta V$ to a baseline simulation run using only the $V_0$ portion of the potential by allowing initial conditions of the two simulations to be matched with identical realizations of field fluctuations. Matching simulations will be explored in detail in Sec. \ref{sec:results}.

% Part about not getting overly focused on the explicit potential form
Recall our goal in setting the potential is to study the phenomenology of $\zeta$ NG generated by a potential feature. The chosen form of the potential given in \eqref{eq:V0} and \eqref{eq:DeltaV} should not be thought of only as a defining a complete inflationary potential, but only as a parameterization within some small region around a feature of interest, in this specific case a feature which breaks symmetry in the transverse direction. So by choosing to aligning $\phi$ with the direction of inflation what we are doing is setting the orienting of $\Delta V$, our parameterization of the potential feature, relative to the direction of flow taken by inflation as it passes this feature, smoothed in a region on the scale of the lattice. %\textcolor{red}{[Reference to figure exploring other potential forms for transverse symmetry breaking]}

%It should be mentioned that while in \eqref{eq:DeltaV} we have specified a functional form of $\Delta V$, our exploration of incomplete phase transition using other functional forms of $\Delta V$ \textcolor{red}{[which were likewise broke/restored symmetry in the transverse direction and were $\pm\chi$ symmetric]} have yielded NG following the same general picture of forming concentrations of NG uncorrelated with the Gaussian part of the field. \textcolor{red}{[Also that the effect persists over a wide range of parameters (growth of fluctuations being exponentially sensitive to $\sqrt{|m^2_{eff}|}$ and duration of instability, but also the displacement of the new minima can be very large compared to the initial size of fluctuations).]}

While in \eqref{eq:DeltaV} we have specified a functional form of $\Delta V$, our exploration of incomplete phase transition using other functional forms of $\Delta V$ parameterizating of a region where symmetry is broken in the transverse direction have yielded NG following the same general picture as \eqref{eq:DeltaV}. That being forming concentrations of NG uncorrelated with the Gaussian part of the field. We will return to alternate paramterizations of a potential yielding an incomplete phase transition in the transverse direction in Sec. \ref{sec:params}. 

\Fpotential

As a final note on this form of potential, although our focus in this paper is on inflation proceeding through an incomplete phase transition the same form of $\Delta V$ can be used to study the case where $\chi$ fluctuations undergo sufficient growth to reach the new transverse minima of the symmetry broken potential and complete the phase transition. The case of inflation proceeding through a completed phase transition proves interesting and we leave its discussion to a dedicated future paper.

\Fpotparam

\textcolor{red}{[Here put the potential parameters for the run and reference to Fig. \ref{fig:potparam} of $\Delta V(\phi=\phi_p,\chi)$ and $V_{,\chi\chi}(\phi,\chi=0)$. Mention what is bounding these choices, ie. not trapping the fields, ending inflation, or completing the phase transition.]}
