% Document for section discussing parameter space.

\section{Parameterization of Instability Strength} \label{sec:params}
% The objective for this section is to layout the regime of validity for the picture of $\Delta\zeta$ given in the results section and to point out directions of future work.
% Introduce the $V = V_0 + u\Delta V$ parameterization as a deformation of the potential with a continuous parameterization.
% Discuss how varying $u$ increases/decreases the strength of the instability.
% Discuss how when the strength of the instability is increased/decreased the character of the asymptotic $\Delta\zeta$ divides the parameter space into regimes (eg. weak, strong, and domain).
% Discuss future work into the other dimensions of parameter space (ie) width measurement and how long the instability is turned on.
% Discuss the qualitative similarity of results for other forms of potential with tachyonic instabilities.

% Figures to make/include:
% Make a figure of $m^2_{eff}(\phi)$ and $u\Delta V(\phi_p,\chi)$ for some sampled values. It could also be useful to put curves for the values of $u$ delineating the different regimes (ie weak, strong, domain).
% Make a rogue's gallery of different runs spanning from small to large values of $u$ showing maps of $\Delta\zeta$
% Make a figure with $u$ along one axis and $\Delta\zeta_\phi/\Delta\zeta_\chi$ along the other.
% Make a figure with $u$ along one axis and a measure of $\Delta\zeta$ amplitude olong the other.

% Things I still need to figure out and do:

%This simplified narrative of a two stage production process of $\Delta\zeta$ highlights the two effects which become most significant to the production of peaks as the strength of the instability is increased. It does not however furnish a complete description of the production of $\Delta\zeta$ in this system for which we must still rely on the full simulation.
In particular as the strength of the instability is decreased the contribution to $\Delta\zeta$ sourced by these effects likewise decrease in significance, and for sufficiently weak instabilities becomes comparable to the contribution from other effects. At this point, while $\Delta\zeta$ can still be calculated from the full numerical solution, the simple two stage production process we have described will no longer provide a useful description. Further exploration of the parameter space bounding the region of applicability for the two stage production process described is left to future work.
\textcolor{red}{[Define what is meant by increasing/decreasing the stength of the instablity in terms of $u\Delta V$.]}

In Sec. \ref{sec:results} we analysed the results of lattice simulations run for a particular set of parameters of $\Delta V$. We found $\Delta\zeta$ to be spatially coherent and characterized by an excess of positive peaks and related the formation of these peaks to a two step production process. % awkward, rework
This simplified narrative of a two stage production process for $\Delta\zeta$ highlights the two effects which become most significant to the production of peaks as the strength of the instability is increased \textcolor{red}{[part about not increasing so much that the phase transition completes]}.
It does not however furnish a complete description of the production of $\Delta\zeta$ in this system for which we must still rely on the full simulation.

In this section we perform a partial survey of parameter space to determine the regime over which the results of the previous section can be considered representative. We will find \textcolor{red}{[basic summary of the regimes we find by varying $u$, situate the results from this paper and point to the areas of parameter space which we will cover in future work]}.

% subsection about ...

% Introduce $u$
% Discuss $u$ in terms of a one dimensional parameterization which continuously deforms the potential between the baseline for $u=0$ and the fiducial for $u=1$ and stronger instabilities for $u>1$.
We extend the potential $V(\phi^A) = V_0(\phi^A) + \Delta V(\phi^A)$ by introducing a single parameter $u$ and defining the parameterized family of potentials
\begin{equation}
  V(\phi^A;u) = V_0(\phi^A) + u\Delta V(\phi^A).
\end{equation}
When $u=0$ we recover the baseline potential $V_0$, taking $u>0$ continuously deforms the potential...
For convieneonce we have chosen to use a parameterization where $u=1$ corresponds to the potential we studied in Sec. \ref{sec:results}...

% Figure of potentials for various values of $u$
%\Fpotparamu
