% Document for section discussing parameter space.

\section{Parameterization of the Instability} \label{sec:params}
%
% Point to a more detailed discussion of parameter space in an upcoming paper and give a brief overview of the differential view of parameter space. Here we just show that the description we give in Sec. \ref{sec:results} is generic for the incipient phase transition and demonstrate where it breaks down ie. domain walls and the weak case.
%
% Includes figures for:
% $\Delta\zeta$ for several sets of $\Delta V$ parameters,
% Effective mass squared and transverse sections of the potential for several sets of $\Delta V$ parameters.
%

\textcolor{blue}{[The aim of this section is to show that form of $\Delta\zeta$ we have discussed is representitive over some range of parameter space and to show/discuss what happens at the edges of that regime of valitity. I think an extended discussion of varying instability parameters belongs in a separate paper and am leaving two open ends to further papers on:
\begin{itemize}
  \item imprinting domain walls on $\zeta$, and
  \item differential variation of $V$ and differential response of $\zeta$.
\end{itemize}
]}

% Intro
In the previous section we analysed simulations run for a single set of potential parameters for the potential given by \eqref{eq:V0} and \eqref{eq:DeltaV}. We found that a two stage production process resulted in a spatially coherent $\Delta\zeta$ characterized by an excess of positive peaks. In this section we will vary the paramters of the instability and discuss to what degree that can be considered a representitive description.

To start we will perform a partial survey of parameter space with the aim to demonstrate the existance of a regime of parameter space over which the results of Sec. \ref{sec:results} can be considered representitive of transverse instabilities during inflation and to show some of the ways the simplified description we have given in terms of the extreme trajectories can break down as be push out of that descriptions regime of validity.

% Control parameters in $\Delta V$
The parameterization we have used for the instability \eqref{eq:DeltaV} has three control paramters which vary it's shape. The width in the transverse direction, or displacement of the new transverse minima from the centre line, is controlled by $v$. The length in the longitudinal direction is controlled by $\phi_w$. And the instability's modification to the effective mass squared of the transverse field is controlled by the combination $\lambda_\chi v^2$. One further parameter, $\phi_p$, controls the position along the longitudinal direction of the potential where the instability is placed.

% Varying control parameters
In Fig. \ref{fig:zetaslicecomp} we have varied $v$, $\phi_w$, and $\lambda_\chi v^2$ through a sequence of simulations and plotted the resulting asymptotic $\Delta\zeta$. The sequence of parameters we have chosen has been designed to show at one extreme the weak case of the instability (as discussed below), and at the other extreme the imprinting of domain walls onto $\zeta$. Between these two extremes are several cases of the incipient phase transition for which the discussion of Sec. \ref{sec:results} is representitive.
\textcolor{red}{[Mention that a thorough exploration of imprinting domain walls is in an upcoming paper.]}
In Fig. \ref{fig:potparamcomp} we have plotted transverse sections of the potential and longitudinal sections of the effective mass squared during the instability corresponding to the choose of parameters used for the runs shown in Fig. \ref{fig:zetaslicecomp}.

% Discuss the weak case
For sufficiently weak or shortlived instabilities, those for which the growth of fluctuations is comparable to their initial variance, the contribution of high frequency effects remains a comparatively large contribution to the dynamics \textcolor{red}{[could be more precise in terms of how much power is stable vs unstable]}.
We call this the `weak case'.
While some notion of extreme trajectories can be recovered on the smoothed fields, in the weak case their contribution to $\Delta\zeta$ becomes comparable to that of other effects.

% Discuss the domain wall case
We have focused on the incipient phase transition, that is to say, the case where the growth of fluctuations during the instability is insufficient to populate the new transverse minima. One the opposite extreme to the weak case the growth of transverse fluctuations reaches the new minima and a quite different form of NG emerges with the formation of domain walls leave a persisting imprint of $\zeta$ even after the symmetry restoration dissolves the domains of the transverse field.

% Discuss trapping, ending, and pausing inflation
\textcolor{red}{[This part needs more checking.]}
There are a number of other ways in which a transverse instability can cause the system to depart from the behaviour we have discussed in Sec. \ref{sec:results} if the fraction of energy transferred out of the potential becomes large.
An instability sufficiently deep to create local minima may trap the fields in a false vacuum.
Or the instabilty may bring the fields to a global minimum, ending inflation and begining reheating.
Or if a large enough energy fraction is transferred from the potential into kinetic and gradient terms, $\epsilon$ will be driven to greater values, potentially exceeding unity and bringing a temporary halt to inflation until the redshifting of the kinetic and gradient energy allows inflation to restart.

\Fzetaslicecomp

\Fpotparamcomp

% Future work in parameter space derivatives
With the sequence of simulations shown in Fig. \ref{fig:zetaslicecomp} we can also observe that with matched realizations of initial fluctuations the NG generated varies continuously as the potential is deformed. In light of this continuity, the $\Delta\zeta$ formalism we have introduced can be extended from the finite difference in $\zeta$ when comparing to a baseline potential to a differential modification to $\zeta$ as the potential is continuously deformed
\begin{equation}
  \pder{\zeta}{u}
\end{equation}
where $u$ is any one of the potential paramters.
This derivative then informs us as to which directions in parameter space are most relevant to the production of NG when deforming the potential. An exploration of parameter space for transverse instabilities during inflation taking this view point of the NG response to differential deformations of the potential is the topic of a follow up paper. 

% extreme trajectories rapidly oscillating, ie low fraction of power undergoing Jeans instability. Or effect size from extreme trajectories not larger than other effects
% formation of domains with nonlinearity in the transverse direction
% Inflation can be ended, either temporarily with a large shift to the equation of state or permanently as in hybrid inflation.
%

%This simplified narrative of a two stage production process of $\Delta\zeta$ highlights the two effects which become most significant to the production of peaks as the strength of the instability is increased. It does not however furnish a complete description of the production of $\Delta\zeta$ in this system for which we must still rely on the full simulation.
%In particular as the strength of the instability is decreased the contribution to $\Delta\zeta$ sourced by these effects likewise decrease in significance, and for sufficiently weak instabilities becomes comparable to the contribution from other effects. At this point, while $\Delta\zeta$ can still be calculated from the full numerical solution, the simple two stage production process we have described will no longer provide a useful description. Further exploration of the parameter space bounding the region of applicability for the two stage production process described is left to future work.

% This part belongs in the section on sourcing $\Delta\zeta$.
%This simplified narrative of a two stage production process for $\Delta\zeta$ highlights the two effects which become most significant to the production of peaks as the strength of the instability is increased \textcolor{red}{[part about not increasing so much that the phase transition completes]}.
%It does not however furnish a complete description of the production of $\Delta\zeta$ in this system for which we must still rely on the full simulation.

% Figure of potentials for various values of $u$


%\subsection{Alternate Forms of $\Delta V$}
%\textcolor{red}{[Place holder. In this section I will discuss alternate forms of $\Delta V$ that could be used be used to parameterize an incomplete phase transition in the transverse direction. The main goal being to demonstrate the broad similarities in the forms of NG generated.]}

%\textcolor{red}{[Give equations for a few alternate forms of $\Delta V$. Discuss Fig. \ref{fig:zetaslicepotform} which will display $\Delta\zeta$ slices for the alternate forms of $\Delta V$ given.]}

%\textcolor{red}{[Discuss similarities in $\Delta\zeta$ in terms of the relaxation stage of production from a deformed state with reference to Fig. \ref{fig:deltapspotform}. (I haven't actually checked this for the other potential forms we've looked at, but it would be nice to show if it works.)]}
