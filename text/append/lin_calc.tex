% lin_calc.tex

% .tex document for an appendix on the linear calculation used when setting initial conditions on the lattice.

\section{Background Equations}
We use a set of background equations to solve for the attractor solution before initializing the perturbative calculation for the spectra from which the lattice is initialized. These equations are solved in cosmic time using the variables $\alpha$, $H$, $\phi^A$, $\dot{\phi}^A$, for which the equations of motion are
\begin{align}
  \frac{\dd\alpha}{\dd t} &= H, \\
  \frac{\dd H}{\dd t} &= -\frac{1}{2}\sum_A\dot{\phi}^{A^2}, \\
  \frac{\dd\phi^A}{\dd t} &= \dot{\phi}^A, \\
  \frac{\dd\dot{\phi}^A}{\dd t} &= -3H\dot{\phi}^A - \frac{\partial V}{\partial\phi^A}.
\end{align}
These equations differ in two ways from those which describe the evolution of the mean fields on the lattice. First, the value of $H$ used in these equations satisfies a Hubble constraint equation which does not include any contribution from the energy density of fluctuations and so differs from the value used on the lattice. And second, the parameters of the potential used in the background equations are identical to those used on the lattice rather than renormalized values. We have ignored any small corrections to our calculations owing to these effects.

We use the slow roll approximation to set initial conditions for the background equations. Solving for the attractor is insensitive initial conditions so long as they are set in a region where an attractor solution exists and $H$ is chosen consistently with the Hubble constraint. 

For both the background equations and the perturbative equations detailed in the next section we use an $8^\mathrm{th}$ order Gauss-Legendre scheme to perform the integration.

\section{Perturbative Calculation of Spectra}
%We define $a\mathcal{N}\tilde{\phi}^A = L_{AB}x_B$ with $\mathcal{N}$ being a constant factor, by enforcing the constraint $\langle x_Ax_B \rangle = \delta_{AB}$ we can interpret $L_{AB}$.

We define $a\mathcal{N}\tilde{\phi}^A = L_{AB}x_B$, where $L_{AB}$ is a real value matrix, $x_{B}$ is a vector of operators, and $\mathcal{N}$ is a normalization factor. By enforcing the constraint $\langle x_Ax_B \rangle = \delta_{AB}$ we can interpret $L_{AB}$ as ...

The calculation is done alongside the background calculation using the variables $\L_{AB}$, $\dot{L}_{AB}$, $\langle x_A\dot{x}_B \rangle$, and $(\langle \dot{x}_A\dot{x}_B \rangle - k^2/a^2\delta{AB})$.

The equations of motion are
\begin{align}
  \frac{\dd}{\dd t} L_{AB} = \dot{L}_{AB},
\end{align}
\begin{align}
  \begin{split}
  \frac{\dd}{\dd t} \dot{L}_{AB} = & L_{AC}A_{CB} - B_{AB} \\
    & + C_{AC}\langle x_C\dot{x}_B \rangle + L_{AC}(\langle \dot{x}_C\dot{x}_B \rangle - k^2/a^2),
  \end{split}
\end{align}
\begin{align}
  \frac{\dd}{\dd t} \langle x_A\dot{x}_B \rangle = A_{AB},
\end{align}
\begin{multline}
  \frac{\dd}{\dd t} (\langle \dot{x}_A\dot{x}_B \rangle - k^2/a^2\delta_{AB}) \\= 
       [\delta_{AC}\delta_{BD} + \delta_{BC}\delta_{AD}]
       [-\langle x_C\dot{x}_E \rangle A_{ED}
         - (L^{-1})_{CE}C_{EF}\langle x_F\dot{x}_G \rangle\langle x_G\dot{x}_D \rangle\\
         + (\langle x_C\dot{x}_F \rangle - (L^{-1})_{CE}C_{EF})(\langle \dot{x}_F\dot{x}_F \rangle - k^2/a^2\delta_{FD})\\
         + (\langle x_C\dot{x}_D \rangle - (L^{-1})_{CE}C_{ED} + H\delta_{CD})k^2/a^2].
\end{multline}
Where the matricies $A_{AB}$, $B_{AB}$, $C_{AB}$ are given by
\begin{align}
  A_{AB} & = \\
  B_{AB} &= -(\dot{H} + 2H^2)L_{AB} + \frac{\partial^2V}{\partial\phi^A\partial\phi^C}L_{CB} + H\dot{L}_{AB}, \\
  C_{AB} &= 2\dot{L}_{AB} + HL_{AB}.
\end{align}

Intitial conditions are set by ...

The 2-pt correlations of the lattice variables are set by ...
