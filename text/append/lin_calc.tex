% lin_calc.tex

% .tex document for an appendix on the linear calculation used when setting initial conditions on the lattice.

\section{Background Equations}
We use a set of background equations to solve for the attractor solution before initializing the perturbative calculation for the spectra from which the lattice is initialized. These equations are solved in cosmic time using the variables $\alpha$, $H$, $\phi^A$, $\dot{\phi}^A$, for which the equations of motion are
\begin{align}
  & \frac{\dd\alpha}{\dd t} = H, \\
  & \frac{\dd H}{\dd t} = -\frac{1}{2}\sum_A\dot{\phi}^{A^2}, \\
  & \frac{\dd\phi^A}{\dd t} = \dot{\phi}^A, \\
  & \frac{\dd\dot{\phi}^A}{\dd t} = -3H\dot{\phi}^A - \frac{\partial V}{\partial\phi^A}.
\end{align}
These equations differ in two ways from those which describe the evolution of the mean fields on the lattice. First, the value of $H$ used in these equations satisfies a Hubble constraint equation which does not include any contribution from the energy density of fluctuations and so differs from the value used on the lattice. And second, the parameters of the potential used in the background equations are identical to those used on the lattice rather than renormalized values. We have ignored any small corrections to our calculations owing to these effects.

We use the slow roll approximation to set initial conditions for the background equations. Solving for the attractor is insensitive initial conditions so long as they are set in a region where an attractor solution exists and $H$ is chosen consistently with the Hubble constraint. 

For both the background equations and the perturbative equations detailed in the next section we use an $8^\mathrm{th}$ order Gauss-Legendre scheme to perform the integration.

\section{Perturbative Calculation of Spectra}
% Introductory line about how we don't need to calculate the full solution, but only the real part of the spectra.
% put in \dagger instead of * where needed
% figure out notation for inner product

To set the initial conditions for fluctuations of the fields and their momenta on the lattice we realize a set of Gaussian random fields drawn from a distribution which matches the real part of the Fourier space two-point correlation functions as determined by linear theory. In this section we detail linear calculation of these correlations.

% I'm suppressing the fourier mode index, but maybe it makes sense to include.
To start we define $a\mathcal{N}\tilde{\phi}^A = L_{AB}x_B$, here $x_{B}$ is a vector of operators and $L_{AB}$ is a real valued lower triangular matrix, $\mathcal{N}$ is a normalization factor. Our goal is now to compute the correlations $\langle\tilde{\phi}^{A*}\tilde{\phi}^B\rangle_\mathrm{Re}$, $\langle\tilde{\Pi}^{A*}\tilde{\phi}^B\rangle_\mathrm{Re}$, and $\langle\tilde{\Pi}^{A*}\tilde{\Pi}^B\rangle_\mathrm{Re}$, where $\langle\cdot\rangle_\mathrm{Re}$ denotes the real part of the correlation. Enforcing the constraint $\langle x_Ax_B \rangle_\mathrm{Re} = \delta_{AB}$ we can write these correlations as
\begin{align}
  & \langle\tilde{\phi}^{A*}\tilde{\phi}^B\rangle_\mathrm{Re} =  a^{-2}\mathcal{N}^{-2}[L_{AC}L_{CB}^T], \\
  & \langle\tilde{\Pi}^{A*}\tilde{\phi}^B\rangle_\mathrm{Re} = a\mathcal{N}^{-2}[HL_{AC}L_{CB}^T + L_{AC}\dot{L}_{CB}^T \nonumber \\
  & \hphantom{\langle\tilde{\Pi}^{A*}\tilde{\phi}^B\rangle_\mathrm{Re} =} + L_{AC}\langle x_C\dot{x}_D\rangle_\mathrm{Re}L_{DB}^T], \\
  & \langle\tilde{\Pi}^{A*}\tilde{\Pi}^B\rangle_\mathrm{Re} =  a^4\mathcal{N}^{-2}[H^2L_{AC}L_{CB}^T + \dot{L}_{AC}\dot{L}_{CB}^T \nonumber \\
  & \hphantom{\langle\tilde{\Pi}^{A*}\tilde{\Pi}^B\rangle_\mathrm{Re} =} + L_{AC}\langle \dot{x}_C\dot{x}_D\rangle_\mathrm{Re}L_{DB}^T + HL_{AC}\dot{L}_{CB}^T \nonumber \\
  & \hphantom{\langle\tilde{\Pi}^{A*}\tilde{\Pi}^B\rangle_\mathrm{Re} =} + H\dot{L}_{AC}L_{CB}^T + \dot{L}_{AC}\langle x_C\dot{x}_D\rangle_\mathrm{Re}L_{CB}^T \nonumber \\
  & \hphantom{\langle\tilde{\Pi}^{A*}\tilde{\Pi}^B\rangle_\mathrm{Re} =} - L_{AC}\langle x_C\dot{x}_D\rangle_\mathrm{Re}\dot{L}_{CB}^T].
\end{align}
Since we are concerned only with the real part of these correlations any reference to the commutators $[\tilde{\phi}^A, \tilde{\Pi}^B]$ and $[x_A,\dot{x}_B]$ can be dropped as their contribution to the two-point correlations is imaginary.

% setting ICs and evolution
% along with background eom forms a closed set of equations

To compute these correlations at the point when the lattice is initialized we work in the variables $L_{AB}$, $\dot{L}_{AB}$, $\langle x_A\dot{x}_B \rangle_\mathrm{Re}$, and $\overline{\langle \dot{x}_A\dot{x}_B \rangle}_\mathrm{Re} \equiv \langle \dot{x}_A\dot{x}_B \rangle_\mathrm{Re} - k^2/a^2\delta_{AB}$. Initializing these variables at a fixed $\frac{k}{aH}$ well within the horizon they can be set to match the Minkowski space correlations in term of the conformally scaled fields $\varphi^A \equiv a\phi^A$ and their conjugate momenta $\Pi_\varphi^A$ we can then evolve them forward to the point when the lattice is to be initialized. 

To do this we need, in addition to the background equations of motion \textcolor{red}{[put in equation numbers]}, equations of motion for $L_{AB}$, $\dot{L}_{AB}$, $\langle x_A\dot{x}_B \rangle_\mathrm{Re}$, and $\overline{\langle \dot{x}_A\dot{x}_B \rangle}_\mathrm{Re}$. These equations of motion can be derived from the linearized equations of motion for $a\tilde{\phi}^A$ and the constraint $\langle x_Ax_B \rangle_\mathrm{Re} = \delta_{AB}$. After some work one finds
\begin{align}
  & \frac{\dd}{\dd t} L_{AB} = \dot{L}_{AB}, \\
  & \frac{\dd}{\dd t} \dot{L}_{AB} = L_{AC}A_{CB} - B_{AB} + C_{AC}\langle x_C\dot{x}_B \rangle_\mathrm{Re} \nonumber \\
  & \hphantom{\frac{\dd}{\dd t} \dot{L}_{AB} = } + L_{AC}\overline{\langle \dot{x}_C\dot{x}_B \rangle}_\mathrm{Re}, \\
  & \frac{\dd}{\dd t} \langle x_A\dot{x}_B \rangle_\mathrm{Re} = A_{AB}, \\
  & \frac{\dd}{\dd t} \overline{\langle \dot{x}_A\dot{x}_B \rangle}_\mathrm{Re} = [\delta_{AC}\delta_{BD} + \delta_{BC}\delta_{AD}][-\langle x_C\dot{x}_E \rangle_\mathrm{Re} A_{ED} \nonumber  \\
    & \hphantom{\frac{\dd}{\dd t} \overline{\langle \dot{x}_A\dot{x}_B \rangle}_\mathrm{Re} =} - (L^{-1})_{CE}C_{EF}\langle x_F\dot{x}_G \rangle_\mathrm{Re}\langle x_G\dot{x}_D \rangle_\mathrm{Re} \nonumber \\
    & \hphantom{\frac{\dd}{\dd t} \overline{\langle \dot{x}_A\dot{x}_B \rangle}_\mathrm{Re} =} + (\langle x_C\dot{x}_F \rangle_\mathrm{Re} - (L^{-1})_{CE}C_{EF})\overline{\langle \dot{x}_F\dot{x}_D \rangle}_\mathrm{Re} \nonumber \\
    & \hphantom{\frac{\dd}{\dd t} \overline{\langle \dot{x}_A\dot{x}_B \rangle}_\mathrm{Re} =} + (\langle x_C\dot{x}_D \rangle_\mathrm{Re} - (L^{-1})_{CE}C_{ED} + H\delta_{CD}) \nonumber \\
    & \hphantom{\frac{\dd}{\dd t} \overline{\langle \dot{x}_A\dot{x}_B \rangle}_\mathrm{Re} =} k^2/a^2].
\end{align}
Where the matrices $A_{AB}$, $B_{AB}$, $C_{AB}$ are given by
\begin{align}
  A_{AB} & =
  \begin{cases}
    -(L^{-1})_{AC}\frac{\partial^2V}{\partial\phi^C\partial\phi^D}L_{DB} \\
    + (L^{-1})_{AC}C_{CD}\langle x_D\dot{x}_B\rangle_\mathrm{Re} \\
    + \overline{\langle \dot{x}_A\dot{x}_B \rangle}_\mathrm{Re}, &  \text{if} \quad A>B \\
    0, &  \text{if} \quad A=B \\
    -A_{AB}^T, & \text{if} \quad A<B \\
  \end{cases}, \\
  B_{AB} & = -(\dot{H} + 2H^2)L_{AB} + \frac{\partial^2V}{\partial\phi^A\partial\phi^C}L_{CB} + H\dot{L}_{AB}, \\
  C_{AB} & = 2\dot{L}_{AB} + HL_{AB}.
\end{align}
The form of $A_{AB}$ is set by requiring $L_{AB}$ to maintain a lower triangular form.

Returning now to setting the initial conditions to match the Minkoswki space correlations well within the horizon, the correlations we would like to match are, in terms of $\varphi^A \equiv a\phi^A$ and $\Pi_\varphi^A$, 
\begin{align}
  & \langle\tilde{\varphi}^A\tilde{\varphi}^B\rangle_\mathrm{Re} = \frac{1}{2}U_{AC}[k^2\delta_{CD} + a^2D^2_{CD} \nonumber \\
    & \hphantom{\langle\tilde{\varphi}^A\tilde{\varphi}^B\rangle_\mathrm{Re} =}- a^2(2H^2 + \dot{H})\delta_{CD}]^{-1/2}U_{DB}^T, \label{eq:mink corr ff 1} \\
  & \langle\tilde{\Pi}_\varphi^A\tilde{\varphi}^B\rangle_\mathrm{Re} = 0, \label{eq:mink corr fdf 1} \\
  & \langle\tilde{\Pi}_\varphi^A\tilde{\Pi}_\varphi^B\rangle_\mathrm{Re} = \frac{1}{2}U_{AC}[k^2\delta_{CD} + a^2D^2_{CD} \nonumber \\
    & \hphantom{\langle\tilde{\Pi}_\varphi^A\tilde{\Pi}_\varphi^B\rangle_\mathrm{Re} =}- a^2(2H^2 + \dot{H})\delta_{CD}]^{1/2}U_{DB}^T. \label{eq:mink corr dfdf 1}
\end{align}
Where $U_{AB}$ is the unitary matrix which diagonalizes the mass matrix as $D^2_{AB} \equiv U_{AC}^T\frac{\partial^2V}{\partial\phi^C\partial\phi^D}U_{CD}$ \textcolor{red}{[check factors of $a$ on mass matrix]}. \textcolor{red}{[These assume the mass matrix is not time dependent.]}
Or, in terms of the variables $L_{AB}$, $\dot{L}_{AB}$, $\langle x_A\dot{x}_B \rangle_\mathrm{Re}$, and $\overline{\langle \dot{x}_A\dot{x}_B \rangle}_\mathrm{Re}$,
\begin{align}
  & \langle\tilde{\varphi}^A\tilde{\varphi}^B\rangle_\mathrm{Re} = \mathcal{N}^{-2}L_{AC}L_{CB}^T, \label{eq:mink corr ff 2} \\
  & \langle\tilde{\Pi}_\varphi^A\tilde{\varphi}^B\rangle_\mathrm{Re} = a\mathcal{N}^{-2}[L_{AC}\dot{L}_{CB}^T + L_{AC}\langle x_C\dot{x}_D\rangle_\mathrm{Re}L_{DB}^T], \label{eq:mink corr fdf 2}\\
  & \langle\tilde{\Pi}_\varphi^A\tilde{\Pi}_\varphi^B\rangle_\mathrm{Re} = a^2\mathcal{N}^{-2}[\dot{L}_{AC}\dot{L}_{CB}^T + \dot{L}_{AC}\langle x_C\dot{x}_D\rangle_\mathrm{Re}L_{DB}^T \nonumber \\
    & \hphantom{\langle\tilde{\Pi}_\varphi^A\tilde{\Pi}_\varphi^B\rangle_\mathrm{Re} =} - L_{AC}\langle x_C\dot{x}_D\rangle_\mathrm{Re}\dot{L}_{DB}^T \nonumber \\
    & \hphantom{\langle\tilde{\Pi}_\varphi^A\tilde{\Pi}_\varphi^B\rangle_\mathrm{Re} =} + L_{AC}(\overline{\langle \dot{x}_C\dot{x}_D\rangle}_\mathrm{Re} + k^2/a^2\delta_{CD})L_{DB}^T]. \label{eq:mink corr dfdf 2}
\end{align}

Equations \eqref{eq:mink corr ff 1} and \eqref{eq:mink corr ff 2} are satisfied by setting $L_{AB}$ by Choleski decomposition of the right hand side of \eqref{eq:mink corr ff 1}. Equations \eqref{eq:mink corr fdf 1} and \eqref{eq:mink corr fdf 2} can be satisfied by setting $\dot{L}_{AB} = 0$ and $\langle x_A\dot{x}_B\rangle_\mathrm{Re} = 0$. Then \eqref{eq:mink corr dfdf 1} and \eqref{eq:mink corr dfdf 2} can be satisfied be setting $\overline{\langle \dot{x}_A\dot{x}_B\rangle}_\mathrm{Re} = a^{-2}\mathcal{N}^2(L^{-1})_{AC}\langle\tilde{\Pi}_\varphi^A\tilde{\Pi}_\varphi^B\rangle_\mathrm{Re}(L^{-1})^T_{DB} - k^2/a^2\delta_{AB}$.


