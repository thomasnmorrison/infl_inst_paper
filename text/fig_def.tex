% Define commands for including figures.

\graphicspath{{./figures/}}

% Figure of the potential
\newcommand{\Fpotential}{
  \begin{figure}
    \centering
    \begin{subfigure}[a]{0.45\textwidth}
      \includegraphics[width=\textwidth]{\FFpotentialA}
      \caption{Potential surface}
    \end{subfigure}
    \begin{subfigure}[a]{0.45\textwidth}
      \includegraphics[width=\textwidth]{\FFpotentialB}
      \caption{Potential contours}
    \end{subfigure}
    \begin{subfigure}[b]{0.45\textwidth}
      \includegraphics[width=\textwidth]{\FFpotentialC}
      \caption{(Un)Stable regions}
    \end{subfigure}
    \begin{subfigure}[b]{0.45\textwidth}
      \includegraphics[width=\textwidth]{\FFpotentialD}
      \caption{Potential gradients}
    \end{subfigure}
    \caption{Place holder potential figures. Would be more clear if there was a plot showing several transverse slices through the potential. Need to update this plot for the potential form being used. I also like the idea of a two panel figure that shows that shows our potential feature as part of some more complicated potential in one panel and then embeded in $V_0$ in the second panel to make the point that we are trying to constrain features in the potential as opposed to the potential as a whole.}
    \label{fig:potential}
  \end{figure}
}

% Figure of the field and momenta spectra and cross-spectra
\newcommand{\Fspec}{
  \begin{figure}
    \centering
    \begin{subfigure}[a]{0.45\textwidth}
        \includegraphics[width=\textwidth]{\FFspecA}
        \caption{$\phi$ and $\Pi^\phi$ (cross)spectra}
    \end{subfigure}
    \begin{subfigure}[a]{0.45\textwidth}
        \includegraphics[width=\textwidth]{\FFspecB}
        caption{$\chi$ and $\Pi^\chi$ (cross)spectra}
    \end{subfigure}
    \caption{Spectra and cross-spectra for the fields. The dashed vertical green lines are positonaed and the start, centre, and end of $\Delta V$. The blue line is $k=aH$. Negative values of the cross-spectra are plotted with dotted contours. \\ This figure still needs to be connected to the main text. The main point is to see where the instability bands are located and to note $\phi$ has an instability near the end of $\Delta V$ which would not be predicted by linear theory around a background with $\chi=0$. There are still some mysteries about the secondary bands.\\ Place holder figure. Some things to change: should do this for a run with more superhorizon modes, Should use some normalizing factor (maybe Hawking temperature) for the spectra and label contours.}
    \label{fig:spec}
  \end{figure}
}

% Figure of the field and momenta spectra determinant
\newcommand{\Fspecdet}{
  \begin{figure}
    \centering
    \begin{subfigure}[a]{0.45\textwidth}
      \includegraphics[width=\textwidth]{\FFspecdetA}
      \caption{}
    \end{subfigure}
    \begin{subfigure}[a]{0.45\textwidth}
      \includegraphics[width=\textwidth]{\FFspecdetB}
      \caption{}
    \end{subfigure}
    \caption{Determinant of of the fluctuation correlation matrix $C(\phi^A,\Pi^A)$ as function of time (clocked by $\alpha$). This quantity is conserved whenever the systems dynamics are linear. \\ Place holder figure. It is probably more clear to use a contour plot here instead of a plethera of sampled $k$.}
    \label{fig:specdet}
  \end{figure}
}

% Figure of a 2d slice of the fields at times before, during, and after nonlinearity as identified in \ref{fig:spec det} showing the foramation and backreaction of the transverese condensate.
\newcommand{\Fslice}{
  \begin{figure}
    \centering
    \includegraphics[width=\textwidth]{\FFslice}
    \caption{Real space field fluctuations $\delta\phi$, $\chi$ and their time derivatives $\delta\dot{\phi}$, $\dot{\chi}$ on a two dimensional slice of a three dimensional lattice, shown at the times slices identidied in figure \ref{fig:specdet}.}
    \label{fig:slice}
  \end{figure}
}

% Figure of phase space projection of the fields and momenta at times before, during, and after nonlineraity as identified in \ref{fig:spec det} showing the correlations that arise.
\newcommand{\Fphasespaceold}{
  \begin{figure}
    \centering
    \begin{subfigure}[a]{0.3\textwidth}
      \includegraphics[width=\textwidth]{\FFphasespaceAold}
      \caption{}
    \end{subfigure}
    \begin{subfigure}[a]{0.3\textwidth}
      \includegraphics[width=\textwidth]{\FFphasespaceBold}
      \caption{}
    \end{subfigure}
    \begin{subfigure}[a]{0.3\textwidth}
      \includegraphics[width=\textwidth]{\FFphasespaceCold}
      \caption{}
    \end{subfigure}
    \begin{subfigure}[b]{0.3\textwidth}
      \includegraphics[width=\textwidth]{\FFphasespaceDold}
      \caption{}
    \end{subfigure}
    \begin{subfigure}[b]{0.3\textwidth}
      \includegraphics[width=\textwidth]{\FFphasespaceEold}
      \caption{}
    \end{subfigure}
    \begin{subfigure}[b]{0.3\textwidth}
      \includegraphics[width=\textwidth]{\FFphasespaceFold}
      \caption{}
    \end{subfigure}
    \caption{Phase space projections of the fields and their momenta shown athe the time slices identified in figure \ref{fig:specdet}.}
    \label{fig:phasespace}
  \end{figure}
}

\newcommand{\Fphasespace}{
  \begin{figure}
    \centering
    \begin{subfigure}[a]{0.45\textwidth}
      \includegraphics[width=\textwidth]{\FFphasespaceA}
      \caption{}
    \end{subfigure}
    \begin{subfigure}[a]{0.45\textwidth}
      \includegraphics[width=\textwidth]{\FFphasespaceB}
      \caption{}
    \end{subfigure}
    \begin{subfigure}[b]{0.45\textwidth}
      \includegraphics[width=\textwidth]{\FFphasespaceC}
      \caption{}
    \end{subfigure}
    \begin{subfigure}[b]{0.45\textwidth}
      \includegraphics[width=\textwidth]{\FFphasespaceD}
      \caption{}
    \end{subfigure}
    \begin{subfigure}[c]{0.45\textwidth}
      \includegraphics[width=\textwidth]{\FFphasespaceE}
      \caption{}
    \end{subfigure}
    \begin{subfigure}[c]{0.45\textwidth}
      \includegraphics[width=\textwidth]{\FFphasespaceF}
      \caption{}
    \end{subfigure}
    \caption{Phase space projections of the fields and their momenta shown at the time slices identified in figure \ref{fig:specdet}.}
    \label{fig:phasespace}
  \end{figure}
}

% Figure of real space slices of $\Delta\zeta$ at specified times.
\newcommand{\Fzetaslice}{
  \begin{figure}
    \centering
    \includegraphics[width=\textwidth]{\FFzetaslice}
    \caption{Real space slice of $\Delta\zeta$ at specified times. The scale bar in the lower left of each panel shows the horizon scale $(aH)^{-1}$. The formation of peaks is clearly visible.}
    \label{fig:zetaslice}
  \end{figure}
}

% Figure of relative abundance of peaks in $\Delta\zeta$ with smoothin at horizon scale at end of sim.
\newcommand{\Fzetapeak}{
  \begin{figure}
    \centering
    \includegraphics[width=\textwidth]{\FFzetapeak}
    \caption{Relative abunance of positive (blue) and negative (red) peaks in $\Delta\zeta$ on the final time step of the simulation filtered at the horizon scale $(aH)^{-1}$ plotted as a function of $\nu$.}
    \label{fig:zetapeak}
  \end{figure}
}

% Figure for 2d histograms of $\dd\zeta/\dd\alpha$ against $\phi^A$, $\Pi^A$.
\newcommand{\Fdzetaphasespace}{
  \begin{figure}
    \centering
    \begin{subfigure}[a]{0.45\textwidth}
      \includegraphics[width=\textwidth]{\FFdzetaphasespaceA}
      \caption{}
    \end{subfigure}
    \begin{subfigure}[a]{0.45\textwidth}
      \includegraphics[width=\textwidth]{\FFdzetaphasespaceB}
      \caption{}
    \end{subfigure}
    \begin{subfigure}[b]{0.45\textwidth}
      \includegraphics[width=\textwidth]{\FFdzetaphasespaceC}
      \caption{}
    \end{subfigure}
    \begin{subfigure}[b]{0.45\textwidth}
      \includegraphics[width=\textwidth]{\FFdzetaphasespaceD}
      \caption{}
    \end{subfigure}
    \caption{Histograms of $P(\frac{\dd\Delta\zeta}{\dd\alpha},\phi^A)$ and $P(\frac{\dd\Delta\zeta}{\dd\alpha},\Pi^A)$ at specified times are plotted in gray contours $P(\frac{\dd\Delta\zeta}{\dd\alpha},\phi^A|\Delta\zeta_{\mathrm{final}}>5\sigma)$ and $P(\frac{\dd\Delta\zeta}{\dd\alpha},\Pi^A|\Delta\zeta_{\mathrm{final}}>5\sigma)$ are overlayed in colour. The probability density spacing between contours is equivalent to a $0.5\sigma$ step for a Gaussian PDF matching the mode of the measured historgram.}
    \label{fig:dzetaphasespace}
  \end{figure}
}

% Figure for 2d histogram of $\phi^A$, $\Pi^A$ conditioned on $\Delta\zeta$.
\newcommand{\Fzetacondps}{
  \begin{figure}
    \centering
    \begin{subfigure}[a]{0.45\textwidth}
      \includegraphics[width=\textwidth]{\FFzetacondpsA}
      \caption{}
    \end{subfigure}
    \begin{subfigure}[a]{0.45\textwidth}
      \includegraphics[width=\textwidth]{\FFzetacondpsB}
      \caption{}
    \end{subfigure}
    \begin{subfigure}[b]{0.45\textwidth}
      \includegraphics[width=\textwidth]{\FFzetacondpsC}
      \caption{}
    \end{subfigure}
    \begin{subfigure}[b]{0.45\textwidth}
      \includegraphics[width=\textwidth]{\FFzetacondpsD}
      \caption{}
    \end{subfigure}
    \caption{Histograms of phase space projections at specified times. The gray contours are for the whole field, colour contours are conditioned on $\Delta\zeta_{\mathrm{final}}>5\sigma$. The probability density spacing between contours is equivalent to a $0.5\sigma$ step for a Gaussian PDF matching the mode of the measured historgram.}
    \label{fig:zetacondps}
  \end{figure}
}
