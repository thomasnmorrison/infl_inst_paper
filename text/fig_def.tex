% Define commands for including figures.

\graphicspath{{./figures/}}

% Figure of the potential
\newcommand{\Fpotential}{
  \begin{figure}
    \centering
    \begin{subfigure}[a]{0.45\textwidth}
      \includegraphics[width=\textwidth]{\FFpotentialA}
      \caption{Potential surface}
    \end{subfigure}
    \begin{subfigure}[a]{0.45\textwidth}
      \includegraphics[width=\textwidth]{\FFpotentialB}
      \caption{Potential contours}
    \end{subfigure}
    \begin{subfigure}[b]{0.45\textwidth}
      \includegraphics[width=\textwidth]{\FFpotentialC}
      \caption{(Un)Stable regions}
    \end{subfigure}
    \begin{subfigure}[b]{0.45\textwidth}
      \includegraphics[width=\textwidth]{\FFpotentialD}
      \caption{Potential gradients}
    \end{subfigure}
    \caption{Place holder potential figures. Would be more clear if there was a plot showing several transverse slices through the potential.}
    \label{fig:potential}
  \end{figure}
}

% Figure of the field and momenta spectra and cross-spectra
\newcommand{\Fspec}{
  \begin{figure}
    \centering

    \caption{Place holder figure}
    \label{fig:spec}
  \end{figure}
}

% Figure of the field and momenta spectra determinant
\newcommand{\Fspecdet}{
  \begin{figure}
    \centering
    \begin{subfigure}[a]{0.45\textwidth}
      \includegraphics[width=\textwidth]{\FFspecdetA}
      \caption{}
    \end{subfigure}
    \begin{subfigure}[a]{0.45\textwidth}
      \includegraphics[width=\textwidth]{\FFspecdetB}
      \caption{}
    \end{subfigure}
    \caption{Determinant of of the fluctuation correlation matrix as function of time (clocked by $\alpha$). This quantity is conserved whenever the systems dynamics are linear.}
    \label{fig:specdet}
  \end{figure}
}

% Figure of a 2d slice of the fields at times before, during, and after nonlinearity as identified in \ref{fig:spec det} showing the foramation and backreaction of the transverese condensate.
\newcommand{\Fslice}{
  \begin{figure}
    \centering
    \includegraphics[width=\textwidth]{\FFslice}
    \caption{Real space field fluctuations $\delta\phi$, $\chi$ and their time derivatives $\delta\dot{\phi}$, $\dot{\chi}$ on a two dimensional slice of a three dimensional lattice, shown at the times slices identidied in figure \ref{fig:specdet}.}
    \label{fig:slice}
  \end{figure}
}

% Figure of phase space projection of the fields and momenta at times before, during, and after nonlineraity as identified in \ref{fig:spec det} showing the correlations that arise.
\newcommand{\Fphasespace}{
  \begin{figure}
    \centering
    \begin{subfigure}[a]{0.3\textwidth}
      \includegraphics[width=\textwidth]{\FFphasespaceA}
      \caption{}
    \end{subfigure}
    \begin{subfigure}[a]{0.3\textwidth}
      \includegraphics[width=\textwidth]{\FFphasespaceB}
      \caption{}
    \end{subfigure}
    \begin{subfigure}[a]{0.3\textwidth}
      \includegraphics[width=\textwidth]{\FFphasespaceC}
      \caption{}
    \end{subfigure}
    \begin{subfigure}[b]{0.3\textwidth}
      \includegraphics[width=\textwidth]{\FFphasespaceD}
      \caption{}
    \end{subfigure}
    \begin{subfigure}[b]{0.3\textwidth}
      \includegraphics[width=\textwidth]{\FFphasespaceE}
      \caption{}
    \end{subfigure}
    \begin{subfigure}[b]{0.3\textwidth}
      \includegraphics[width=\textwidth]{\FFphasespaceF}
      \caption{}
    \end{subfigure}
    \caption{Phase space projections of the fields and their momenta shown athe the time slices identified in figure \ref{fig:specdet}.}
    \label{fig:phasespace}
  \end{figure}
}
