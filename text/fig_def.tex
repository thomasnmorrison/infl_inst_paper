% Define commands for including figures.

% Figure files can be downloaded from:
% https://www.cita.utoronto.ca/~morrison/downloads/Paper_figs.tar.gz

\graphicspath{{../figures/}{./figures/}{./Paper_figs}}
\captionsetup{parskip=5pt}
%\captionsetup[figure]{justification=justified}
%\captionsetup[subfigure]{justification=justified}

% fig:potential
\newcommand{\Fpotential}{
  \begin{figure*}
    \centering
    \begin{subfigure}[a]{0.45\textwidth}
      \includegraphics[width=\textwidth]{\FFpotentialA}
      \caption{}
    \end{subfigure}
    \begin{subfigure}[a]{0.45\textwidth}
      \includegraphics[width=\textwidth]{\FFpotentialB}
      \caption{}
    \end{subfigure}
    \caption{Surface plots of $V_0$ and $V=V_0+\Delta V$ as defined by \eqref{eq:V0} and \eqref{eq:DeltaV}, shown here with potential parameters scaled arbitrarily for visual clarity. \textcolor{red}{[Talk about how $V$ and $V_0$ are the same outside of the region $\phi\in(\phi_p-\phi_w,\phi_p+\phi+w)$, but within this region $\Delta V$ drives an instability.]}}
    \label{fig:potential}
  \end{figure*}
}

% fig:potparam
\newcommand{\Fpotparam}{
  \begin{figure}
    \centering
    \begin{subfigure}[a]{0.45\textwidth}
      \includegraphics[width=\textwidth]{\FFpotparamA}
    \end{subfigure}
    \caption{In the left panel we plot $V_{,\chi\chi}(\phi,0)$, the effective mass squared of the $\chi$ field along the $\chi=0$ centre line of the instability. In the right panel we plot transverse sections of the potential at $\phi=\phi_w$, the point of minimum $V_{,\chi\chi}(\phi,0)$ and off-set to pass through zero. The potential parameters used in this figure match those of the example case used in Sec. \ref{sec:results}, \ParamV.}
    \label{fig:potparam}
  \end{figure}
}

% fig:zetaslice
\newcommand{\Fzetaslice}{
  \begin{figure*}
    \centering
    \begin{subfigure}[a]{0.3\textwidth}
      \includegraphics[width=\textwidth]{\FFzetasliceA}
      \caption{}
    \end{subfigure}
    \begin{subfigure}[a]{0.3\textwidth}
      \includegraphics[width=\textwidth]{\FFzetasliceB}
      \caption{}
    \end{subfigure}
    \begin{subfigure}[a]{0.3\textwidth}
      \includegraphics[width=\textwidth]{\FFzetasliceC}
      \caption{}
    \end{subfigure}
    \caption{Comparison of $\zeta$, $\zeta_0$, and the difference $\Delta\zeta=\zeta-\zeta_0$ on a two dimensional slice of a three dimensional lattice. \textcolor{red}{[Mention that this is however many e-folds after $\Delta V$ as $\Delta\zeta$ is approaching its asymptotic value and whatever the potential parameters are for the run used.]} By running two simulations from identical initial conditions, one for the full potential $V$ and the other for the baseline potential $V_0$, we can calculate the system's response to the potential feature $\Delta V$. In the case of $\Delta\zeta$ the response carries information about $\zeta_\mathrm{NG}$. \textcolor{red}{[Point out that $\Delta\zeta$ shows concentration of NG.]}}
    \label{fig:zetaslice}
  \end{figure*}
}

% fig:zetapeak
\newcommand{\Fzetapeak}{
  \begin{figure*}
    \centering
    \begin{subfigure}[a]{0.45\textwidth}
      \includegraphics[width=\textwidth]{\FFzetapeakA}
      \caption{}
    \end{subfigure}
    \begin{subfigure}[a]{0.45\textwidth}
      \includegraphics[width=\textwidth]{\FFzetapeakB}
      \caption{}
    \end{subfigure}
    \caption{Peak counts for maximum and minimum peaks of $\zeta$ (red), $\zeta_0$ (blue), and $\Delta\zeta$ (green) smoothed with a Gaussian filter at $k=aH$ at \textcolor{red}{[mention time slice (same as for Fig. \ref{fig:zetaslice})]}. The solid line are the mean peak counts over an ensemble of 32 simulations, the light and dark shaded regions are 90\% and 50\% contours over the ensemble. The dashed lines are the expected peak counts for Gaussian fields with matching spectra.}
    \label{fig:zetapeak}
  \end{figure*}
}

% fig:spec
\newcommand{\Fspec}{
  \begin{figure*}
    \centering
      \begin{subfigure}[a]{\textwidth}
        \includegraphics[width=\textwidth]{\FFspecB}
        \caption{}
      \end{subfigure}
    \caption[speccap]{Spectra and cross-spectra of the fields and momenta using the full potential $V=V_0+\Delta V$. From left to right are, on the top row $\langle|\tilde{\phi}_k|^2\rangle$, $\mathrm{Re}\langle\tilde{\phi}^*_k\tilde{\Pi}^\phi_k\rangle $, $\langle|\tilde{\Pi}^\phi_k|^2\rangle$, and on the bottom row $\langle|\tilde{\chi}_k|^2\rangle$, $\mathrm{Re}\langle\tilde{\chi}^*_k\tilde{\Pi}^\chi_k\rangle $, $\langle|\tilde{\Pi}^\chi_k|^2\rangle$. The axes and scaling are as in Fig. \ref{fig:specbl} which can be compared as a baseline using only the baseline potential $V_0$. The region where $\Delta V$ is nonzero runs from the left hand edge of each plot to the dashed vertical blue line.
    
    The growth of $\langle|\tilde{\chi}_k|^2\rangle$ and $\langle|\tilde{\Pi}^\chi_k|^2\rangle$ within the instability band of $\Delta V$ is clearly visible. The positive correlatio of $\mathrm{Re}\langle\tilde{\chi}^*_k\tilde{\Pi}^\chi_k\rangle$ within the instability band is due $\tilde{\chi}_k$ and $\tilde{\Pi}^\chi_k$  matching onto the growing mode of the instability. Initially more puzzling may be the sweeping bands of positive correlation of $\mathrm{Re}\langle\tilde{\chi}^*_k\tilde{\Pi}^\chi_k\rangle$ which appear after the end of $\Delta V$. These bands are in fact a linear effect and stem from coherent oscillations of the $\chi$ field, which are set up as modes that were driven toward the growing mode of the instability, once $\Delta V$ ends these mode are again stable and begin to oscillate. These coherent oscillations likewise produce bands of power, shifted in phase, in $\langle|\tilde{\chi}_k|^2\rangle$ and $\langle|\tilde{\Pi}^\chi_k|^2\rangle$.

    The deviation of $\langle|\tilde{\phi}_k|^2\rangle$, $\mathrm{Re}\langle\tilde{\phi}^*_k\tilde{\Pi}^\phi_k\rangle $, and $\langle|\tilde{\Pi}^\phi_k|^2\rangle$ from the baseline case shown in Fig. \ref{fig:specbl} is an effect not predicted by linear theory around a $\langle\chi\rangle=0$ background. We find the origins of this nonlinear effect are more readily exlained in real space rather than Fourier space.}
    \label{fig:spec}
  \end{figure*}
}

% fig:specbl
\newcommand{\Fspecbl}{
  \begin{figure*}
    \centering
      \begin{subfigure}[a]{\textwidth}
        \includegraphics[width=\textwidth]{\FFspecA}
        \caption{}
      \end{subfigure}
    \caption{Spectra and cross-spectra real parts from a baseline run using the potential $V=V_0$. From left to right on the top row are $\langle|\tilde{\phi}_{0k}|^2\rangle$, $\mathrm{Re}\langle\tilde{\phi}^*_{0k}\tilde{\Pi}^\phi_{0k}\rangle $, $\langle|\tilde{\Pi}^\phi_{0k}|^2\rangle$, and from felt to right on the bottom row are $\langle|\tilde{\chi}_{0k}|^2\rangle$, $\mathrm{Re}\langle\tilde{\chi}^*_{0k}\tilde{\Pi}^\chi_{0k}\rangle $, $\langle|\tilde{\Pi}^\chi_{0k}|^2\rangle$. On the horizontal axis time is measured by $\alpha$. The verical axis measures the logarithm of the wavenumber normalized to the Hubble scale $\ln(k/(aH))$. The colour bar measures spectra/cross-spectra (labelled as a component of $C(\phi^A,\Pi^A;k)$ as defined in \textcolor{red}{[equation reference]}). In the normalization $k^3/(2\pi^2)$ is a volume factor, $H^2/(2\pi)^2$ is the Hawking temperature, and $a^{3N_\Pi}H^{N_\Pi}$ is a scaling factor chosen to simplify the evolution of the baseline case with $N_\Pi$ being the number of momentum factors in $C_{ij}$. The diagonal dashed green lines indicate the initialized band of $k$-modes, and the vertical dashed blue lines bound the region where $\Delta V$ is nonzero \textcolor{red}{[remove the dashed blue lines for the baseline case]}\textcolor{red}{[Add a footnote that $\mathrm{symln}(x)\equiv\mathrm{sign}(x)\ln(1+|x|)$].}}
    \label{fig:specbl}
  \end{figure*}
}

% fig:spec
\newcommand{\Fspecold}{
  \begin{figure*}
    \centering
    \begin{subfigure}[a]{0.3\textwidth}
      \includegraphics[width=\textwidth]{\FFspecA}
      \caption{}
    \end{subfigure}
    \begin{subfigure}[a]{0.3\textwidth}
      \includegraphics[width=\textwidth]{\FFspecB}
      \caption{}
    \end{subfigure}
    \begin{subfigure}[a]{0.3\textwidth}
      \includegraphics[width=\textwidth]{\FFspecC}
      \caption{}
    \end{subfigure}
    \begin{subfigure}[b]{0.3\textwidth}
      \includegraphics[width=\textwidth]{\FFspecD}
      \caption{}
    \end{subfigure}
    \begin{subfigure}[b]{0.3\textwidth}
      \includegraphics[width=\textwidth]{\FFspecE}
      \caption{}
    \end{subfigure}
    \begin{subfigure}[b]{0.3\textwidth}
      \includegraphics[width=\textwidth]{\FFspecF}
      \caption{}
    \end{subfigure}
    \caption{Spectra and real part of cross-spectra \textcolor{red}{[some of the cross-spectra]}. The vertical dashed blue lines indicate the region where $\Delta V$ is nonzero. The diagonal dashed green lines indicate the initialized band of $k$-modes. Contours are logrithmically spaced. Negative regions are hashed and plotted in absolute value. \textcolor{red}{[Mention normalization of $H^2/(2\pi^2)$, $a^3H^3/(2\pi^2)$, $2a^6H^4/(2\pi^2)$.]}}
    \label{fig:specold}
  \end{figure*}
}

% fig:specdet
\newcommand{\Fspecdet}{
  \begin{figure}
    \centering
    \begin{subfigure}[a]{0.45\textwidth}
      \includegraphics[width=\textwidth]{\FFspecdet}
    \end{subfigure}
    \caption{\textcolor{red}{[Place holder for the $\mathrm{det}C(\phi^A,\Pi^A;k)$ figure. This is not actually $\mathrm{det}C(\phi^A,\Pi^A;k)$.]}}
    \label{fig:specdet}
  \end{figure}
}

% fig:traj
\newcommand{\Ftraj}{
  \begin{figure*}
    \centering
    \begin{subfigure}[a]{0.225\textwidth}
      \includegraphics[width=\textwidth]{\FFtrajA}
      \caption{}
    \end{subfigure}
    \begin{subfigure}[a]{0.225\textwidth}
      \includegraphics[width=\textwidth]{\FFtrajB}
      \caption{}
    \end{subfigure}
    \begin{subfigure}[a]{0.225\textwidth}
      \includegraphics[width=\textwidth]{\FFtrajC}
      \caption{}
    \end{subfigure}
    \begin{subfigure}[a]{0.225\textwidth}
      \includegraphics[width=\textwidth]{\FFtrajD}
      \caption{}
    \end{subfigure}
    \begin{subfigure}[b]{0.225\textwidth}
      \includegraphics[width=\textwidth]{\FFtrajE}
      \caption{}
    \end{subfigure}
    \begin{subfigure}[b]{0.225\textwidth}
      \includegraphics[width=\textwidth]{\FFtrajF}
      \caption{}
    \end{subfigure}
    \begin{subfigure}[b]{0.225\textwidth}
      \includegraphics[width=\textwidth]{\FFtrajG}
      \caption{}
    \end{subfigure}
    \begin{subfigure}[b]{0.225\textwidth}
      \includegraphics[width=\textwidth]{\FFtrajH}
      \caption{}
    \end{subfigure}
    \caption{Density contours of trajectories of $\phi$, $\frac{\dd\phi}{\dd\alpha}$, $\chi$, and $\frac{\dd\chi}{\dd\alpha}$ marginalized for position. Grey contours are level sets of the distribution over the whole lattice. Red contours for points where $\Delta\zeta_\mathrm{end}$ gives $\nu>5$. Blue dashed lines bound the region where $\Delta V$ is nonzero. Comparing the red and grey contours we see the trajectories which lead to large concentrations of $\Delta\zeta$ are exactly those which udergo the largest excursions during the instability and for which $\frac{\dd\phi}{\dd\alpha}$ is most strongly deviated with the onset of nonlinearity.}
    \label{fig:traj}
  \end{figure*}
}

% fig:slice
\newcommand{\Fslice}{
  \begin{figure*}
    \centering
    \begin{subfigure}[a]{0.225\textwidth}
      \includegraphics[width=\textwidth]{\FFsliceA}
      \caption{}
    \end{subfigure}
    \begin{subfigure}[a]{0.225\textwidth}
      \includegraphics[width=\textwidth]{\FFsliceB}
      \caption{}
    \end{subfigure}
    \begin{subfigure}[a]{0.225\textwidth}
      \includegraphics[width=\textwidth]{\FFsliceC}
      \caption{}
    \end{subfigure}
    \begin{subfigure}[a]{0.225\textwidth}
      \includegraphics[width=\textwidth]{\FFsliceD}
      \caption{}
    \end{subfigure}
    \caption{Two dimensional slices throught a three dimensional lattice showing $\phi$, $\frac{\dd\phi}{\dd\alpha}$, $\chi$, and $\frac{\dd\chi}{\dd\alpha}$ \textcolor{red}{[mention time slice is when $\Delta V \to 0$]}. The instabilty induced by $\Delta V$ has caused a large amplification of $\chi$ and $\frac{\dd\chi}{\dd\alpha}$ fluctuations on scales within the instability band. In the $\phi$ and $\frac{\dd\phi}{\dd\alpha}$ slices nonlinear interactions have formed a numbre of prominent peaks which, refering to Fig. \ref{fig:zetaslice}, are located where $\Delta\zeta$ will be concentrated.}
    \label{fig:slice}
  \end{figure*}
}

% fig:phasespace
\newcommand{\Fphasespace}{
  \begin{figure*}
    \centering
    \begin{subfigure}[a]{0.225\textwidth}
      \includegraphics[width=\textwidth]{\FFphasespaceA}
      \caption{}
    \end{subfigure}
    \begin{subfigure}[a]{0.225\textwidth}
      \includegraphics[width=\textwidth]{\FFphasespaceB}
      \caption{}
    \end{subfigure}
    \begin{subfigure}[a]{0.225\textwidth}
      \includegraphics[width=\textwidth]{\FFphasespaceC}
      \caption{}
    \end{subfigure}
    \begin{subfigure}[a]{0.225\textwidth}
      \includegraphics[width=\textwidth]{\FFphasespaceD}
      \caption{}
    \end{subfigure}
    \begin{subfigure}[b]{0.225\textwidth}
      \includegraphics[width=\textwidth]{\FFphasespaceE}
      \caption{}
    \end{subfigure}
    \begin{subfigure}[b]{0.225\textwidth}
      \includegraphics[width=\textwidth]{\FFphasespaceF}
      \caption{}
    \end{subfigure}
    \begin{subfigure}[b]{0.225\textwidth}
      \includegraphics[width=\textwidth]{\FFphasespaceG}
      \caption{}
    \end{subfigure}
    \begin{subfigure}[b]{0.225\textwidth}
      \includegraphics[width=\textwidth]{\FFphasespaceH}
      \caption{}
    \end{subfigure}
    \caption{Density contours for projections of the position marginalized phase space distributions taken on a \textcolor{red}{[mention time slice at the end of $\Delta V$]} time slice. \textcolor{red}{[Mention this is for the same run as shown in the other plots.]} Grey contours are level sets of the distribution over the whole lattice. Red contours for only those points where $\Delta\zeta_\mathrm{end}$ gives $\nu>5$ \textcolor{red}{[I don't think I've yet defined $\Delta\zeta_\mathrm{end}$]}.}
    \label{fig:phasespace}
  \end{figure*}
}

% fig:deltaps
\newcommand{\Fdeltaps}{
  \begin{figure}
    \centering
      \begin{subfigure}[a]{0.45\textwidth}
        \includegraphics[width=\textwidth]{\FFdeltapsA}
      \end{subfigure}
      \caption{\textcolor{red}{[Here $\delta\phi$ has been replaced by $\Delta\phi$ (defined analogously to $\Delta\zeta$ as the point by point difference of $\phi$ as calculated using the full potential $V=V_0 + \Delta V$ and the baseline potential $V_0$ alone). Compare to Fig. \ref{fig:phasespace}. Descirbe in terms of an equation of state.]}}
    \label{fig:deltaps}
  \end{figure}
}

% fig:deltaphizetacorr
\newcommand{\Fdeltaphizetacorr}{
  \begin{figure}
    \centering
    \begin{subfigure}[a]{0.45\textwidth}
      \includegraphics[width=\textwidth]{\FFdeltaphizetacorr}
    \end{subfigure}
    \caption{\textcolor{red}{[Place holder caption for the unequal time correlation between $\Delta\zeta_f$ and $\Delta\phi_{\Delta V \mathrm{off}}$.]}}
    \label{fig:deltaphizetacorr} 
  \end{figure}
}

% fig:zetatraj
\newcommand{\Fzetatrajold}{
  \begin{figure*}
      \centering
    %\begin{subfigure}[a]{0.3\textwidth}
    %  \includegraphics[width=\textwidth]{\FFzetatrajA}
    %  \caption{}
    %\end{subfigure}
    %\begin{subfigure}[a]{0.3\textwidth}
    %  \includegraphics[width=\textwidth]{\FFzetatrajB}
    %  \caption{}
    %\end{subfigure}
    %\begin{subfigure}[a]{0.3\textwidth}
    %  \includegraphics[width=\textwidth]{\FFzetatrajC}
    %  \caption{}
    %\end{subfigure}
    \begin{subfigure}[a]{0.3\textwidth}
      \includegraphics[width=\textwidth]{\FFzetatrajD}
      \caption{}
    \end{subfigure}
    \begin{subfigure}[a]{0.3\textwidth}
      \includegraphics[width=\textwidth]{\FFzetatrajE}
      \caption{}
    \end{subfigure}
    \begin{subfigure}[a]{0.3\textwidth}
      \includegraphics[width=\textwidth]{\FFzetatrajF}
      \caption{}
    \end{subfigure}
    \caption{Density contours for trajectories of $\Delta\zeta$, $\frac{\dd\Delta\zeta_\phi}{\dd\alpha}$, and $\frac{\dd\Delta\zeta_\chi}{\dd\alpha}$. Contour colours as are in Fig. \ref{fig:traj}. Dashed blue lines bound the region where $\Delta V$ is nonzero. Two distinct bursts of $\Delta\zeta$ production are involved in producing $\Delta\zeta$ concentrations. The first burst of production comes through the $\frac{\dd\Delta\zeta_\chi}{\dd\alpha}$ channel \textcolor{red}{[some words about this]}. The second burst of production comes through the $\frac{\dd\Delta\zeta_\phi}{\dd\alpha}$ channel and is of opposite sign to the first.}
    \label{fig:zetatrajold}
  \end{figure*}
}

% fig:dzetatraj
\newcommand{\Fzetatraj}{
  \begin{figure*}
    \centering
    \begin{subfigure}[a]{0.225\textwidth}
      \includegraphics[width=\textwidth]{\FFzetatrajA}
      \caption{}
    \end{subfigure}
    \begin{subfigure}[a]{0.225\textwidth}
      \includegraphics[width=\textwidth]{\FFzetatrajB}
      \caption{}
    \end{subfigure}
    \begin{subfigure}[a]{0.225\textwidth}
      \includegraphics[width=\textwidth]{\FFzetatrajC}
      \caption{}
    \end{subfigure}
    \begin{subfigure}[a]{0.225\textwidth}
      \includegraphics[width=\textwidth]{\FFzetatrajD}
      \caption{}
    \end{subfigure}
    \caption{Density contours for trajectories of $\Delta\zeta$, $\frac{\dd\Delta\zeta}{\dd\alpha}$, and two of the sourcing channels \eqref{eq:zeta source lap}, $\frac{\dd\Delta\zeta_{\dot{\phi}\nabla^2\phi}}{\dd\alpha}$ and $\frac{\dd\Delta\zeta_{\dot{\chi}\nabla^2\chi}}{\dd\alpha}$ (the sourcing channels \eqref{eq:zeta source gdg} give a subdominatnt contribution to the peaks of $\Delta\zeta$). Contour colours as are in Fig. \ref{fig:traj}. Dashed blue lines bound the region where $\Delta V$ is nonzero. Two distinct bursts of $\Delta\zeta$ production are involved in producing $\Delta\zeta$ concentrations. The first burst is negative and sourced through the $\frac{\dd\Delta\zeta_{\dot{\chi}\nabla^2\chi}}{\dd\alpha}$ channel. The second burst of production comes through the $\frac{\dd\Delta\zeta_{\dot{\phi}\nabla^2\phi}}{\dd\alpha}$ channel and is of opposite sign to the first. \textcolor{red}{[Link this caption to the real space picture of Fig. \ref{fig:zetaslicemulti}.]}}
    \label{fig:zetatraj}
  \end{figure*}
}

% fig:zetaslicemulti
\newcommand{\Fzetaslicemulti}{
  \begin{figure*}
    \centering
    \includegraphics[width=0.9\textwidth]{\FFzetaslicemulti}
    \caption{A two dimensional slice through a three dimensional lattice showing $\Delta\zeta$ at \textcolor{red}{[time slices]}. The two stage production process of $\Delta\zeta$ shown in Fig. \ref{fig:zetatraj} can be seen here to first generate a set of negative $\Delta\zeta$ peaks in the first stage of production which change sign during the second stage of production to become positive peaks in the asymptotic $\Delta\zeta$.}
    \label{fig:zetaslicemulti}
  \end{figure*}
}

% fig:zetapstraj
\newcommand{\Fzetapstraj}{
  \begin{figure*}
    \centering
    \begin{subfigure}[a]{0.45\textwidth}
      \includegraphics[width=\textwidth]{\FFzetapstrajA}
      %\caption{}
    \end{subfigure}
    \begin{subfigure}[a]{0.45\textwidth}
      \includegraphics[width=\textwidth]{\FFzetapstrajB}
      %\caption{}
    \end{subfigure}
    \caption[zetapstrajcap]{Trajectories through the $(\Delta\zeta,\frac{\dd\Delta\zeta}{\dd\alpha})$ psuedo phase space for $\alpha \in [0.43,1.99]$ or $\langle\phi\rangle/\mpl \in [8.52,8.15]$, which corresponds to trajectories starting from roughly the midpoint of the instability and running long enough for $\Delta\zeta$ to approach its asymptotic value. Both the time weighted density of trajectories and a sample of individual trajectories are plotted.
    
    % Talk about the regular vs extreme trajectories
    In the left panel the grey contours are plotted for all trajectories, while in the right panel The contours for all trajectories have been overlayed with a set of red contours plotted for only the extreme trajectories. Note also the relative scaling of the axes between the panels with the left panel zoomed in by a factor of four compared to the right panel.
    % what is plotted by the contours
    In both panels the contours show the time weighted psuedo phase space density define as $(\alpha_2 - \alpha_1)^{-1}\mathcal{N}\int_{\alpha_1}^{\alpha_2}P(\Delta\zeta,\frac{\dd\Delta\zeta}{\dd\alpha})\dd\alpha$, which we calculate using a two-dimensional histogram to estimate the probablity density and a partial sum to estimate the integral, the normalization factor $\mathcal{N}$ is chosen to scale the maximum of time weighted density to unity.
    \textcolor{red}{Talk about the low speed vs high density degeneracy in the contour plot and how the arrows on the sampled trajectories help to disentangle the two effects.}
    
    % sampled trajectories
    Plotted over these contours is a sample of trajectories plotted as black curves using various line styles to allow the individual trajectories to be more clearly distinguished. In the left panel these trajectories are sampled from all trajectories on the lattice, while in the right panel they are sampled from the extreme trajectories. The left panel also shows a single extreme trajectory plotted in red.
    % Talk about the imergence of simplicity in the extreme trajectories
    Comparison between sampled trajectories in the left and right panels reveals a remarkable simplification in the complexity of trajectories when considering only the extreme trajectories as opposed to all trajectories. % elaborate on this point in the main text

    % the black arrow markers
    The passage of time along each of the sampled trajetories is marked by four black arrows pointing in the direction of time which, together with the endpoints, divide the trajectories into five segments of equal length as measured by change in $\alpha(\tau)$. Sections of a trajectory which are traversed more quickly have a greater spacing between the black arrows.
    % approach to an asymptotic $\Delta\zeta$
    The black arrows pile up as a trajectory approaches its asymptotic value, as can be seen in the right panel along the positve $\frac{\dd\Delta\zeta}{\dd\alpha}=0$ axis as the extreme trajectories approach there respective asymptotic values of $\Delta\zeta$.
    
    % the blue arrow marker
    The blue arrows are placed at $\langle\phi\rangle = \phi_p - \phi_w$ to mark the mean point where $\Delta V$ turns off and the trajectories are returned to the baseline quadratic potential $V_0$. The portion of these trajectories occuring after the blue arrow is the production of $\Delta\zeta$ during the relaxation phase.

    % Talk about plotting artifacts
    The feathery edge pattern visible on the upper edge of the contours is an artifact caused by binning the discrete time outputs from the lattice simulation. For a given discrete time sampling of the lattice there is a trade-off between the coarseness of the binning and the prominance of this feathery edge artifact. We have adjusted the phase space bin size to balance these two effects in a way which makes the figure clear.    
    }
    \label{fig:zetapstraj}
  \end{figure*}
}
% fig:alphatraj
\newcommand{\Falphatraj}{
  \begin{figure}
    \centering
    \begin{subfigure}[a]{0.45\textwidth}
      \includegraphics[width=\textwidth]{\FFalphatraj}
    \end{subfigure}
    \caption[alphatrajcap]{Contour plot of the distribution of $\delta\alpha(x,\tau)$ trajectories as calculated by integration of the local Hubble as in \eqref{eq:alphaloc}.
    % reference to similar figure for explanation of how to read figure and exactly what is plotted
    % reference to sampled trajectories
    \textcolor{red}{Reference an earlier trajectory figure to explain how to read this plot.}

    % talk about initial condition being set to zero increases variance of initial bundle by ~ a factor of 2
    The initial transient in the distribution is due to setting the initial condition $\alpha(x,\tau_0)$ to zero over the lattice rather than matching onto linear theory. After phases have mixed, this is equivalent to adding a random field to $\delta\alpha$ with variance accounting for approximately half the variance of the initial bundle of trajectories after the transient behaviour has settled.
    % reference to $\ln(\rho)$ figure and compare size of effect to justify dropping inhomogeneity in $\delta\alpha$ for the calculation of $\zeta$.
    This is acceptable as we do not need the precise value of the individual trajectories of $\delta\alpha$, but only an estimate for their distribution to check the validity of neglecting the inhomgeneity of $\alpha$ in our calculation of $\zeta$. Comparision to $\ln(\rho/\langle\rho\rangle)$ shown in Fig. \ref{fig:lnrhotraj}, the distribution of which is approximately two orders of magnitude wider that that of $\delta\alpha$, can be used to verify the validity of this assumption as described in \textcolor{red}{reference to $\zeta$ setup section}.
    }
    \label{fig:alphatraj}
  \end{figure}
}

\newcommand{\Flnrhotraj}{
  \begin{figure}
    \centering
    \begin{subfigure}[a]{0.45\textwidth}
      \includegraphics[width=\textwidth]{\FFrhotraj}
    \end{subfigure}
    \caption[lnrhotrajcap]{Contour plot of the distribution of $\ln(\rho/\langle\rho\rangle)$ trajectories. \textcolor{red}{reference earlier trajectory figure to explain how this figure is read.}
    % talk about the large variance at the top being due to the subhorizon fluctuations (the window of k modes on the lattice redshifts during the expansion)
    The initial narrowing of the width of this distribution with time is due to the lattice covering a fixed range of comoving scales rather than physical scales. The redshifting of these comoving scales as the simulation proceeds, means the physical scales to which these fixed comoving scales correspond will redshift during the simulation and $\ln(\rho/\langle\rho\rangle)$ on the lattice will be measured on progressively fewer modes which are subhorizon modes and progressively more superhorizon modes.
    
    % Talk about the similarity to one term of $\delta\zeta$ when scaled up by a factor
    Taken in conjunction with Fig. \ref{fig:alphatraj}, this figure can be used to verify the validity of neglected the $\alpha$ in our calculation of $\zeta$ by having made the FRW anzats.
    }
    \label{fig:lnrhotraj}
  \end{figure}
}

% fig:zetaslicecomp
\newcommand{\Fzetaslicecomp}{
  \begin{figure*}
    \centering
    \begin{subfigure}[a]{0.3\textwidth}
      \includegraphics[width=\textwidth]{\FFzetaslicecompA}
      \caption{}
    \end{subfigure}
    \begin{subfigure}[a]{0.3\textwidth}
      \includegraphics[width=\textwidth]{\FFzetaslicecompB}
      \caption{}
    \end{subfigure}
    \begin{subfigure}[a]{0.3\textwidth}
      \includegraphics[width=\textwidth]{\FFzetaslicecompC}
      \caption{}
    \end{subfigure}
    \begin{subfigure}[b]{0.3\textwidth}
      \includegraphics[width=\textwidth]{\FFzetaslicecompD}
      \caption{}
    \end{subfigure}
    \begin{subfigure}[b]{0.3\textwidth}
      \includegraphics[width=\textwidth]{\FFzetaslicecompE}
      \caption{}
    \end{subfigure}
    \begin{subfigure}[b]{0.3\textwidth}
      \includegraphics[width=\textwidth]{\FFzetaslicecompF}
      \caption{}
    \end{subfigure}
    \caption[zetaslicecompcap]{Two dimensional slices through a set of lattices run with varying the parameters of $\Delta V$ showing the onset of spatially intermittent peaks during the incipient phase transtion and continuing to the eventual breakdown of the effect with the completion of the phase transition and production of domain walls.

    From left to right and top to bottom the simulations are run with increasing values of $\lambda_\chi$ and $\phi_w$, and decreasing values of $v$. This sequence of parameters, as shown in Fig. \ref{fig:potparamcomp}, has been tailored to catch the two extremes of the weak instability in (a), through to the formation of domain walls in (f).
    \textcolor{red}{[I need to figure out why (a) does not appear as positive peaks and label this reason as the opposite extreme of producing domain walls]}

    The fluctuations in these simulations have been matched at the point that $\Delta V$ turns on so that differences between the slices is attributable as a response to varying the parameters of $\Delta V$.

    We see that those points which spawn the intermitent peaks during the incipient phase go on to seed the formation of domains if the phase transition is allowed to complete. And that there is a regime, covering (b), (c), and (d), where $\Delta\zeta$ remains similar with the variartion of $\Delta V$ up to a scaling of its magnitude. It is to this intermediate regime that the example disscussed in detail in Sec. \ref{sec:results} belongs.

    \textcolor{red}{[In the main text, talk about how by matching the fluctuations at the point $\Delta V$ turns on we can extend the idea of $\Delta\zeta$ to a response function to the parameters of $\Delta V$.]}
    \textcolor{red}{[This caption is in need of an edit for improved flow.]}
    }
    \label{fig:zetaslicecomp}
  \end{figure*}
}

% fig:zetaslicepotform
\newcommand{\Fzetaslicepotform}{
  \begin{figure}
    \centering
    %\includegraphics[width=0.9\textwidth]{}
    \caption{
    \textcolor{red}{[Place holder for a figure of $\Delta\zeta$ slices for instablities using different forms of $\Delta V$ for a few values of instability strengths. Comment on similarities and differences.]}}
    \label{fig:zetaslicepotform}
  \end{figure}
}

% fig:deltapspotform
\newcommand{\Fdeltapspotform}{
  \begin{figure}
    \centering
    %\includegraphics[width=0.9\textwidth]{}
    \caption{\textcolor{red}{[Place holder for a figure similar to Fig. \ref{fig:deltaps} for the alternate potentials used in Fig. \ref{fig:zetaslicepotform}.]}}
    \label{fig:deltapspotform}
  \end{figure}
}

% fig:potparamcomp
\newcommand{\Fpotparamcomp}{
  \begin{figure*}
    \centering
    \begin{subfigure}[a]{0.45\textwidth}
      \includegraphics[width=\textwidth]{\FFpotparamcompA}
    \end{subfigure}
    \begin{subfigure}[a]{0.45\textwidth}
      \includegraphics[width=\textwidth]{\FFpotparamcompB}
    \end{subfigure}
    % include graphics
    \caption[potparamcompcap]{% talk about how this sequence is designed to show the onset and breakdown in a single sequence.
    % Mention that the labels correspond to the runs in Fig. \ref{fig:zetaslicecomp}
    In the left panel we plot $V_{,\chi\chi}(\phi,0)$, the effective mass squared of the $\chi$ field along the $\chi=0$ centre line of the instability. And in the right panel we plot transverse sections of the potential at $\phi=\phi_w$, the point of minimum $V_{,\chi\chi}(\phi,0)$ and off-set to pass through zero. Each curve corresponds to one choice of parameters used in the runs of Fig. \ref{fig:zetaslicecomp} as labeled in the legend.% discuss the shape of the profile and how $\chi_w$, $\lambda_\chi$, and $\phi_w$ relate.
    % Mention that effective mass and $\phi_w$ are both increased logrithmically from (a) to (f)

    % Mention that $\chi_w$ is decreased linearly to force the phase transition to complete which a smaller growth of $\chi$ fluctuations.
    }
    \label{fig:potparamcomp}
  \end{figure*}
}

% fig:zetapkcomp
\newcommand{\Fzetapkcomp}{
  \begin{figure*}
    \centering
    \begin{subfigure}[a]{0.45\textwidth}
      \includegraphics[width=\textwidth]{\FFzetapkcompA}
    \end{subfigure}
    \begin{subfigure}[a]{0.45\textwidth}
      \includegraphics[width=\textwidth]{\FFzetapkcompB}
    \end{subfigure}
    \begin{subfigure}[b]{0.45\textwidth}
      \includegraphics[width=\textwidth]{\FFzetapkcompC}
    \end{subfigure}
    \begin{subfigure}[b]{0.45\textwidth}
      \includegraphics[width=\textwidth]{\FFzetapkcompD}
    \end{subfigure}
    \caption{\textcolor{red}{[Place holder caption for peak counts of the runs shown in Fig. \ref{fig:zetaslicecomp}. Left to right, top to bottom are: $\zeta$ positive peaks, $\zeta$ negative peaks, $\Delta\zeta$ postitive peaks, $\Delta\zeta$ negative peaks. All are normalized by the BBKS expected number of peaks.]}}
    \label{fig:zetapkcomp}
  \end{figure*}
}

% fig:zetapdfcomp
\newcommand{\Fzetapdfcomp}{
  \begin{figure*}
    \centering
    \begin{subfigure}[a]{0.45\textwidth}
      \includegraphics[width=\textwidth]{\FFzetapdfcompA}
    \end{subfigure}
    \begin{subfigure}[a]{0.45\textwidth}
      \includegraphics[width=\textwidth]{\FFzetapdfcompB}
    \end{subfigure}
    \caption{\textcolor{red}{[Place holder caption for PDFs of runs shown in Fig. \ref{fig:zetaslicecomp}. On the left for $\zeta$ and on the right for $\Delta\zeta$.]}}
    \label{fig:zetapdfcomp}
  \end{figure*}
}


% fig:uresponsemag
%\newcommand{/Furesonsemag}{
  %\begin{figure}
    %\centering
    % include graphics
    %\caption{\textcolor{red}{[This will be a figure showing how changing $u$ changes the height of $\zeta$ peaks. Do something like what is the 0.01 highest fraction of $\Delta\zeta$ as a function of $u$.]}}
    %\label{fig:uresponsemag}
  %\end{figure}
%}

% fig:uresponseratio
%\newcommand{/Furesponseratio}{
  %\begin{figure}
    %\centering
    % include graphics
    %\caption{\textcolor{red}{[This will be a figure comparing the ratio between different $\Delta\zeta_{part}$ as a function of $u$.]}}
    %\label{fig:uresponseratio}
  %\end{figure}
%}

% Figure of the field and momenta spectra and cross-spectra
%\newcommand{\Fspec}{
%  \begin{figure}
%    \centering
%    \begin{subfigure}[a]{0.3\textwidth}
%        \includegraphics[width=\textwidth]{\FFspecA}
%        \caption{$\frac{k^3}{2\pi^2}\langle |\phi_k|^2 \rangle$}
%        \label{fig:specA}
%    \end{subfigure}
%    \begin{subfigure}[a]{0.3\textwidth}
%        \includegraphics[width=\textwidth]{\FFspecB}
%        \caption{$\frac{k^3}{2\pi^2}\langle a^{-6}|\Pi^\phi_k|^2 \rangle$}
%        \label{fig:specB}
%    \end{subfigure}
%    \begin{subfigure}[a]{0.3\textwidth}
%        \includegraphics[width=\textwidth]{\FFspecC}
%        \caption{$\frac{k^3}{2\pi^2}\mathrm{Re}\langle a^{-3}\phi_k^*\Pi^\phi_k \rangle$}
%        \label{fig:specC}
%    \end{subfigure}
%    \begin{subfigure}[b]{0.3\textwidth}
%        \includegraphics[width=\textwidth]{\FFspecD}
%        \caption{$\frac{k^3}{2\pi^2}\langle |\chi_k|^2 \rangle$}
%        \label{fig:specD}
%    \end{subfigure}
%    \begin{subfigure}[b]{0.3\textwidth}
%        \includegraphics[width=\textwidth]{\FFspecE}
%        \caption{$\frac{k^3}{2\pi^2}\langle a^{-6}|\Pi^\chi_k|^2 \rangle$}
%        \label{fig:specE}
%    \end{subfigure}
%    \begin{subfigure}[b]{0.3\textwidth}
%        \includegraphics[width=\textwidth]{\FFspecF}
%        \caption{$\frac{k^3}{2\pi^2}\mathrm{Re}\langle a^{-3}\chi_k^*\Pi^\chi_k \rangle$}
%        \label{fig:specF}
%    \end{subfigure}
%    \caption{Contour plots of the power in some of the auto/cross-spectra, contours are spaced logrithmically with dotted lines indicating negative values. The horizontal axis $\alpha$ measures time and on the vertical axis $k/(aH)$ measures the wavenumber relative to the horizon. The diagonal dashed blue lines indicate the band in which modes are initially populated on the lattice. The vertical dashed green lines indicate the times when $\Delta V$ is activate/deactivated and its midpoint.}% \\ In the second row panels (d) and (e) show, with the activation of $\Delta V$, the power of $\chi$ and $\Pi^\chi$ rise across the linear instability band and panel (f) shows they become positively correlated. In the first row the panels (a) and (b) show the power of $\phi$ and $\Pi^\phi$ initially match the the behaviour expected from linear theory, but some time after the midpoint of $\Delta V$ when $\alpha \approx 0.8$ there is a sudden growth in power and panel (c) shows a change in sign of the cross-spectrum. This effect is not predicted from linear theory taking $\langle \phi \rangle_\lat$ and $\langle \chi \rangle_\lat$ as the background fields}
%    \label{fig:spec}
%  \end{figure}
%}

% Figure of the field and momenta spectra determinant
%\newcommand{\Fspecdet}{
%  \begin{figure}
%    \centering
%    \begin{subfigure}[a]{0.45\textwidth}
%      \includegraphics[width=\textwidth]{\FFspecdetA}
%      \caption{}
%    \end{subfigure}
%    \begin{subfigure}[a]{0.45\textwidth}
%      \includegraphics[width=\textwidth]{\FFspecdetB}
%      \caption{}
%    \end{subfigure}
%    \caption{Determinant of of the fluctuation correlation matrix $C(\phi^A,\Pi^A)$ as function of time (clocked by $\alpha$). This quantity is conserved whenever the systems dynamics are linear. \\ Place holder figure. It is probably more clear to use a contour plot here instead of a plethera of sampled $k$.}
%    \label{fig:specdet}
%  \end{figure}
%}

% Figure of a 2d slice of the fields at times before, during, and after nonlinearity as identified in \ref{fig:spec det} showing the foramation and backreaction of the transverese condensate.
%\newcommand{\Fslice}{
%  \begin{figure}
%    \centering
%    \includegraphics[width=\textwidth]{\FFslice}
    %\caption{Real space field fluctuations $\delta\phi$, $\chi$ and their time derivatives $\delta\dot{\phi}$, $\dot{\chi}$ on a two dimensional slice of a three dimensional lattice, shown at the times slices identidied in figure \ref{fig:specdet}.}
%    \caption{Cross-sections of the lattice showing field fluctuations $\delta\phi$ and $\chi$, and the time derivatives $\frac{\dd\delta\phi}{\dd\alpha}$ and $\frac{\dd\chi}{\dd\alpha}$ (related to the momenta by $\frac{\dd\phi^A}{\dd\alpha} = a^{-3}H^{-1}\Pi^A$). The same time slices are shown as in figure \ref{fig:phasespace}, with the addition of two additional intermediate slices. On each panel a white scale bar indicates the horizon scale $(aH)^{-1}$.}
%    \label{fig:slice}
%  \end{figure}
%}

% Figure of phase space projection of the fields and momenta at times before, during, and after nonlineraity as identified in \ref{fig:spec det} showing the correlations that arise.
%\newcommand{\Fphasespaceold}{
%  \begin{figure}
%    \centering
%    \begin{subfigure}[a]{0.3\textwidth}
%      \includegraphics[width=\textwidth]{\FFphasespaceAold}
%      \caption{}
%    \end{subfigure}
%    \begin{subfigure}[a]{0.3\textwidth}
%      \includegraphics[width=\textwidth]{\FFphasespaceBold}
%      \caption{}
%    \end{subfigure}
%    \begin{subfigure}[a]{0.3\textwidth}
%      \includegraphics[width=\textwidth]{\FFphasespaceCold}
%      \caption{}
%    \end{subfigure}
%    \begin{subfigure}[b]{0.3\textwidth}
%      \includegraphics[width=\textwidth]{\FFphasespaceDold}
%      \caption{}
%    \end{subfigure}
%    \begin{subfigure}[b]{0.3\textwidth}
%      \includegraphics[width=\textwidth]{\FFphasespaceEold}
%      \caption{}
%    \end{subfigure}
%    \begin{subfigure}[b]{0.3\textwidth}
%      \includegraphics[width=\textwidth]{\FFphasespaceFold}
%      \caption{}
%    \end{subfigure}
%    \caption{Phase space projections of the fields and their momenta shown athe the time slices identified in figure \ref{fig:specdet}.}
%    \label{fig:phasespace}
%  \end{figure}
%}

%\newcommand{\Fphasespace}{
%  \begin{figure}
%    \centering
%    \begin{subfigure}[a]{0.45\textwidth}
%      \includegraphics[width=\textwidth]{\FFphasespaceA}
%      \caption{$P(\phi,\Pi^\phi)$}
%    \end{subfigure}
%    \begin{subfigure}[a]{0.45\textwidth}
%      \includegraphics[width=\textwidth]{\FFphasespaceB}
%      \caption{$P(\chi,\Pi^\chi)$}
%    \end{subfigure}
%    \begin{subfigure}[b]{0.45\textwidth}
%      \includegraphics[width=\textwidth]{\FFphasespaceC}
%      \caption{$P(\phi,\chi)$}
%    \end{subfigure}
%    \begin{subfigure}[b]{0.45\textwidth}
%      \includegraphics[width=\textwidth]{\FFphasespaceD}
%      \caption{$P(\Pi^\phi,\Pi^\chi)$}
%    \end{subfigure}
    %\begin{subfigure}[c]{0.45\textwidth}
    %  \includegraphics[width=\textwidth]{\FFphasespaceE}
    %  \caption{}
    %\end{subfigure}
    %\begin{subfigure}[c]{0.45\textwidth}
    %  \includegraphics[width=\textwidth]{\FFphasespaceF}
    %  \caption{}
    %\end{subfigure}
%    \caption{Distributions of phase space projections for various combinations fields and momenta. Each set of three panels shows the distribution at three characteristic times: before the onset of nonlinearity, when nonlinearity fully develops, and during the relaxation of the system after $\Delta V$ has turned off. For each panel the axes have been scaled relative to the amplitude of fluctuations. The probablity density spaceing between contours is equivalent to a $0.5\sigma$ step for a Gaussian PDF matching the mode of the measured histogram.}
%    \label{fig:phasespace}
%  \end{figure}
%}

% Figure of real space slices comparing $\zeta_0$, $\zeta$, and $\Delta\zeta$.
%\newcommand{\Fzetadiff}{
%  \begin{figure}
%    \centering
%    \begin{subfigure}[a]{0.3\textwidth}
%      \includegraphics[width=\textwidth]{\FFzetadiffA}
%      \caption{$\zeta_0$}
%      \label{fig:zetadiffA}
%    \end{subfigure}
%    \begin{subfigure}[a]{0.3\textwidth}
%      \includegraphics[width=\textwidth]{\FFzetadiffB}
%      \caption{$\zeta$}
%      \label{fig:zetadiffB}
%    \end{subfigure}
%    \begin{subfigure}[a]{0.3\textwidth}
%      \includegraphics[width=\textwidth]{\FFzetadiffC}
%      \caption{$\Delta\zeta$}
%      \label{fig:zetadiffC}
%    \end{subfigure}
%    \caption{[Place holder caption. This is the figure illustrating how $\Delta\zeta$ is constructed.]}
%    \label{fig:zetadiff}
%  \end{figure}
%}

% Figure of real space slices of $\Delta\zeta$ at specified times.
%\newcommand{\Fzetaslice}{
%  \begin{figure}
%    \centering
%    \includegraphics[width=\textwidth]{\FFzetaslice}
%    \caption{Several time slices of a cross-section of $\Delta\zeta$ through the lattice. The time slices and cross-section are matched to those of figure \ref{fig:slice}. The formation of peaks of $\Delta\zeta$ is apparent. Comparing with figure \ref{fig:slice} shows these peaks are formed at locations where $\phi$ and $\chi$ interact nonlinearly. The white scale bar in each panel shows the horizon scale $(aH)^{-1}$.}
%    \label{fig:zetaslice}
%  \end{figure}
%}

% Figure of relative abundance of peaks in $\Delta\zeta$ with smoothin at horizon scale at end of sim.
%\newcommand{\Fzetapeak}{
%  \begin{figure}
%    \centering
%    \includegraphics[width=\textwidth]{\FFzetapeak}
%    \caption{[Place-holder figure. I'd like to make a figure which plots the positive and negative peak counts as a fraction of the expected peak count for a Gaussian field matching the measured power spectrum of $\Delta\zeta$ and do a number of runs which sample over random seed to get some uncertainty contours.] Relative abunance of positive (blue) and negative (red) peaks in $\Delta\zeta$ on the final time step of the simulation filtered at the horizon scale $(aH)^{-1}$ plotted as a function of $\nu$.}
%    \label{fig:zetapeak}
%  \end{figure}
%}

% Figure for 2d histograms of $\dd\zeta/\dd\alpha$ against $\phi^A$, $\Pi^A$.
%\newcommand{\Fdzetaphasespace}{
%  \begin{figure}
%    \centering
%    \begin{subfigure}[a]{0.45\textwidth}
%      \includegraphics[width=\textwidth]{\FFdzetaphasespaceA}
%      \caption{}
%    \end{subfigure}
%    \begin{subfigure}[a]{0.45\textwidth}
%      \includegraphics[width=\textwidth]{\FFdzetaphasespaceB}
%      \caption{}
%    \end{subfigure}
%    \begin{subfigure}[b]{0.45\textwidth}
%      \includegraphics[width=\textwidth]{\FFdzetaphasespaceC}
%      \caption{}
%    \end{subfigure}
%    \begin{subfigure}[b]{0.45\textwidth}
%      \includegraphics[width=\textwidth]{\FFdzetaphasespaceD}
%      \caption{}
%    \end{subfigure}
%    \caption{Histograms of $P(\frac{\dd\Delta\zeta}{\dd\alpha},\phi^A)$ and $P(\frac{\dd\Delta\zeta}{\dd\alpha},\Pi^A)$ at specified times are plotted in gray contours $P(\frac{\dd\Delta\zeta}{\dd\alpha},\phi^A|\Delta\zeta_{\mathrm{final}}>5\sigma)$ and $P(\frac{\dd\Delta\zeta}{\dd\alpha},\Pi^A|\Delta\zeta_{\mathrm{final}}>5\sigma)$ are overlayed in colour. The probability density spacing between contours is equivalent to a $0.5\sigma$ step for a Gaussian PDF matching the mode of the measured historgram.}
%    \label{fig:dzetaphasespace}
%  \end{figure}
%}

% Figure for 2d histogram of $\phi^A$, $\Pi^A$ conditioned on $\Delta\zeta$.
%\newcommand{\Fzetacondps}{
%  \begin{figure}
%    \centering
%    \begin{subfigure}[a]{0.45\textwidth}
%      \includegraphics[width=\textwidth]{\FFzetacondpsA}
%      \caption{}
%    \end{subfigure}
%    \begin{subfigure}[a]{0.45\textwidth}
%      \includegraphics[width=\textwidth]{\FFzetacondpsB}
%      \caption{}
%    \end{subfigure}
%    \begin{subfigure}[b]{0.45\textwidth}
%      \includegraphics[width=\textwidth]{\FFzetacondpsC}
%      \caption{}
%    \end{subfigure}
%    \begin{subfigure}[b]{0.45\textwidth}
%      \includegraphics[width=\textwidth]{\FFzetacondpsD}
%      \caption{}
%    \end{subfigure}
%    \caption{Histograms of phase space projections at specified times. The gray contours are for the whole field, colour contours are conditioned on $\Delta\zeta_{\mathrm{final}}>5\sigma$. The probability density spacing between contours is equivalent to a $0.5\sigma$ step for a Gaussian PDF matching the mode of the measured historgram.}
%    \label{fig:zetacondps}
%  \end{figure}
%}

% Figure of $\phi^A$, $\Pi^A$ trajectory contours with post selected trajectories shown in colour.
%\newcommand{\Fzetacondfldtraj}{
%  \begin{figure}
%    \centering
%    \begin{subfigure}[a]{0.45\textwidth}
%      \includegraphics[width=\textwidth]{\FFzetacondfldtrajA}
%      \caption{}
%      \label{fig:zetacondfldtrajA}
%    \end{subfigure}
%    \begin{subfigure}[a]{0.45\textwidth}
%      \includegraphics[width=\textwidth]{\FFzetacondfldtrajB}
%      \caption{}
%      \label{fig:zetacondfldtrajB}
%    \end{subfigure}
%    \begin{subfigure}[b]{0.45\textwidth}
%      \includegraphics[width=\textwidth]{\FFzetacondfldtrajC}
%      \caption{}
%      \label{fig:zetacondfldtrajC}
%    \end{subfigure}
%    \begin{subfigure}[b]{0.45\textwidth}
%      \includegraphics[width=\textwidth]{\FFzetacondfldtrajD}
%      \caption{}
%      \label{fig:zetacondfldtrajD}
%    \end{subfigure}
%    \caption{Distribution of trajectories for \ref{fig:zetacondfldtrajA} $\phi$, \ref{fig:zetacondfldtrajB} $\chi$, \ref{fig:zetacondfldtrajC} $\frac{\dd\delta\phi}{\dd\alpha}$, \ref{fig:zetacondfldtrajD} $\frac{\dd\delta\chi}{\dd\alpha}$. Gray scale contours are for all trajectories, coloured contours are for trajectories post selected with asypmtotic $\Delta\zeta > 5\sigma$. Contours are drawn relative to the maximum probablity density for a given field on each time slice and spaced at intervals of $\exp(-n^2/8)$ for $n=1,2,...$, corresponding to one contour per $\sigma/2$ for a Gaussian distribution. \\ The extreme trajectories are found to be those which undergo the largest excursions in the $\chi$ direction during the instability and interact nonlinearly into $\phi$ and $\Pi_\phi$.}
%    \label{fig:zetacondfldtraj}
%  \end{figure}
%}

% Figure of $\Delta\zeta$ and $\dd\zeta_A/\dd\alpha$ trajectory contours with post selected trajectories shown in colour.
%\newcommand{\Fzetacondzetatraj}{
%  \begin{figure}
%    \centering
%    \begin{subfigure}[a]{0.45\textwidth}
%      \includegraphics[width=\textwidth]{\FFzetacondzetatrajA}
%      \caption{}
%      \label{fig:zetacondzetatrajA}
%    \end{subfigure}
%    \begin{subfigure}[a]{0.45\textwidth}
%      \includegraphics[width=\textwidth]{\FFzetacondzetatrajB}
%      \caption{}
%      \label{fig:zetacondzetatrajB}
%    \end{subfigure}
%    \caption{Distribution of trajectories for \ref{fig:zetacondzetatrajA} $\Delta\zeta$, \ref{fig:zetacondzetatrajB} $\frac{\dd\zeta_{\phi^A}}{\dd\alpha}$. Gray scale contours are for all trajectories, coloured contours are for trajectories post selected with asypmtotic $\Delta\zeta > 5\sigma$. Contour spacing is determined similarly to figure \ref{fig:zetacondfldtraj}. \\ The extreme trajectories show a reversal in the sign of $\Delta\zeta$. Examining the $\Delta\zeta$ production via the $\frac{\dd\zeta_\phi}{\dd\alpha}$ and $\frac{\dd\zeta_\chi}{\dd\alpha}$ channels shows two distinct pulses of production. In the first pulse the post selected trajectories undergo a negative production of $\Delta\zeta$ through the $\frac{\dd\zeta_\chi}{\dd\alpha}$ channel. In the second pulse these trajectories undergo positive production through the $\frac{\dd\zeta_\phi}{\dd\alpha}$ channel.}
%    \label{fig:zetacondzetatraj}
%  \end{figure}
%}
