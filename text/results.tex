% Document for results section

\section{Results} \label{sec:results}

In this section we present the lattice simulation results for inflation proceeding through an incomplete phase transition paying particular attention to the generation of $\zeta$ NG. In Sec \ref{sec:zeta ng} we calculate $\Delta\zeta$, the $\zeta$ response to the potential feature $\Delta V$, and find the NG to be spatially concentrated into peaks. We then compute the $\zeta$ peak statistics as a measure of NG. In Sec. \ref{sec:field dynamics} we give an account of the field dynamics of the system including a determination for the onset of nonlinearity. Finally, in Sec. \ref{sec:zeta production} we examine the relation between the field dynamics and sourcing of $\Delta\zeta$. Using the asymptotic $\Delta\zeta$ to perform a post-selection of lattice sites we isolate the production of the NG concentrations of $\Delta\zeta$ to those trajectories which undergo the most extreme excursions during the $\Delta V$ instability and subsiquently back-react on $\phi$.

%\textcolor{red}{[Part about the potential parameters we used, and mention how the concentrated NG is a general feature over a range of parameters].}

\subsection{$\zeta$ NG} \label{sec:zeta ng}
While $\zeta$ as sourced by \eqref{eq:zeta source fld} can be integrated terms of quantities known on the lattice, the result is the sum of both a Gaussian and NG component, $\zeta_\mathrm{G}$ and $\zeta_\mathrm{NG}$ respectively. For the purpose of characterizing $\zeta_\mathrm{NG}$ we would like to subtract off $\zeta_\mathrm{G}$, effectively increasing the signal to noise on the NG part of the singal. However, since $\zeta_\mathrm{G}$ is unkown we cannot subtract it from $\zeta$ directly, we can however construct a correleated quantity.

To do this we run two simulations from identical initial conditions, one using the full potential $V=V_0+\Delta V$ and the other using the baseline potential $V_0$. We define the difference $\Delta\zeta=\zeta-\zeta_0$ between the to simulations where $\zeta$ is computed using the full potential $V$, and $\zeta_0$ using only the baseline potential $V_0$, with the difference being taken on slices of constant $\tau_\mathrm{FRW}$. To the extent that $\zeta_0$ correlates with $\zeta_\mathrm{G}$ (this correlation relies on the use of identical initial condtions between the runs), subtracting it from the total $\zeta$ will concentrate the NG part into $\Delta\zeta$. This subtraction between paired runs is shown in Fig. \ref{fig:zetaslice} for a two dimensional slice through the lattice. \textcolor{red}{[Could give a teaser of the differential $\dd\zeta/\dd u$, but I think we should leave a full exploration of that to a future paper.]}

From Fig. \ref{fig:zetaslice} we can see the largest contributions to $\Delta\zeta$ come in the form of local concentrations interspersed over a more uniform field. Although these concentrations of $\Delta\zeta$ are local they represent a radical departure from the perturbative description of local NG $\zeta = \zeta_\mathrm{G} + f_{NL}\zeta^2_\mathrm{G}$. Here the form of NG beign generated is such that $\Delta\zeta$ and $\zeta_0$ are effecively uncorrelated.

The local concentration of $\Delta\zeta$ suggests characterizing this form of NG will be more usefully done in real space rather than Fourier space, where the local concentration implies the signal will encoded in the phase correlations of a broad range of Fourier modes. Peak counts, which have well known statistics for Gaussian fields, provide a suitable measure for this form of NG in real space. In Fig. \ref{fig:zetapeak} the peak counts of $\zeta$, $\zeta_0$, and $\Delta\zeta$ are compared to the Gaussian expectation for Gaussian fields with matching spectra. 

Here it is worth pointing out that in calculating $\Delta\zeta$ we are deriving a substantial benifit from the having the full field information on the lattice. It has allowed us to calculate a realization of the NG. So rather than making a blind search for what statistics best characterizes the NG of a particular system we can use the knowledge gained by having realizations of the NG to \textcolor{red}{[part about tailoring NG stats and searches].}

\textcolor{red}{[should probably point out peak counts won't be an optimal measure, since information is being discarded about the peak-peak correlation and peak profiles, but in principle a better measure can be constructed using $\Delta\zeta$].}

%This assumption informs our choice of $V_0$ as a quadratic potential, but also means $\Delta\zeta$ is by construction blind to any NG in the initial condition.
%So rather than identifying $\Delta\zeta$ as a proxy to $\zeta_\mathrm{NG}$ we might more correctly identify it as the response of $\zeta$ to the deformation of the potential by $\Delta V$.

\Fzetaslice
\Fzetapeak

\subsection{Field Dynamics} \label{sec:field dynamics}
% definition of $\delta$
% Intro paragraph
% Anologous to an in-out process ...
% initially linear growth of transverse fluctuations
% onset of nonlinearity
% system relaxes

% Linear evolution and spectra
At the linear level the mean field $\langle\phi\rangle$ provides a clock which modulates the effective mass of the $\chi$ field, with $m^2_\mathrm{eff}$ becoming negative as $\langle\phi\rangle$ crosses $\Delta V$ as seen in Fig. \ref{fig:potparam}. While $m^2_\mathrm{eff}<0$, $\tilde{\chi}_k$ and $\tilde{\Pi}^\chi_k$ fluctuations within the instabilty band undergo exponential growth, building the correlation $\langle\tilde{\chi}^*_k\tilde{\Pi}^\chi_k\rangle$ as the system within the instability band approaches the growing mode.

We plot a selection of the spectra and cross-spectra real parts as measured from the lattice for the baseline case of $V=V_0$ and in Fig. \ref{fig:specbl} and for the full potential $V=V_0 + \Delta V$ in Fig. \ref{fig:spec}.

\Fspecbl
\Fspec
\Fspecdet

% Breakdown of linearity and specdet
Eventually the growth of fluctuations will lead to the onset of nonlinearity and the information provided by the spectrum will need to be supplements with other measures to get full picture of the evolution of the fields.

We can construct a test for nonlinearity by considering the phase space density of the system and constructing the quantity
\begin{equation}
  C(\phi^A,\Pi^A;k) =
  \mathrm{Re}\left[\langle\tilde{Z}_k\tilde{Z}_k^\dagger\rangle\right],
\end{equation}
where $Z^T=[\phi,\Pi^\phi,\chi,\Pi^\chi]$. Classical evolution of the system, as is done for evolution on the lattice, conseverves phase space density. If the system evolves linearly then each Fourier mode evolves independently and provided the modes of the system are initially independent a stronger statement can be made that the phase space density of the system is conserved mode-by-mode. When the averages in $C(\phi^A,\Pi^A;k)$ are taken over an ensemble of lattice realizations (with Gaussian statistics) the quantity $\sqrt{\mathrm{det}(C(\phi^A,\Pi^A;k))}$ provides a measure of the partial phase space volume for a single mode. So monitoring $\sqrt{\mathrm{det}(C(\phi^A,\Pi^A;k))}$ allows us to detect the onset of nonlinearity via non-conservation of the partial phase space volume of individual modes.

Although properly the averages in $C(\phi^A,\Pi^A;k)$ should be taken over to an ensemble of lattice realizations, we find that taking the averages over bands of $|k|$ for a single realization of the lattice still allows detection of the onset of nonlinearity \textcolor{red}{[possible caveat for the low statistics of the longwavength modes]}. In Fig. \ref{fig:specdet} we measure $\det(C(\phi^A,\Pi^A;k))$ on the lattice \textcolor{red}{[part about averagin gin $|k|$ bands]} \textcolor{red}{[summary of Fig. \ref{fig:specdet}, still need to add the figure].}

%\Fspecdet

% Real space
% Motivated by how $\Delta\zeta$ was better seen in real space, so let's look at things in a real space basis.
% Advantage of using a real space basis is concentrations of NG can be easily identified in real space from $\Delta\zeta$, so we can perform post selection based on a local criteria of $\Delta\zeta$.
% What we learn is: 
% Forward reference that a post selection or the lattice sites will be done in the next section

\textcolor{red}{[I think the main line I want to follow for the nonlinear part goes something like:}
  \begin{itemize}
    \color{red}
    \item nonlinearity sets up some nonGaussian correlations in the system
    \item we can solve the system numerically, but a priori don't know the natural basis for the correlations
    \item we look at the correlations in real space (one-point), because $\Delta\zeta$ concentrates in real space
    \item doing this in real space will be useful in identifying what leads to the $\Delta\zeta$ concentrations since their positions can be isolated in real space and used to make a post selection on the lattice
  \end{itemize}
      
In Sec. \ref{sec:zeta ng} we learned that $\Delta\zeta$ is concentrated in real space, this suggests the real space description of the fields may be more useful than the Fourier space one for this system once it has become nonlinear. In Fig. \ref{fig:slice} we plot two dimensional slices through a three dimensional lattice of the fields and their momenta \textcolor{red}{[mention time slice is constant $\tau_\mathrm{FRW}$ where $\langle\phi\rangle=\phi_p-\phi_w$]}. Comparing Fig. \ref{fig:slice} back to Fig. \ref{fig:zetaslice} we can see the same pattern of $\Delta\zeta$ concentrations are already imprinted on $\delta\phi$ at the earlier time (similarly with $\Pi^\phi$), we will investigate this correlation further in Sec. \ref{sec:zeta production}.
\textcolor{red}{[Effective description point by point, relate to separate universe approximation and adjustment to the fluctation condensate split.]}

\Fslice

% Phase space
% Motivated by correlations set up by the nonlinearity
% What we learn is: The effect of onset of nonlinearity has been to build up a nonlinear correlation between $\phi$ and $\chi$ (likewise with their momenta). This gives rise to a one-sided tail in the $(\phi,\Pi^\phi)$ plane. The correlation of the $(\chi,\Pi^\chi)$ plane has not developed the distinct shape of two lobes with a thin connection confirming the phase transition to be incomplete.

% The interpretation is those trajectories which undergo the most growth in terms of $\chi$ fluctuations interact nonlinearly to slow $\Pi^\phi$.

\textcolor{red}{[Put in some connection to the previous paragraph.]}
In principle the phase space on the lattice is very high dimensional, with each lattice site hosting it's own degrees of freedom (though technically not all independent as initializing the lattice with a $k$ space cutoff below the Nyquist frequency imposes a constraint). A complete phase space distribution would have to be built up from a large number of simulations, each sampling a single tajectory through this phase space. Here, rather than dealing with the full phase space we will build a distribution by stacking the degrees of each lattice site, effectively substituting a volume for an ensemble while marginalizing over position.

In Fig. \ref{fig:phasespace} we plot several projections of this position marginalized phase space taken at fixed time $\tau_\mathrm{FRW} = \tau_\mathrm{Off}$, when $\langle\phi\rangle = \phi_p - \phi_w$. The $P(\chi,\Pi^\chi)$ projection shows the \textcolor{red}{[stretching out during instability, relate to Fourier modes and frozen phase]}... As the case we are considering is an incomplete phase transition \textcolor{red}{[part about $\chi$, $\Pi^\chi$ correlation becoming bimodal if phase transition completes]}...

For a completed phase transition $P(\chi,\Pi^\chi)$ becomes nonlinear and develops a bimodality as the system organizes into domains. However, in the present case of an incomplete phase transition we are examining a system away from that regime.

The $P(\chi,\phi)$ and $P(\Pi^\chi,\Pi^\phi)$ projections show these correlations between the fields have been made visibly nonlinear. Consistant with the nonlinear correlations of $P(\phi,\chi)$ and $P(\Pi^\phi,\Pi^\chi)$, is the extended tail in $P(\phi,\Pi^\phi)$ towards positive values of $\delta\phi$ and $\delta\Pi^\phi$. Comparison to Fig. \ref{fig:slice}, which shows a real space slice of the fields at the same time, this tail manifests in real space as collection of prominant peaks \textcolor{red}{[we will see in Sec. \ref{sec:zeta production} ... the role these peaks play in $\zeta$ NG]}. 

% Might want to look at the spatial derivative quantities as well.
% Should probably do fitting or something for the $\phi$, $\chi$ phase space projection plot.

\textcolor{red}{[Put in a part about how the changing $\Delta V$ creates a stressed system and when $\Delta V \to 0$ this built up strain in the system is allowed to relax. Is that the way to think about this?]}

\Fphasespace

% Pointwise trajectories
% Provide supplimental information about time dependence
% I should maybe mark the onset of nonlinearity on the trajectory plots
% What we learn is: The time dependence of the fields and momenta

\textcolor{red}{[Connect the text to Fig. \ref{fig:traj} in terms of time dependence of the fields and momenta, basically just supplimental information to Fig. \ref{fig:phasespace}.]}

In Fig. \ref{fig:traj} density contours of the trajectories of the fields marginalized over postion are plotted.  

\Ftraj

% Summary of field dynamics section
\textcolor{red}{[Put in a summary paragraph for the field dynamics.]}

% How to talk about the $\chi$ condensate
% From the phase space figure we know there is nonlinearity between $\phi$ and $\chi$ which is strongest at large $|\chi|$ as measured locally.
% Viewing the fields in real space will then reveal the local structure of where the nonlinear interactions are occuring.
%The large amplitude of the $\chi$ condensate allows the fluctuations to explore nonlinear parts of the potential.
%A course graining of the $\chi$ fluctuations can be viewed as a condensate which carries fluctuations into nonlinear portions of the potential.


%\subsection{Average Inflation parameters} \label{sec:infl}
% Discuss things like energy fractions and the slow roll parameters here.

\subsection{Sourcing NG} \label{sec:zeta production}
% This subsection is to tie together the field dynamics and NG of $\Delta\zeta$.

% Give a name to the post selected trajectories, maybe "extreme excursions" or "extreme trajectories".

% Things I want to cover:
% The idea that trajectories through a deformed potential pick up different $\epsilon$ histories
% $\Delta\zeta$ is produced most strongly along those trajectories which undergo the largest $\chi$ excursions
% There are two main phases in the $\Delta\zeta$ production: when the trajectory goes through the $\chi$ excursion, and when $\dot{\phi}$ is slowed with the symmetry restoration

% Figures I want to include:
% 2d historgam showing $\dd\Delta\zeta/\dd\alpha$ vs $\phi^A$ and $\Pi^A$.

Having seen both the field dynamics and $\Delta\zeta$ in this system we now turn to the relation between the two.
We have seen $\Delta\zeta$ is dominated by localized features where deviations from Gaussianity are much greater than in the surrounding field, the story of NG in this system is about outliers rather than averages.

%A complete description of the system can be supplied both Fourier and position space (along with any number of other bases), the basis that is chosen makes no difference to the dynamics.
%However, when post selecting trajectories we project out elements of whichever basis we choose to use, the system will no longer be complete.
%In order to be of practical use we must perform the post selection in a basis which preserves the information relavent to the formation of NG.

To focus on those parts of the system which lead to the most prominent features of NG we perform a post selection based on the asymptotic $\Delta\zeta$.
Ideally this post selection would be done in a basis which optimally encodes all information relevant to the formation of NG in this system.
However, construction of such a basis is elusive and practical terms, although not optimal, the bases of position space or Fourier space are the available options. 
The most prominent features of NG being localized suggests post selecting trajectories in position space will be more useful, as a Fourier basis encodes localization in the phase coherence between a possibly large number of modes.

We post select lattice sites where the asymptotic $\Delta\zeta$ deviates by $\nu \ge 5$ standard deviations from the mean and will refer to their trajectories as `extreme trajectories'.
The dynamics of the extreme trajectories forms a subset of the field dynamics of as a whole.

In Fig. \ref{fig:traj} we compare trajectory distributions of the whole field to those of the extreme trajectories, colouring the latter in red. This shows the extreme trajectories, post selected on $\Delta\zeta$, are exactly those trajectories which are driven to the largest excursion away from $\chi=0$ during the transverse instability.
Likewise, in Fig. \ref{fig:phasespace} we compare distributions of phase space projections of the whole field to those of the extreme trajectories.
Nonlinear efects on the extreme trajectories have built NG correlations in the $(\phi,\chi)$ plane (likewise with the momenta) and to populate the extended tail of the $\langle\phi\Pi^\phi\rangle$ correlation.

Time dependence for the production of $\Delta\zeta$ is shown in Fig. \ref{fig:zetatraj}. The extreme trajectories undergo a sign change in $\Delta\zeta$. This sign change can be traced to a two stage production process: initial negative production through the $\frac{\dd\Delta\zeta_\chi}{\dd\alpha}$ channel, followed by a delayed positive production through the $\frac{\dd\Delta\zeta_\phi}{\dd\alpha}$ channel.
\textcolor{red}{[Discuss how broadly applicable this picute is, ie what about very weak or very strong instabilities.]}

\Fzetatraj

\textcolor{red}{[Discuss sourcing through the $\frac{\dd\Delta\zeta_\chi}{\dd\alpha}$]}
\begin{itemize}
  \color{red}
  \item Has no contribution at the linear level since $\langle\dot{\chi}\rangle=0$.
  \item Sourcing through this channel is due to the instability building up a $\dot{\chi}$ condensate, so the local inflaton direction picks up a $\dot{\chi}$ contribution (I think we can support this).
\end{itemize}

\textcolor{red}{[Discuss sourcing through the $\frac{\dd\Delta\zeta_\phi}{\dd\alpha}$]}
\begin{itemize}
  \color{red}
  \item At the linear level $\zeta$ is sourced through the $\dot{\phi}\nabla^2\phi$ term.
  \item Sourcing $\Delta\zeta$ through this channel means $\phi$ has been deviated from its baseline trajectory, in this system that is a nonlinear effect.
  \item Because of this, this channel turns on nonlinearly with the strength of the instability.
  \item For sufficiently weak instabilities this channel is subdominant in the sourcing of $\Delta\zeta$.
\end{itemize}

%In order for the $\frac{\dd\Delta\zeta_\phi}{\dd\alpha}$ channel to be activated there must be sufficient growth in the $\chi$ fluctuations to interact nonlinearly with $\phi$.
%Because of this, for sufficiently weak instabilities the sign change of $\Delta\zeta$ is absent.
%The case of instabilities sufficiently weak to not activate the $\frac{\dd\zeta_\phi}{\dd\alpha}$ production channel is left to future work along with a thorough exploration of the parameter space.


\textcolor{red}{[Put in a part comparing with $\Delta\phi$ since this provides a tighter correlation.]}
\Fdeltaps
\Fdeltaphizetacorr
\Fzetaslicemulti
