% Document for results section

\section{Results} \label{sec:results}

In this section we present the lattice simulation results for inflation proceeding through an incomplete phase transition paying particular attention to the generation of $\zeta$ NG. In Sec \ref{sec:zeta ng} we calculate $\Delta\zeta$, the $\zeta$ response to the potential feature $\Delta V$, and find the NG to be spatially concentrated into peaks. We then compute the $\zeta$ peak statistics as a measure of NG. In Sec. \ref{sec:field dynamics} we give an account of the field dynamics of the system including a determination for the onset of nonlinearity. Finally, in Sec. \ref{sec:zeta production} we examine the relation between the field dynamics and sourcing of $\Delta\zeta$. Using the asymptotic $\Delta\zeta$ to perform a post-selection of lattice sites we isolate the production of the NG concentrations of $\Delta\zeta$ to those trajectories which undergo the most extreme excursions during the $\Delta V$ instability and subsequently back-react on $\phi$.

%\textcolor{red}{[Part about the potential parameters we used, and mention how the concentrated NG is a general feature over a range of parameters].}

\subsection{$\zeta$ NG} \label{sec:zeta ng}
While $\zeta$ as sourced by \eqref{eq:zeta source fld} can be integrated terms of quantities known on the lattice, the result is the sum of both a Gaussian and NG component, $\zeta_\mathrm{G}$ and $\zeta_\mathrm{NG}$ respectively. For the purpose of characterizing $\zeta_\mathrm{NG}$ we would like to subtract off $\zeta_\mathrm{G}$, effectively increasing the signal to noise on the NG part of the signal. However, since $\zeta_\mathrm{G}$ is unknown we cannot subtract it from $\zeta$ directly, we can however construct a correlated quantity.

To do this we run two simulations from identical initial conditions, one using the full potential $V=V_0+\Delta V$ and the other using the baseline potential $V_0$. We define the difference $\Delta\zeta=\zeta-\zeta_0$ between the to simulations where $\zeta$ is computed using the full potential $V$, and $\zeta_0$ using only the baseline potential $V_0$, with the difference being taken on slices of constant $\tau_\mathrm{FRW}$. To the extent that $\zeta_0$ correlates with $\zeta_\mathrm{G}$ (this correlation relies on the use of identical initial conditions between the runs), subtracting it from the total $\zeta$ will concentrate the NG part into $\Delta\zeta$. This subtraction between paired runs is shown in Fig. \ref{fig:zetaslice} for a two dimensional slice through the lattice. \textcolor{red}{[Could give a teaser of the differential $\dd\zeta/\dd u$, but I think we should leave a full exploration of that to a future paper.]}

From Fig. \ref{fig:zetaslice} we can see the largest contributions to $\Delta\zeta$ come in the form of local concentrations interspersed over a more uniform field. Although these concentrations of $\Delta\zeta$ are local they represent a radical departure from the perturbative description of local NG $\zeta = \zeta_\mathrm{G} + f_{NL}\zeta^2_\mathrm{G}$. Here the form of NG being generated is such that $\Delta\zeta$ and $\zeta_0$ are effectively uncorrelated.

The local concentration of $\Delta\zeta$ suggests characterizing this form of NG will be more usefully done in real space rather than Fourier space, where the local concentration implies the signal will encoded in the phase correlations of a broad range of Fourier modes. Peak counts, which have well known statistics for Gaussian fields, provide a suitable measure for this form of NG in real space. In Fig. \ref{fig:zetapeak} the peak counts of $\zeta$, $\zeta_0$, and $\Delta\zeta$ are compared to the Gaussian expectation for Gaussian fields with matching spectra. 

Here it is worth pointing out that in calculating $\Delta\zeta$ we are deriving a substantial benefit from the having the full field information on the lattice. It has allowed us to calculate a realization of the NG. So rather than making a blind search for what statistics best characterizes the NG of a particular system we can use the knowledge gained by having realizations of the NG to \textcolor{red}{[part about tailoring NG stats and searches].}

\textcolor{red}{[should probably point out peak counts won't be an optimal measure, since information is being discarded about the peak-peak correlation and peak profiles, but in principle a better measure can be constructed using $\Delta\zeta$].}

%This assumption informs our choice of $V_0$ as a quadratic potential, but also means $\Delta\zeta$ is by construction blind to any NG in the initial condition.
%So rather than identifying $\Delta\zeta$ as a proxy to $\zeta_\mathrm{NG}$ we might more correctly identify it as the response of $\zeta$ to the deformation of the potential by $\Delta V$.

\Fzetaslice
\Fzetapeak

\subsection{Field Dynamics} \label{sec:field dynamics}
% definition of $\delta$
% initially linear growth of transverse fluctuations
% onset of nonlinearity
% system relaxes

% Linear evolution and spectra
At the linear level the mean field $\langle\phi\rangle$ provides a clock which modulates the effective mass of the $\chi$ field, with $m^2_\mathrm{eff}$ becoming negative as $\langle\phi\rangle$ crosses $\Delta V$ as seen in Fig. \ref{fig:potparam}. While $m^2_\mathrm{eff}<0$, $\tilde{\chi}_k$ and $\tilde{\Pi}^\chi_k$ fluctuations within the instability band undergo exponential growth, building the correlation $\langle\tilde{\chi}^*_k\tilde{\Pi}^\chi_k\rangle$ as the system within the instability band approaches the growing mode.

We plot a selection of the spectra and cross-spectra real parts as measured from the lattice for the baseline case of $V=V_0$ and in Fig. \ref{fig:specbl} and for the full potential $V=V_0 + \Delta V$ in Fig. \ref{fig:spec}.

\Fspecbl
\Fspec
\Fspecdet

% Breakdown of linearity and specdet
Eventually the growth of fluctuations will lead to the onset of nonlinearity and the information provided by the spectrum will need to be supplements with other measures to get full picture of the evolution of the fields.

We can construct a test for nonlinearity by considering the phase space density of the system and constructing the quantity
\begin{equation}
  C(\phi^A,\Pi^A;k) =
  \mathrm{Re}\left[\langle\tilde{Z}_k\tilde{Z}_k^\dagger\rangle\right],
\end{equation}
where $Z^T=[\phi,\Pi^\phi,\chi,\Pi^\chi]$. Classical evolution of the system, as is done for evolution on the lattice, conserves phase space density. If the system evolves linearly then each Fourier mode evolves independently and provided the modes of the system are initially independent a stronger statement can be made that the phase space density of the system is conserved mode-by-mode. When the averages in $C(\phi^A,\Pi^A;k)$ are taken over an ensemble of lattice realizations (with Gaussian statistics) the quantity $\sqrt{\mathrm{det}(C(\phi^A,\Pi^A;k))}$ provides a measure of the partial phase space volume for a single mode. So monitoring $\sqrt{\mathrm{det}(C(\phi^A,\Pi^A;k))}$ allows us to detect the onset of nonlinearity via non-conservation of the partial phase space volume of individual modes.

Although properly the averages in $C(\phi^A,\Pi^A;k)$ should be taken over to an ensemble of lattice realizations, we find that taking the averages over bands of $|k|$ for a single realization of the lattice still allows detection of the onset of nonlinearity \textcolor{red}{[possible caveat for the low statistics of the long wavelength modes]}. In Fig. \ref{fig:specdet} we measure $\det(C(\phi^A,\Pi^A;k))$ on the lattice \textcolor{red}{[part about averaging in $|k|$ bands]} \textcolor{red}{[summary of Fig. \ref{fig:specdet}, still need to add the figure].}

%\Fspecdet

% Real space
% Motivated by how $\Delta\zeta$ was better seen in real space, so let's look at things in a real space basis.
% Advantage of using a real space basis is concentrations of NG can be easily identified in real space from $\Delta\zeta$, so we can perform post selection based on a local criteria of $\Delta\zeta$.
% What we learn is: 
% Forward reference that a post selection or the lattice sites will be done in the next section

\textcolor{red}{[I think the main line I want to follow for the nonlinear part goes something like:}
  \begin{itemize}
    \color{red}
    \item nonlinearity sets up some nonGaussian correlations in the system
    \item we can solve the system numerically, but a priori don't know the natural basis for the correlations
    \item we look at the correlations in real space (one-point), because $\Delta\zeta$ concentrates in real space
    \item doing this in real space will be useful in identifying what leads to the $\Delta\zeta$ concentrations since their positions can be isolated in real space and used to make a post selection on the lattice
  \end{itemize}
      
In Sec. \ref{sec:zeta ng} we learned that $\Delta\zeta$ is concentrated in real space, this suggests the real space description of the fields may be more useful than the Fourier space one for this system once it has become nonlinear. In Fig. \ref{fig:slice} we plot two dimensional slices through a three dimensional lattice of the fields and their momenta \textcolor{red}{[mention time slice is constant $\tau_\mathrm{FRW}$ where $\langle\phi\rangle=\phi_p-\phi_w$]}. Comparing Fig. \ref{fig:slice} back to Fig. \ref{fig:zetaslice} we can see the same pattern of $\Delta\zeta$ concentrations are already imprinted on $\delta\phi$ at the earlier time (similarly with $\Pi^\phi$), we will investigate this correlation further in Sec. \ref{sec:zeta production}.
\textcolor{red}{[Effective description point by point, relate to separate universe approximation and adjustment to the fluctuation condensate split.]}

\Fslice

% Phase space
% Motivated by correlations set up by the nonlinearity
% What we learn is: The effect of onset of nonlinearity has been to build up a nonlinear correlation between $\phi$ and $\chi$ (likewise with their momenta). This gives rise to a one-sided tail in the $(\phi,\Pi^\phi)$ plane. The correlation of the $(\chi,\Pi^\chi)$ plane has not developed the distinct shape of two lobes with a thin connection confirming the phase transition to be incomplete.

% The interpretation is those trajectories which undergo the most growth in terms of $\chi$ fluctuations interact nonlinearly to slow $\Pi^\phi$.

\textcolor{red}{[Put in some connection to the previous paragraph.]}
In principle the phase space on the lattice is very high dimensional, with each lattice site hosting it's own degrees of freedom (though technically not all independent as initializing the lattice with a $k$ space cutoff below the Nyquist frequency imposes a constraint). A complete phase space distribution would have to be built up from a large number of simulations, each sampling a single trajectory through this phase space. Here, rather than dealing with the full phase space we will build a distribution by stacking the degrees of each lattice site, effectively substituting a volume for an ensemble while marginalizing over position.

In Fig. \ref{fig:phasespace} we plot several projections of this position marginalized phase space taken at fixed time $\tau_\mathrm{FRW} = \tau_\mathrm{Off}$, when $\langle\phi\rangle = \phi_p - \phi_w$. The $P(\chi,\Pi^\chi)$ projection shows the \textcolor{red}{[stretching out during instability, relate to Fourier modes and frozen phase]}... As the case we are considering is an incomplete phase transition \textcolor{red}{[part about $\chi$, $\Pi^\chi$ correlation becoming bimodal if phase transition completes]}...

For a completed phase transition $P(\chi,\Pi^\chi)$ becomes nonlinear and develops a bimodality as the system organizes into domains. However, in the present case of an incomplete phase transition we are examining a system away from that regime.

The $P(\chi,\phi)$ and $P(\Pi^\chi,\Pi^\phi)$ projections show these correlations between the fields have been made visibly nonlinear. Consistent with the nonlinear correlations of $P(\phi,\chi)$ and $P(\Pi^\phi,\Pi^\chi)$, is the extended tail in $P(\phi,\Pi^\phi)$ towards positive values of $\delta\phi$ and $\delta\Pi^\phi$. Comparison to Fig. \ref{fig:slice}, which shows a real space slice of the fields at the same time, this tail manifests in real space as collection of prominent peaks \textcolor{red}{[we will see in Sec. \ref{sec:zeta production} ... the role these peaks play in $\zeta$ NG]}. 

% Might want to look at the spatial derivative quantities as well.
% Should probably do fitting or something for the $\phi$, $\chi$ phase space projection plot.

\textcolor{red}{[Put in a part about how the changing $\Delta V$ creates a stressed system and when $\Delta V \to 0$ this built up strain in the system is allowed to relax. Is that the way to think about this?]}

\Fphasespace

% Pointwise trajectories
% Provide supplimental information about time dependence
% I should maybe mark the onset of nonlinearity on the trajectory plots
% What we learn is: The time dependence of the fields and momenta

\textcolor{red}{[Connect the text to Fig. \ref{fig:traj} in terms of time dependence of the fields and momenta, basically just supplemental information to Fig. \ref{fig:phasespace}.]}

In Fig. \ref{fig:traj} density contours of the trajectories of the fields marginalized over position are plotted.  

\Ftraj

% Summary of field dynamics section
\textcolor{red}{[Put in a summary paragraph for the field dynamics.]}

% How to talk about the $\chi$ condensate
% From the phase space figure we know there is nonlinearity between $\phi$ and $\chi$ which is strongest at large $|\chi|$ as measured locally.
% Viewing the fields in real space will then reveal the local structure of where the nonlinear interactions are occuring.
%The large amplitude of the $\chi$ condensate allows the fluctuations to explore nonlinear parts of the potential.
%A course graining of the $\chi$ fluctuations can be viewed as a condensate which carries fluctuations into nonlinear portions of the potential.


%\subsection{Average Inflation parameters} \label{sec:infl}
% Discuss things like energy fractions and the slow roll parameters here.

\subsection{Sourcing NG} \label{sec:zeta production}
% This subsection is to tie together the field dynamics and NG of $\Delta\zeta$.

% Things I want to cover:
% The idea that trajectories through a deformed potential pick up different $\epsilon$ histories
% $\Delta\zeta$ is produced most strongly along those trajectories which undergo the largest $\chi$ excursions
% There are two main phases in the $\Delta\zeta$ production: when the trajectory goes through the $\chi$ excursion, and when $\dot{\phi}$ is slowed with the symmetry restoration

Having seen both the field dynamics and $\Delta\zeta$ in this system we now turn to the relation between the two.
We have seen $\Delta\zeta$ is dominated by localized features where deviations from Gaussianity are much greater than in the surrounding field, the story of NG in this system is about outliers rather than averages.

To focus on those parts of the system which lead to the most prominent features of NG we perform a post selection based on the asymptotic $\Delta\zeta$.
Ideally this post selection would be done in a basis which optimally encodes all information relevant to the formation of NG in this system.
However, construction of such a basis is elusive and in practical terms, although not optimal, the bases of position space or Fourier space are the available options. 
The most prominent features of NG being localized suggests post selecting trajectories in position space will be more useful, as a Fourier basis encodes localization in the phase coherence between a possibly large number of modes.

% Simplicity of extreme trajectories
We post select lattice sites where the asymptotic $\Delta\zeta$ deviates by $\nu \ge 5 \sigma$ from the mean and will refer to their trajectories as `extreme trajectories'. Among these extreme trajectories there is an emergence of simplicty, which can be seen in Fig. \ref{fig:zetapstraj} were we plot the distribution of trajectories through the pseudo phase space of $\Delta\zeta$ versus $\frac{\dd\Delta\zeta}{\dd\alpha}$ both for the extreme trajectories and for all trajectories on the lattice. The extreme trajectories through the $\Delta\zeta$ versus $\frac{\dd\Delta\zeta}{\dd\alpha}$ plane are all rather similar and simply described as completing a three-quarter clockwise turn through the pseudo phase space, starting from an initial negative production of $\Delta\zeta$ and ending with an approach to a positive asymptotic $\Delta\zeta$. This is however not the case in general, trajectories which have not been post selected do not exhibit the same uniformity of behaviour and may exhibit more complicated behaviour not so easily described as that of the extreme trajectories. 

\Fzetapstraj

As the extreme trajectories admit a simple description, and are more relevant to the production of NG peaks in this system, we will focus on them during our discussion of the sourcing of $\Delta\zeta$. While we do so, keep in mind that a description of the extreme trajectories is not representitive of those trajectories which have not been post selected. The description we provide here is howver representitive of the sourcing of $\Delta\zeta$ leading to the most prominant peaks in this system.

% Extreme trajectories in the fields and momenta
In Fig. \ref{fig:traj} we compare trajectory distributions of the whole field to those of the extreme trajectories, colouring the latter in red. This shows the extreme trajectories, post selected on $\Delta\zeta$, have two clearly distinguishable features: they are exactly those trajectories for which $\chi$ is driven through the largest excursions away from $\chi=0$ by the transverse instability, and $\Pi^\phi$ receives the strongest nonlinear kick in the positive direction as the field pass through the end of $\Delta V$ (our convention of choosing and initial state with $\langle\phi\rangle$ positive and decreasing during inflation means a kick to $\Pi^\phi$ in the positive direction has the effect of slowing the local roll of $\phi$ down the potential).

Likewise, in Fig. \ref{fig:phasespace} we compare distributions of phase space projections of the whole field to those of the extreme trajectories on a time slice of uniform $\tau$ when $\langle\phi\rangle$ crosses the end of $\Delta V$. There nonlinear effects on the extreme trajectories build NG correlations in the $(\phi,\chi)$ plane (likewise with the momenta) and populate an extended tail in the $\langle\phi\Pi^\phi\rangle$ correlation.


The time dependence of the production of $\Delta\zeta$ shows, on the extreme trajectories, it is initially negative before undergoing a sign change in a distinctly two-staged production process. Separating the source term \eqref{eq:zeta source fld} into the channels \eqref{eq:zeta source lap} and \eqref{eq:zeta source gdg} for both $\phi$ and $\chi$ reveals the first stage of production is through the
%\eqref{eq:zeta source lap} channel for the $\chi$ field,
$\frac{\dd\Delta\zeta_{\dot{\chi}\nabla^2\chi}}{\dd\alpha}$ channel,
while the second stage of production which reverses the sign of $\Delta\zeta$ is through the
%\eqref{eq:zeta source lap} channel for the $\phi$ field.
$\frac{\dd\Delta\zeta_{\dot{\phi}\nabla^2\phi}}{\dd\alpha}$ channel.
Sourcing through the \eqref{eq:zeta source gdg} channels for both fields give subdominant contributions to the peaks. This time dependence is shown in Fig. \ref{fig:zetatraj} where we have plotted $\Delta\zeta$, its derivative, and the \eqref{eq:zeta source lap} production channels for both $\phi$ and $\chi$.

\Fzetatraj

The production of $\Delta\zeta$ sourced through the
$\frac{\dd\Delta\zeta_{\dot{\chi}\nabla^2\chi}}{\dd\alpha}$ channel
%\eqref{eq:zeta source lap} channel for the $\chi$ field
, when $\langle\dot{\chi}\rangle = 0$, has no contribution at the linear level, the leading order contribution being quadratic in the fluctuations of $\chi$. Never the less, the exponential growth of $\chi$ fluctuations during the transverse instability mean the contribution through this channel is non-negligible.
This is sourcing $\zeta$ NG through a nonlinear source term which is present even if considering only linear dynamics and Gaussian fluctuations of the fields.

Sourcing through the $\frac{\dd\Delta\zeta_{\dot{\phi}\nabla^2\phi}}{\dd\alpha}$ channel is activated when the evolution of $\phi$ and $\Pi^\phi$ deviate from the baseline case run in the $V_0$ potential (or by modifications to $\rho + P$ which may be due to $\chi$ and $\Pi^\chi$ at nonlinear level). This deviation from the baseline case is due to the nonlinear interactions between the $\phi$ and $\chi$ fields.
This is sourcing $\zeta$ NG through NG fluctuations of the fields which is present even if considering only the linear contribution of the source term.

\Fzetaslicemulti

% Move this to fit more with the flow, maybe before the zeta traj figure where the two stages of production are identified.
A sequence of two dimensional slices through the lattice in Fig. \ref{fig:zetaslicemulti} shows the spatial dependence of the evolution of $\Delta\zeta$. Initially the production of $\Delta\zeta$ forms a set of negative peaks, as the system continues to evolve and $\Delta\zeta$ approaches its asymptotic state these negative peaks are replaced by positive peaks at the same locations. Comparison to Fig. \ref{fig:zetatraj} shows the negative peaks correspond to the first stage of production through the $\frac{\dd\Delta\zeta_{\dot{\chi}\nabla^2\chi}}{\dd\alpha}$ channel while the change of sign to positive peaks corresponds to the second stage of production through the $\frac{\dd\Delta\zeta_{\dot{\phi}\nabla^2\phi}}{\dd\alpha}$ channel.

% Relaxation response
\subsection{$\zeta$ Production During the Relaxation of the Fields}
From Fig. \ref{fig:zetatraj} we can also see that a portion of the production of $\Delta\zeta$ occurs after the fields have passed through $\Delta V$ and are again feeling the effects of only of the $V_0$ part of the potential \textcolor{red}{[point this out in the Fig. \ref{fig:zetatraj} caption]}. This portion of the production occurs as the state of the system, being deformed by having passed through $\Delta V$, is allowed to relax to the attractor of the baseline potential. For this reason we will refer to it as the `relaxation stage'. \textcolor{red}{[I should make it clear that the relaxation stage is distinct for the two-stage production that is disussed above.]}

During the relaxation stage the sole role of $\Delta V$ is to generate an initial condition by deforming the state away from the attractor of the baseline potential. Once this deformed state has been generated the relaxation stage can be studied without reference to the details of how it was generated. So, although the details of how this initial state is prepared may be nontrivial and may involve its own generation of NG, the relaxation stage is decoupled from these details and can be treated separately.

\textcolor{red}{[Emphasize the point here that different choices of $\Delta V$ can still have commonality in their relaxation stages if they produce similar deformed states.]}

The production of $\Delta\zeta$ during the relaxation stage can be written as a response function to the deformed state generated by $\Delta V$. Using the subscript `off' to refer to the time at which $\Delta V$ turns off and the system is returned to the $V_0$ potential we call the initial deformed state $Z_\mathrm{off}$. The response function can then be written as 
\begin{equation}
  \Delta\zeta_f = R[Z_{off};V_0] + \Delta\zeta_\mathrm{off}.
\end{equation}
\textcolor{red}{[The argument is a set of fields not numbers, so I guess it's a response functional.]}
\textcolor{red}{[I'm using ${}_{off}$ for the initial state, which may be somewhat confusing notation.]}
\textcolor{red}{[Do I actually need $V_0$ here? What really matters is the attractor and how trajectories approach it.]}
\textcolor{red}{[Writing the response functional as $R[Z;Z_\mathrm{attractor}]$ is maybe better as it relates more easily to the $\epsilon$ vs $\epsilon_\mathrm{free}$ formalism for sourcing $\zeta$. Also, if I am claiming $\Delta\zeta_f = R[Z_{off};V_0] + \Delta\zeta_\mathrm{off}$ then I need to include some reference to $\zeta_0$ or $Z_0$ in the response functional.]}

% This part is still rough


While this is a generic description, a response functional which depends on a complete description of a set of fields and must be measured from simulations does little to simplify the description of the production of $\Delta\zeta$.
%To gain a simplified view we would like to expand $Z$ into a basis set and project ... transforming the response functional into a response function on a finite (and ideally small) number of variables. 
\textcolor{red}{[Part about projecting the response function onto a smaller basis that the full field to simplify the description (while introducing variance into the response)]}

Being a functional over fields, the complete $R[Z_{off};V_0]$ is impractical for reasons of dimensionality. A more manageable quantity can be obtained by projecting to the response onto a smaller set of variables. In Fig. \ref{fig:deltaphizetacorr} we measure the response of $\Delta\zeta_f$ to a single variable $\Delta\phi_\mathrm{off}$. In doing so we drastically
(stuff about going from a field to a single local variable in the response function)
reduce the number of variables in the response function and as a result introduced variance into the relationship.

% talk about generating $\zeta$ in the relaxation stage.
% define what is meant by the relaxation stage
% discuss how this is part of a more generally applicable framework. While the specifics of the generation of $\Delta\zeta$ during $\Delta V$ may be complicated by model dependent details, the relaxation stage is
% I think we can say the relaxation stage holds some memory of the state of the system after the potential feature.
% How we deal with the relaxation stage is agnostic to how the stressed state was produced (I still am not sure exactly how to define a stressed state in this context).

% The key points are that the generation of $\Delta\zeta$ during the relaxation phase is generic. In this system the relaxation is fairly simple, the prepared state has a nonlinear correlation of peaks which translate to peaks in $\Delta\zeta$.

% Use the language of response functions. Specifically the asymptotic $\Delta\zeta - Delta\zeta_{off}$ is a response function to the deformed state $z_{off}$. That is a general description in which the role of $\Delta V$ is to prodive a distribution of deformed states $z_{off}$. What we do here is to project that response function onto a single variable (reducing the number of DoF while introducing a variance to the response).
% The other response function we bring up is $\Delta\phi_{off}(\chi_{off})$.

\textcolor{red}{[Put in a part comparing with $\Delta\phi$ since this provides a tighter correlation.]}
\Fdeltaps
\Fdeltaphizetacorr

