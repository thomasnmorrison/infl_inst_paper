% Document for results section

\section{Results} \label{sec:results}
\marginpar{
  Check about how the attractor is defined, ie as a single trajectory or as a field configuration.
}
\marginpar{
  Should things be worded in terms of an $\epsilon$ historesis, ie trajectories picking up differing $\epsilon$ histories as they are kicked off and return to the attractor? We only have $\epsilon$ (in terms of $-\dd \ln(H)/\dd \alpha$ directly) as an average over the lattice.

  There is also the particle creation perspective which is more or less direct: start with free field, turn interaction on/off, compare result to free field. 
}
We can think of this system as an in/out process experiencing an interaction as the fields pass through the nonzero region of $\Delta V$.
The in state has the system evolving in the $V_0$ potential with trajectories tracking towards the attractor.% on superhorizon scales.
With the transverse symmetry breaking the system departs the attractor, trajectories diverge and pick up differring inflationary histories as they pass through $\Delta V$.
The out state has the transverse symmetry restored, trajectories displaced by $\Delta V$ again track towards the attractor.

The effect of this process is to leave an imprint of the local variations to the inflationary history of trajectories passing through $\Delta V$, which we measure by integrating $\zeta$ along each trajectories.
The effect can be isolated by comparing to $\zeta$ for an identical in state which is not subject to the symmetry breaking, but instead is allowed to follow the attractor directly to the out state.
The difference of $\zeta$ between these two systems $\Delta\zeta$ partially isolates the NG.

[Place holder for organization of this section] The remainder of this section discusses the details of this in terms of field dynamics, modification to the mean inflation parameters, sourcing of $\zeta$, and the NG of $\Delta\zeta$.

\subsection{Field Dynamics} \label{sec:field dynanics}
% Start by explaining in broad terms the field dynamics
% Key points in field dynamics:
% symmetry breaking with initial linear growth of transverse fluctuations (squeezed state)
% effective horizon receives an effective mass correction and forms a condensate
% nonlinear $\chi$ - $\phi$ interactions have statistical homogeneity broken by the condensate realization
% particle production

% Plots to include:
% (Cross)Spectra: plot of the spectra and cross spectra vs time. overplot with the instability band and horizon.
% Fluctuation operator determinant: use this plot to show when nonlinearity occurs and to choose time slices to show in other plots, before during and after nonlinearity. Show where in the potential these slices are taken.
% Real space slices: show surface plots of 2d slices of the fields at the times picked out in the fluctuation operator determinant plots.
% Phase space projections: show projections of the phase space corresponding to the same time slices

% Talk about the linear instability and the break down with the formation of the condensate
% Give spectrum plot to show linstability band (compare to linear instability band)
% Give C plot to show when nonlinearity occurs (mark the nonlinearity times)
% Discuss particle production?
% Give realspace slices of fields at marked times. Here talk about the formation of the condensate and the idea of fluctuation condensate split.
% Phase space projections: show projections of the phase space corresponding to the same time slices. This shows the correlations that build up during the nonlinear interactions.

With the symmetry breaking term $\Delta V$, there is a transverse instabilty around $\chi=0$.
Within the instability band the growth of $\chi$ fluctuations is initally described by linear dynamics, however as fluctuations continue to amplify the system experiences the onset of nonlinearity.

\marginpar{
  I'm not sure of the best way to explain the transition between linear and nonlinear evolution. The straight forward answer is linear evolution is seperable in terms of Fourier modes, and as the modes grow mode-mode coupling becomes significant.
  We are seeing the the breakdown of linearity with the formation of a $\chi$ condensate that carries fluctuations into regions of field space where potential interactions are not dominated by the interaction with the mean field. 
}

% How to talk about the $\chi$ condensate
The large amplitude of the $\chi$ condensate allows the fluctuations to explore nonlinear parts of the potential.
A course graining of the $\chi$ fluctuations can be viewed as a condensate which carries fluctuations into nonlinear portions of the potential.
\marginpar{
  The realization of the condensate breaks the statistical homogeneity of the system so leads to a breakdown of the assumptions in perturbations around a homogeneous background.
  The consensate is a real space feature (features amplitude growth and decay as opposed to wavelike behaviour), so nonlinearity in this system will tend to show up in localized regions where the condensate has brought the fluctuations out of the linear regime.
}

[Part about the back reaction on $\phi$]

[Part about the phase space and relate to figure]

\Fspec

\Fspecdet

% Identifying non-linearity
We would like to construct a test for the onset on nonlinearity.
To do this we start with the observation that for linear dynamics a system is seperable in terms of it's Fourier modes.
This allows Louville's theorem can be applied to each Fourier individually, so for linear dynamics the phase space volume for each Fourier mode is conserved.
This implies the following quantity is conserved during linear evolution.
\begin{equation}
  C(\phi^A,\Pi^A) =
  \det\Re\left[
    \begin{matrix}
      |\tilde{\phi}_k^2| & \tilde{\phi}_k^*\tilde{\Pi_{\phi, k}} & \tilde{\phi}_k^*\tilde{\chi}_k & \tilde{\phi}_k^*\tilde{\Pi}_{\chi, k} \\
      \tilde{\Pi}_{\phi,k}^*\tilde{\phi}_k & |\tilde{\Pi}_{\phi, k}|^2 & \tilde{\Pi}_{\phi,k}^*\tilde{\chi}_k & \tilde{\Pi}_{\phi,k}^*\tilde{\Pi}_{\chi, k} \\
      \tilde{\chi}_k^*\tilde{\phi}_k & \tilde{\chi}_k^*\tilde{\Pi}_{\phi, k} & |\tilde{\chi}_k|^2 & \tilde{\phi}_k^*\tilde{\Pi}_{\chi, k} \\
      \tilde{\Pi}_{\chi,k}^*\tilde{\phi}_k & \tilde{\Pi}_{\chi, k}^*\tilde{\Pi}_{\phi, k} & \tilde{\Pi}_{\chi,k}^*\tilde{\chi}_k & |\tilde{\Pi}_{\chi, k}|^2
    \end{matrix}.
  \right]
%  \det\Re\left[
%    \begin{matrix}
%      |\phi_k^2| & \phi_k^*\Pi_{\phi, k} & \phi_k^*\chi_k & \phi_k^*\Pi_{\chi, k} \\
%      \Pi_{\phi,k}^*\phi_k & |\Pi_{\phi, k}|^2 & \Pi_{\phi,k}^*\chi_k & \Pi_{\phi,k}^*\Pi_{\chi, k} \\
%      \chi_k^*\phi_k & \chi_k^*\Pi_{\phi, k} & |\chi_k|^2 & \phi_k^*\Pi_{\chi, k} \\
%      \Pi_{\chi,k}^*\phi_k & \Pi_{\chi, k}^*\Pi_{\phi, k} & \Pi_{\chi,k}^*\chi_k & |\Pi_{\chi, k}|^2
%    \end{matrix}
%    \right]
\end{equation}

\marginpar{
  For relating to what people are familiar with we should point out the relation between the onset of nonlinearity and particle production. In that framing we are calculating the $\Delta\zeta$ signal associated with a particle production event, which I think is an idea people are familiar with at least in the context of the bispectrum.
}

Figure \ref{fig:specdet} shows a plot of $C(\phi^A,\Pi^A)$ which can be used to identify characteristic times during the evolution.
With the onset of nonlinearity the fields are no longer completely described, in the statistical sense, by the (cross)spectra.
The nonlinear interactions can be visualized by the correlations they impose in phase space, see figure \ref{fig:phasespace}.
When we consider the NG of $\zeta$ sourced during this process it will also be informative to view the evolution of the fields in real space, see figure \ref{fig:slice}.

\Fphasespace

\Fslice

\marginpar{
  Still need some words explaining the significance of figures \ref{fig:phasespace} and \ref{fig:slice}.
  I think the things to point out are the squeezing of $P(\chi,\Pi^\chi)$, the particle production, and the back reaction in $P(\phi,\chi)$.
  And the formation of the $\chi$ condensate with the realspace backreaction onto $\phi$.
}

\subsection{Average Inflation parameters} \label{sec:infl}
% Discuss things like energy fractions and the slow roll parameters here.

%\subsection{Sourcing $\zeta$} \label{sec:sourcing zeta}
% Explain the regimes of what source term is active

% Plots to include:
% Have the fluctuation determinant with verical lines denoting time slices before, during, and after nonlinearity.
% For each of those time slices show a $\dot{\zeta}$ vs fld histogram.
\marginpar{
The structure formation in $\phi$ will prove to be an important part in generating the $\Delta\zeta$ signal.
This sources spatially localized modification to the $\epsilon$ history of the trajectory
(The whole story is about modifying the $\epsilon$ history of the trajectories, so should emphasis this point in the $\Delta\zeta$)
}

\subsection{$\zeta$ NG} \label{sec:zeta ng}
% This section will cover the peak statistics and realspace plots of $\Delta\zeta$

Both Gaussian and NG components ($\zeta_G$ and $\zeta_{NG}$ respectively) are present in $\zeta$.
For the purposes of characterizing the NG compontent, $\zeta_G$ is a noise term which we would like to suppress, effectively increasing our signal to noise ratio for NG.

It is unclear how to make an exact identification of $\zeta_G$ in any one realization. We can however, identify a correlated quantitiy by making a comparison to the baseline potential.  
Analogous to our splitting of the potential $V = V_0 + \Delta V$, we define the split $\zeta = \zeta_0 + \Delta\zeta$ where $\zeta_0$ is found using a simulation run from identical initial conditions, but using the baseline potential $V_0$ in place of the full potential $V$.
%$\Delta\zeta$ is then identifiable as the response in $\zeta$ to the deformation of the potential $\Delta V$.
Provided $\zeta_0$ and $\zeta_G$ are correlated and $\zeta_0$ is not significantly NG, $\Delta\zeta$ will increase the signal to nosie ratio for identifying $\zeta_{NG}$.

$\Delta\zeta$ is identifiable as the response of $\zeta$ to the deformation of the potential $\Delta V$ and is the key quantity of interest in our search for NG.
Real space slices of $\Delta\zeta$ are shown in % reference to figure
Visual inspection of $\Delta\zeta$ in real space shows spatially localized prominences, suggesting peak statistics to be a suitable measure of the NG.
% reference to figure

\marginpar{
  We should mention the caveat that although peak statistics seem to be an appropriate measure of NG in $\Delta\zeta$ that does not necessarily mean it will be an appropriate measure of NG in the total $\zeta$.
}

\subsection{Sourcing NG}
% This subsection is to tie together the field dynamics and NG of $\Delta\zeta$.

% Things I want to cover:
% The idea that trajectories through a deformed potential pick up different $\epsilon$ histories;
% $\Delta\zeta$ is produced most strongly along those trajectories which undergo the largest $\chi$ excursions;
% There are two main phases in the $\Delta\zeta$ production: when the trajectory goes through the $\chi$ excursion, and when $\dot{\phi}$ is slowed with the symmetry restoration;
% This is the most logical place to cover the parts of $\Delta\zeta$.

% Figures I want to include:
% 2d historgam showing $\dd\Delta\zeta/\dd\alpha$ vs $\phi^A$ and $\Pi^A$.

Having seen both the field dynamics and $\zeta$ NG in this system we now turn to the relation between the two.

\Fzetaslice
\Fzetapeak
\Fdzetaphasespace
\Fzetacondps
